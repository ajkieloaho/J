\section*{Preface}
\label{preface0}

The \textbf{J} software is the succsssor of the JLP software which has been used as the optimizer in the Mela software
and in the Norwegian GAYA planning software which was then called GAYA-JLP.
\textbf{J}LP was published 1992 and Finnish Forest Research Institute (Metla) started to distribute it
version 0.9.3 was published in August 2004. The GAYA group was involved in the development and
started to use \textbf{J} immediatemently in the planning system called thereafter GAYA-J. Mela has not
included \textbf{J} as
as the optimizer, but it is possible to pull out simulated schedules from Mela and then
define and solve linear programming problems using J. Simosol started to use \textbf{J} as an
optimizer from the beginning.
\textbf{J} version 2.0. published in 2013 made it possible optimize simultatenously forestry and transportations to factories
and factory production.

In 2015 Metla and
two other institutes were merged into Institute of Natural Resources Finland (Luke).
When my retirement was approaching, I suggested Luke that \textbf{J} would be published as
an open source software. The leaders of Luke told that this a good idea, and this is so important that
Luke cannot let me take care of the publication but the leaders wanted to publish \textbf{J}
themselves. This started a frustrating and humillating process which lasted until 1.10. 2021 when
I got the permission to publish the software and manuals as I like. The development of JLP and \textbf{J} is
decribed in some detail in section 'J story'.

When Luke did not prevent the development of \textbf{J} any more, I have rewritten all central parts
of J. \textbf{J} is now a 64-bit application. And \textbf{J} is now finally published as an open source software
at github.com/juhalappi/J. New versions are published there all the time (J contains still many bugs).

\textbf{J} is utilizing linear programming subroutines of Roger Fletcher.
Even Bergseng from NMBU helped in the first version of J. The role of Reetta Lempinen was essential
when factories were included. She adviced how to use Github for the distribution. She has
also helped to test the new version, and she participated in the writing of the manual before
I developed new tools for the making of manuals. Victor Strimbu adviced how to change \textbf{J} into 64-bit and
we are developing together solutions for using \textbf{J} with the new simulator of GAYA.
Lauri Meht\"atalo helped to convince the leaders of Luke that Luke's interests do not require that
they continue preventing the publication and development of J. With Lauri, we have developed a solution for using the optimizer of \textbf{J} from
R. More advanced solutions will come later. I thank my grand daughter Ella
for the permission to use the portrait she drawed for my 70 yr birthday.

The web - launch of J3.0 was 22.4. 2022. The J-story was then part of the preface. Following the advice
of Heikki Smolander, I separated this preface and the J-story at 25.5. 2022.

\textbf{J}uha Lappi
\section{Introduction}
\label{intro}
The \textbf{J} software can be used as is, i.e. using exe files. The User guide concentrates
on using binary files, but some reference is also made to additional possibilities offered
by the open \textbf{J} code.
\subsection{Using \textbf{J} exe files}
\label{intro1}
\textbf{J} is a general program for many different tasks. In one end, \textbf{J} is a programming language which
can be used to program several kind of applications and tasks, starting from computing 1+1, either
at \textcolor{Red}{sit>} promt or inside a trasformation object.
In the other end, it contains many functions which
can do several tasks in statistics, plotting figures,
deterministic and stochastic simulation and linear optimization. Forest planning with factories
using \textcolor{VioletRed}{jlp}() function can take several hours.
The general \textbf{J} functions and \textbf{J} trasformations can be combined in many ways.

There are several alternatives for doing general mathematical and statistical
computations available in J, most prominent being R. For  users of R, the most interesting
functions in \textbf{J} are evidently the linear programming functions which utilize the the structure
of typical forest managment planning problems. The forest management planning can now
combined with factories. R users can do general data management in R and use
\textbf{J} only for linear programming. But after learning basics of J, it may be more
straightforward to do also data management in J.

\textbf{J} can now be used for general matrix computations. I have included in \textbf{J} all matrix functions of
Matlab which I found useful in a consulting project. \textbf{J} uses Gnuplot for making figures.
It is straigtforward to extend these graphics functions. \textbf{J} can be used an
interface to Gnuplot graphics.  \textbf{J} contains many tools to deal with classified data.
\subsection{Developing own \textbf{J} functions in Fortran}
\label{intro2}
The user's of \textbf{J} can utilize the open source of \textbf{J} in principally two different ways.
Either the user can develop new versions of existing \textbf{J} functions or the user can can make new functions.
It both cases the user should make new developments using so called own functions, which can
be independently of the main \textbf{J} functions. When modifying an existing \textbf{J} function, the user should make a copy of the
function under a different name. Then the old and new versions
can exist simultaneously in the function
space of J. It is very easy to add new functions in \textbf{J} and even more easy to add new options.

When deloping a new methods in J, it is possible to first use the \textbf{J}
script language to make developments.
Then the user can make an own function where the method is written in Fortran to make the method faster.
When writing new methods in Fortran, the user can concentrate on essential parts of the method, and utilize
the standard data management services provided by J.
\section{Installing \textbf{J} from Github}
\label{git}
The Github ditribution was made under the guidance of Reetta Lempinen. This section decribes the
folders of the Github distribution and also how to
install \textbf{J} and Gnuplot.
\subsection{Git package}
\label{gitpack}
The \textbf{J}  package can be loaded by pressing load zip button under the green
code button in the right side of the page github.com/juhalappi/j.  The package
contains the following components

\begin{itemize}
\item[\textbf{J}] Data: \hspace{1cm} folder for data files
\begin{itemize}
\item[\textbf{J}] test.cda \hspace{1cm}example unit data file for small \textcolor{VioletRed}{jlp}() example
\item[\textbf{J}] test.xda \hspace{1cm}example schedule file for small \textcolor{VioletRed}{jlp}() example
\item[\textbf{J}]  test.inc \hspace{1cm}include file for the \textcolor{VioletRed}{jlp}() test

\end{itemize}
\item[\textbf{J}] J\_R: \hspace{1cm} using \textbf{J} from R, courtesy of Lauri Meht\"atalo.

\begin{itemize}
\item[\textbf{J}] j.par \hspace{1cm}decaul include file for starting \textbf{J}
\item[\textbf{J}] JR\_0.0.tar.gz \hspace{1cm} File needed to use Fortran subroutines in R
\item[\textbf{J}] testr.cda
\item[\textbf{J}] test.xda
\item[\textbf{J}] testr.inc \hspace{1cm} include file for an jlp-example

\item[\textbf{J}] testr.inc


\end{itemize}

\item[\textbf{J}] Jbin:\hspace{1cm} binary .exe files and dll files
\begin{itemize}
\item[\textbf{J}] j3d.exe \hspace{1cm} Debug version of J, release version is
provided after some testing period
\item[\textbf{J}] jindent.exe \hspace{1cm} indentation
\item[\textbf{J}] jmanual.exe \hspace{1cm} makes the latex code file jmanual.tex
\item[\textbf{J}] jpre.exe \hspace{1cm} the precompiler which generates the code
for accessing variables in module
\item[\textbf{J}] dll:\hspace{1cm} libgcc\_s\_seh-1.dll, libgfortran-5.dll, libquadmath-0.dll and
libwinpthread-1.dll which must be available in the path. e.g., in the same folder as the exe

\end{itemize}
\item[\textbf{J}] Jdoc\_demo \hspace{1cm}documents and include file for running examples from User's guide
\begin{itemize}
\item[\textbf{J}] hyvonenetal2019.pdf \hspace{1cm} A paper utilizing the factory optimization.
\item[\textbf{J}] J3.0\_userguide\_2021.pdf \hspace{1cm} old version of the manual, explains better some functions thant j3.pdf
\item[\textbf{J}]	J3.pdf \hspace{1cm} Users guide made with Latex
\item[\textbf{J}]	 J3\_setup\_development.docx \hspace{1cm} not up-to-date manual for developers
\item[\textbf{J}] jexamples.inc \hspace{1cm}include file which can be used to run all examples in the manual
and which is generated with jmanual.exe
\item[\textbf{J}] jlp92.pdf \hspace{1cm}Manual of old JLP which explains the theory behind the jlp algorithm
\item[\textbf{J}] lappilempinen.pdf \hspace{1cm} Paper explaining the theory behing factory optimization.
\end{itemize}
\item[\textbf{J}] Jmanual: \hspace{1cm} Source files for making Latex code for
the manual and the include file for running examples
\begin{itemize}
\item[\textbf{J}] jmanual.f90 \hspace{1cm}source for making the Latex code and jexamples.inc
\item[\textbf{J}] jmanual.tex \hspace{1cm}Latex code genrated with jmanual.exe
\item[\textbf{J}] jsections.txt \hspace{1cm}describes such manual sections not in source files
\item[\textbf{J}] jsections2.txt \hspace{1cm}tells in what order sections foun in jsections.txt
and source files are put into the manual and what is the level of the sections
\item[\textbf{J}] main.tex  \hspace{1cm}Preample code containing Latex definitions
\item[\textbf{J}] Makefile\_debug \hspace{1cm}Makefile for making jmanual.exe
\end{itemize}

\item[\textbf{J}] Source :\hspace{1cm} source code before precompilation
\begin{itemize}
\item[\textbf{J}] fletcherd.for \hspace{1cm}Fletchers subroutines turned into double precision
\item[\textbf{J}] j.f90 \hspace{1cm}code for \textbf{J} functions
\item[\textbf{J}] jlp.f90 \hspace{1cm}code for linear programming
\item[\textbf{J}] j.main \hspace{1cm}main prorgam for calling \textbf{J} when used as is, if \textbf{J} is used as
a subroutine then this must be made a subroutine
\item[\textbf{J}] jmodules.f90 \hspace{1cm}data structure definitions
\item[\textbf{J}] jutilities.f90 \hspace{1cm}subroutines for handling objects etc.
\item[\textbf{J}] jsysdep\_gfortran.f90 \hspace{1cm}system dependent routines
\item[\textbf{J}] matsub.f \hspace{1cm}subroutines obtainde from other sources, e.g. from Netlib
\item[\textbf{J}] other subroutines for setting up users own  function

\end{itemize}
\item[\textbf{J}] Source2 :\hspace{1cm} source code files after precompilation
in addition to files in Source (such files which are not precompiled are put in both folders)
\begin{itemize}
\item[\textbf{J}] makefile\_debug \hspace{1cm}makefile for making debug- version
\item[\textbf{J}] makefile\_release\hspace{1cm} makefile for making release- version
\end{itemize}
\item[\textbf{J}] LICENSE:the license file
\item[\textbf{J}] README.m :readme file

\end{itemize}

\subsection{Loading the package}
\label{gitload}

The installation is here described from the viewpoint of an user
who just want to use J, not develop (yet).

The load zip button loads file J-master.zip. Copy this file into a proper folder.
Let it be Jgit. Clicking the J-master.zip icon shows folder J-master in
7Zip (if this is installed). Clicking unpack 7zip unpacks it to the folder Jgit.
Too many folder levels are avoided, if everyhing in folder J-master is copied directly in
folder Jgit.

The j3d.exe is the debug version of J3. The debug version should be
used as long as the most memory acces probems are solved.
Let us assume that the working folder is jtest.
\section{Starting to us \textbf{J} and Gnuplot}
\label{start}
This section tells how to install \textbf{J} so that the user can start running the example
\subsection{Istalling Gnuplot and \textbf{J}}
\label{install}
Let us then install Gnuplot. Go to page gnuplot.info, and select there download,
and in the download page select green 'download latest version', which loads
'gp543-min64-mingw.exe' (or similar), and if you let it install with usual yes-next procedures
gnuplot is installed  in folder c:\program files\gnuplot. Open then command prompt.
Give command gnuplot. If the compuler says that it is not operable command,
this means that gnuplot is not properly on the PATH of the system. I had some problems to get the PATH to work
properly. When I edited the Environmental variable for system and added new item by adding C:\program files\gnuplot\bin having
space between 'program' and 'files', I did get gnuplot to the PATH, and PATH command showed that it was
actually two times, but anyhow it worked.

Also folder Jgit/Jbin  should be put on the PATH. This happens by adding item C:\jgit\Jbin to the
environmental system variable PATH.
Restart then the computer.

If you have problems  with the environmental variable PATH \textbf{J} and Gnuplot can be put
into path by defining file jpath.bat  in the working directory containing the lines:\\
PATH =\%PATH\%;C:\program files\gnuplot\bin\gnuplot\\
PATH =\%PATH\%;C:\Jgit\Jbin\\
When starting to work in the working directory, launch Command prompt, got ot the
working directory using cd commands, and when in the working directory
write jpath which the puts both j3d and Gnuplot temporalrily into the path.

\subsection{Running examples}
\label{runex}

Copy then jexamples.inc and j3.pdf  from jdoc\_demo folder into the Jtest folder.
It is not wise to start
working in the Jdoc\_demo folder, because if you load a new version of \textbf{J} in a similar way into Jgit folder
and allow the computer replace existing folders with the same name,
you would loose the work done in Jgit folder.

Open the command prompt and move to the directory 'jtest' by 'cd' commands.
It is possible to use \textbf{J} also directly, but its necessary to use \textbf{J} through the command prompt
at this testing phase, bacause then, if \textbf{J} crashes, the error messages do not disappear.
In first time, you may want to change
the colors etc of the commant prompt (color may take effect only after
closing command prompt and reopening it). It may be reasonable to tick all editing
properties under the properties button of the command prompt.

If you would like that when starting j3d in this folder, \textbf{J} immediately starts with examples,
make file j.par to folder jtest and write
to it

\textcolor{Red}{;incl}(jexamples.inc)

It is possible to write to the fisrts line of j.par

*3000

which would mean that \textbf{J} would generate intially 3000 named objects.


Then start j by giving command j3d at the command prompt.
If you have not done j:par file, write at the \textcolor{Red}{sit>} prompt:  \textcolor{Red}{;incl}(jexamples.inc).
Alternatively you can diretly start in the jexamples.inc by launching \textbf{J} by\\
\textbf{J}3d jexamples.inc

\textbf{J} will then print all shortcuts available. You can run examples one by one by giving example shortcuts
individually, or you can run all examples by shortcut ALL. If you give shortcut ALL, \textbf{J} asks whether
a \textcolor{Red}{;pause} is generated after each example ('pause after each example(1/0)>').
Even if \textcolor{VioletRed}{pause}() is not generated after each example, it is generated after each plot.
This can be prevented by giving at \textcolor{Red}{sit>} promt 'fcont=1' before giving ALL. Examples in the ALL section can anyhow be started
at any point by putting label ';current:' to any point after label ;ALL: and giving shortcut
'current'. This is also handy when examples have errors which break the execution. If the errors are made
intentionally to demonstrate error situations, the interruption of excution after these intentional errors are
prevented so that the jexamples.inc have command Continue=1 before the intentional error. Theafter the normal
error handling is put on by command Continue=0.

It is useful to keep the manual open and follow simultaneously the manual and the execution of examples.



\subsection{\textbf{J} functions}
\label{jfuncs0}
The structure of \textbf{J} functions is easiest to explain using examples.
For instance a data object can be created with \\
data1=\textcolor{VioletRed}{data}(\textcolor{blue}{read->}(x1...x3),\textcolor{blue}{maketrans->}mt,\textcolor{blue}{in->}) \\
Here \textcolor{teal}{data1} is the output. \textcolor{blue}{read->} is an option which tells what are the variable names in the data.
Options can transmit object names or numeric values, here \textcolor{blue}{read->} transmits object names.
maketras->mt tells that for each observation the trasformations defined in the trasformation
object mt are first computed and the output variables are included in the data object. Option
\textcolor{blue}{in->} tells that the data are read from the following input paragraph.

Let us then look at a function with arguments and codeoptions. Fuction \\
\textcolor{VioletRed}{stat}(x1,x3,\textcolor{blue}{min->},\textcolor{blue}{max->},\textcolor{blue}{filter->}(x2.gt.3)) \\
Now the arguments x1 and x2 tell that stasistics are computed for variables  x1 and x2.
The options \textcolor{blue}{min->} and \textcolor{blue}{max->} tell that the minimum of \textcolor{teal}{x1} is stored in variable \textcolor{teal}{x1}\%min and maximum in
variable \textcolor{teal}{x1}\%max.

 Option \textcolor{blue}{filter->} is a code option is trasmits a piece of code. \textcolor{blue}{filter->}(x2.gt.3) tells that
only such observations are accepted for which \textcolor{teal}{x} is greater than 3. All operations
which look like logical operations are actually numeric operations which produce
value 1 for TRUE and value 0 for FALSE. Other options are
interpreted only once when entering the function, but the code within a code option is here
computed repeatedly for all observations. If the code needed in a code option is long or must
be computed using several lines, the code can be put into a transformation object, and e.g.
\textcolor{blue}{filter->}(tr(a).gt.3) is done so that transformation \textcolor{teal}{tr} is first
computed and \textcolor{teal}{tr}(\textcolor{teal}{a}) trasmits the value of variable
\textcolor{teal}{a} after the transforamation is computed.

\textcolor{VioletRed}{stat}() function does not produce an output object.
For such function which produce output but the name of the output is not given, the name of the output
is Result. Try e.g. '1+1;' If an output is produced, the it is printed if the code line
ends with ';' or ';;' . See section \ref{joperation} how the printing is controlled with the
variable \textcolor{teal}{Printoutput}.

For an option which basically transmits on/off information, it may be necesssary to
control wheter the option is on by a control variable. For instance, in plotting figures
option \textcolor{blue}{continue->} indicates that the execution continues without \textcolor{VioletRed}{pause}()
when the figure is plotted. The option \textcolor{blue}{continue->}fcont that the option is on when variable
fcont has non-zero value.

The value of an option needs to be put into parenthesis when it is not a single
constant, object name or simple function call as in\\
fi=\textcolor{VioletRed}{draw}(\textcolor{blue}{func->}\textcolor{VioletRed}{sin}(x),\textcolor{blue}{x->}x,\textcolor{blue}{xrange->}(0,10)).
Here there arguments can be computed using computations. For istansce \\
fi=\textcolor{VioletRed}{draw}(\textcolor{blue}{func->}\textcolor{VioletRed}{sin}(x),\textcolor{blue}{x->}x,\textcolor{blue}{xrange->}(0,\textcolor{VioletRed}{max}(10*ran(),3)))\\
would compute a random upper range for the x-axes. Note here \textcolor{blue}{x->}x means that the name
of the x-axes variable is x, which used as the argument in \textcolor{VioletRed}{sin}([x[)

\textbf{In a function all the arguments must be before all options.}

The computations are done going through the parse tree form right to left.
This should be taken into account when getting the values of option arguments
using transformations, for instance if a and b are computed in transformation tr
then \textcolor{blue}{filter->}(tr(a).gt.b) does not work in the intended way but \textcolor{blue}{filter->}(a.gt.tr(b)) does.
The reason for this nonstandard direction for going through
the parse tree is that the options mus be computed before entering the function, and it had
made the processing of the parse tree more complicated.

Some of the errors made by the user are noted at the time the function is interpreted, some
errors are noted after entering the function. Some errors are not recognized and
unintended results are obtained. In some cases the system will crash.


The general (non arithmetic) \textbf{J} functions are used either in statements

func(arg1,…,argn,opt1->value1,….,optm->valuem)

or

output=func(arg1,…,argn,opt1->value1,….,optm->valuem)\\
If there is no output for a function in a statement, then there can be three different cases:
i) The function does not produce any output (if an output would be given, then \textbf{J} would just
ignore it\\
ii) The function is producing output, and a default name is used for the output (e.g. Result
for arithmetic and matrix operations, Figure in graphic functions).\\
iii) The function is a sub expression within a transformation consisting of several parts including
other function or arithmetic operations. Then the output is put into a temporary unnamed
object which is used by upper level functions as an argument (e.g. a=\textcolor{VioletRed}{inverse}(b)*t(c))
If the value of an option is not a single object or numeric constant, then it must be enclosed in
parenthesis.\\
It is useful to think that options define additional argument sets for a function. Actually
an alternative for options would be to have long argument lists where the position of an
argument determines its interpretation. Hereafter generic term 'argument' may refer also to
the value of an option.

When \textbf{J} is interpreting a function, it is checking that the option names and the syntax
are valid, but it is not checking if an option is used by the function. Also when executing the
function, the function is reacting to all options it recognizes but it does not notice if there are
extra options, and these are thus just ignored.

An argument for a \textbf{J} function can be either functional statements producing a \textbf{J} object as its
value, or a name of \textbf{J} object. Some options can be without any argument (indicating that the
option is on).

An essential feature in \textbf{J} functions is that the driver subroutine which is computing
the functions is recursive. This recursion is used when code options SEE? are used to compute
the the output of an code option or when transformation objects are explcitly called
within an transformation object or a function dealing with data is computing \textcolor{blue}{trans->} transforamtion for each
observation.
\\
a = \textcolor{VioletRed}{sin}(\textcolor{VioletRed}{cos}(c)+b) &\textcolor{green}{!\,Usual\,arithmetic\,functions\,have\,numeric\,values\,as}
\\
arguments
\\
here the value of the argument of cos is obtained by 'computing' the
\\
value of real variable c.
\\
\\
\textcolor{VioletRed}{stat}(D,H,\textcolor{blue}{min->},\textcolor{blue}{max->}) &\textcolor{green}{!\,Here\,arguments\,must\,be\,variable\,names}
\\
\\
\textcolor{VioletRed}{plotyx}(H,D,\textcolor{blue}{xrange->}(\textcolor{VioletRed}{int}(D\%min,5), \textcolor{VioletRed}{ceiling}(D\%max,5))) &\textcolor{green}{!arguments\,of}
\\
the function are variables, arguments of option \textcolor{blue}{xrange->} are numeric
\\
values
\\
\\
c = \textcolor{VioletRed}{inverse}(h+\textcolor{VioletRed}{t}(g)) &\textcolor{green}{!\,The\,argument\,can\,be\,intermediate\,result\,from}
\\
matrix computations.
\\
If it is evident if a function or option should have object names or values as their arguments, it
\\
is not indicated with a special notation. If the difference is emphasized, then the values are
\\
indicated by val1,…valn, and objects by obj1,…,objn, or the names of real variables are
\\
indicated by var1,…,varn.
\\
There are some special options which do not refer to object names or values. Some options
\\
define a small one-statement transformation to be used to compute something repeatedly.
\\
\\
\textcolor{VioletRed}{stat}(D,H,\textcolor{blue}{filter->}(\textcolor{VioletRed}{sin}(D).gt.\textcolor{VioletRed}{cos}(H+1)) &!
\\
only those observations are
\\
accepted which pass the filter
\\
\\
\textcolor{VioletRed}{draw}(\textcolor{blue}{func->}(\textcolor{VioletRed}{sin}(\$x)+1),\textcolor{blue}{x->}\$x,\textcolor{blue}{xrange->}(0,10,1)) &\textcolor{green}{!\,the\,\textcolor{blue}{func->}\,option}
\\
transmits the function to be drawn not a single value.
\\
\subsection{Structure of general \textbf{J} functions}
\label{jfunc}
The general (non arithmetic) \textbf{J} functions are used either in statements

func(arg1,…,argn,opt1->value1,….,optm->valuem)

or

output=func(arg1,…,argn,opt1->value1,….,optm->valuem)

If there is no output for a function in a statement, then there can be three different cases:
\begin{itemize}
\item[\textbf{J}] The function does not produce any output (if an output would be given, then \textbf{J} would just
ignore it

\item[\textbf{J}] The function is producing output, and the default name {Result} is used for the output
for arithmetic and matrix operations.
\item[\textbf{J}] The function is a sub expression within a transformation consisting of several parts including
other function or arithmetic operations. Then the output is put into a temporary unnamed
object which is used by upper level functions as an argument (e.g. a=\textcolor{VioletRed}{inverse}(b)*t(c))
\end{itemize}
If the value of an option is not a single object or numeric constant, then it must be enclosed in
parenthesis.

\begin{note}
It is useful to think that options define additional argument sets for a function. Actually
an alternative for options would be to have long argument lists where the position of an
argument determines its interpretation. Hereafter generic term 'argument' may refer also to
the arguments of an option.
\end{note}
\begin{note}
When \textbf{J} is interpreting a function, it is checking that the option names and the syntax
are valid, but it is not checking if an option is used by the function. Also when executing the
function, the function is reacting to all options it recognizes but it does not notice if there are
extra options, and these are thus just ignored.
An argument for a \textbf{J} function can be either functional statements producing a \textbf{J} object as its
value, or a name of \textbf{J} object. Some options can be without any argument (indicating that the
option is on).
\end{note}
\begin{example}[jfuncex]Examples of J-functions\\
\label{jfuncex}
a = \textcolor{VioletRed}{sin}(\textcolor{VioletRed}{cos}(c)+b) \textcolor{green}{!\,Usual\,arithmetic\,functions\,have\,numeric\,values\,as\,arguments}\\
\textcolor{green}{!\,here\,the\,value\,of\,the\,argument\,of\,cos\,is\,obtained\,by\,'computing'\,the\,value\,of\,real\,variable\,c.}\\
Dm=\textcolor{VioletRed}{matrix}(\textcolor{blue}{do->}(0.1,40))\\
nob=\textcolor{VioletRed}{nrows}(Dm)\\
e=\textcolor{VioletRed}{matrix}(nob)\\
e=\textcolor{VioletRed}{rann}()\\
Hm=0.5+Dm**0.7+e\\
dat=\textcolor{VioletRed}{newdata}(Dm,Hm,\textcolor{blue}{read->}(D,H))\\
\textcolor{VioletRed}{stat}(D,H,\textcolor{blue}{min->},\textcolor{blue}{max->}) \textcolor{green}{!\,Here\,arguments\,must\,be\,variable\,names}\\
\textcolor{VioletRed}{plotyx}(H,D) \textcolor{green}{!arguments\,of\,the\,function\,are\,variables}\\
h=\textcolor{VioletRed}{matrix}(5,5);\\
h=\textcolor{VioletRed}{rann}();\\
g=\textcolor{VioletRed}{matrix}(5,\textcolor{blue}{do->}5);\\
c = \textcolor{VioletRed}{inverse}(h+\textcolor{VioletRed}{t}(g)); \textcolor{green}{!\,The\,argument\,can\,be\,intermediate\,result\,from\,matrix\,computations.}
\end{example}
\subsection{\textbf{J} objects}
\label{objintro}
\textbf{J} objects have a simple yet efficient structure. Each object is associate with a double precision
variable, two integer vectors, one single precision vector, one double precision vector,
one vector of characters and one vector of text lines. All vectors are allocated dynamically.
There are several object types which store data differently in these vectors.
Object can be either simple or compound objects. Compound objects are linked to other
objects which can be used also directly utilizing the standard naming conventions. All objects are
global, i.e. also users can acces all objects even if some predefined objects are locked so that
users cannot change them.

\subsection{System requirement}
\label{system}
The current binary versions of \textbf{J} are developed using Gfortran Fortran 90 compiler in MSYS2 MINGW 64-bit environment
under Windows 10. There are both release and debug versions available.
Binary versions are ordinary console applications. It is recommended that \textbf{J} is used in command
prompt window, so that
if execution of \textbf{J} terminates unexpectedly, the error debugging information, remains visible.
With the debug version the problematic line is indicated.
See chapter ? for more information of error handling.

See J3.0 Development Guide to start develop the software or to add
own functions. The development package contains, in addition to source code for
the standard \textbf{J} software, program Jmanual
which can be used to generate Latex code for the manual, program Jindent to make indentations for
Fortran source files and
a precompiler Jpre which writes necessary Fortran statements to access all globa \textbf{J} data structures.

Figures are made with Gnoplot.
Gnuplot is freely available at


https://sourceforge.net/projects/gnuplot/files/gnuplot/5.4.2/

Download  download gp542-win64-mingw.exe. This will install gnuplot on your windows
system under C:\Program Files\gnuplot\. \textbf{J} will strat Gnuplot automatically when
plotting figures if Gnuplot is on the PATH (see section Installing Gnuplot and J).

\subsection{Exiting \textbf{J}}
\label{exit}
To exit \textbf{J} program and close console window, just give end command:

\textcolor{Red}{sit>}end
\subsection{Typographical conventions}
\label{typo}
In this manual function names are written in red, option names in blue, object types in
capital letters, object names within the text are written in this color: {Object}.
Names like {ob1} are used as generic names for object and names like  {var1}
are generic names for REAL variables.
If there is no output for an operation command line,
the object Result is used as the default output.
if the line ends with a single semicolon ';' or double semicolon ';;', then
the output may be printed depending MIHIN KOHTAAN
In many cases there is no output object, and a possible
output given is ignored. If an explicit output is necessary, then '=' is
put in front of the function name. Notation ‘[=]’ means that the output can be given but
it is not necessary. In most cases the default output Result is then used:
In some cases no output is then generated (this will be indicated).
For functions that return real values or matrices which can be
used directly in arithmetic or matrix operations,
the existence of output is not indicated.

\subsectione{Subroutines obtained from other sources}
\label{license}
The following subroutines are obtained from other sources.
\begin{itemize}

\item[\textbf{J}]  subroutine tautsp used in j\_function tautspline
from Carl de Boor (1978) A practical guide to splines. Springer, New York, p.310-314
No licence restrictions known
distribution:  http://pages.cs.wisc.edu/$\sim$deboor/pgs/tautsp.f

\item[\textbf{J}]  real function ppvalu used in \textbf{J} function value
from Carl de Boor (1978) A practical guide to splines. Springer, New York, p.310-314
No licence restrictions known
distribution: http://pages.cs.wisc.edu/$\sim$deboor/pgs/tautsp.f

\item[\textbf{J}] subroutine interv , used in function ppvalue
from Carl de Boor (1978) A practical guide to splines. Springer, New York, p.310-314
No licence restrictions known
obtained from: http://pages.cs.wisc.edu/$\sim$deboor/pgs/tautsp.f
Lapack matrix routines

\item[\textbf{J}] Several subroutines from www.netlib.org/lapack
licence : http://www.netlib.org/lapack/LICENSE.txt
\end{itemize}
\subsection{Operation of \textbf{J}}
\label{joperation}
In this section the main tools in code devlopment in a project are presented. It is usefule to organize
the project sript into one script file, which conatians swections starting with a lable and ending
with \textcolor{Red}{;return}.
I think that it is more difficult to have several script files.
The example file jexamples.inc is a good eaxample of a script file.
\textcolor{green}{!The\,script\,files\,should\,start\,with\,shortcuts\,for\,each\,section.\,If\,it\,is\,possible\,that}
different versions of the same script file are stored in different names, it is useful to have
as first line something like
this=\textcolor{VioletRed}{thisfile}()//
Then it is not necessary to change anything if the file is stored in a different name.
Thereafter comes the shortcut defintiosn for different sections.
the same
For instance, if the the section
label is //
;thistask://
then the shortcut definition could be
thistaskh='\textcolor{Red}{;incl}(this,\textcolor{blue}{from->}thsitask)'//

The section should end with//
\textcolor{Red}{;return}//
After defining shortcuts for all sections, it is useful to have a shortcut as://
again=\textcolor{Red}{;incl}(this)//
If new sections are added, then one needs to give just shortcut//
again//
and then the new shortcuts will be defined. It does not matter if the earlier shortcuts are redefined.

The last shortcut could be //
current='\textcolor{Red}{;incl}(this,\textcolor{blue}{from->}current)
The label 'current' can be a floating label which is put into the section which is under developed
in place the problems started.

If a \textbf{J} code line, either in the input paragraph defining a transforamtion object
or outside it, ends with ';' or ';;', the the output object of the code line may be printed.
The output of ';'-line is printed if the variable \textcolor{teal}{Printoutput} has value 1 or 3 at the time when the the
code line is computed.
The output of ';;'-line is printed if the variable \textcolor{teal}{Printoutput} has value 2 or 3 at the time when the the
code line is computed.

If a code line within a tranformation has function \textcolor{VioletRed}{pause}('text'), then a pause is generated during which
the user can give any commands except
input programming commands. If the user will press <return> then the exceution continues. If the user
presses 'e' and <return>, the control comes to the \textcolor{Red}{sit>} promt similarly as during an error.

If the line outside the transforasmtion definition paragraph is '\textcolor{Red}{;pause}', then
a similar pause is generated except also input programming commands can be give.

If the variable \textcolor{teal}{Debugtrans} has value 1, them a \textcolor{VioletRed}{pause}() is genertated before each line within
a tranformation object is executed.  If variable \textcolor{teal}{Debugconsole} has value 1,
a '\textcolor{Red}{;pause}' is generated before  the line is excuted. In bot cases the user can
give new values for \textcolor{teal}{Debugtrans} and \textcolor{teal}{Debugconsole}.

What happens when an error is encountered is dependent on the value of variable \textcolor{teal}{Continue}. If \textcolor{teal}{Continue} has
value 0 8teh default case), the control comes into \textcolor{Red}{sit>} prompt when an real error occurs or if an atrficial
error condition is generated with \textcolor{VioletRed}{errexit}(). If \textcolor{teal}{Continue} has value 1 then the computation continues in
the same script file where the error occured. This property is used in file jexaples.inc to demonstrate
possible error conditions so that the computation continues as if no error had occured.

\begin{example}[operexr]Exapmple of operation of J.\\
\label{operexr}

\end{example}

\section{Command input and output}
\label{cominout}
\textbf{J} has two programming levels. First level, called input programming, generates text lines which are then
transmitted to the interpreter which generates code which is the put into transformations sets or
excuted directly. Input programming loops make it possible to generate large number of command lines
in a compact and short form. This chapter describes input programming concepts and commands.
\subsection{Input record and input line}
\label{inpuline}
\textbf{J} reads input records from the current input channel which may be terminal, file or a text object.
When \textbf{J} interprets input lines, spaces between limiters and function or object names are not
significant. In input programming, functions start with ';' which is part of the function name (and
there can thus be no space immediately after ';'). If a line (record) ends with ',' ,'+', '*´, '-',
'(', '=' or with '>', then the next record is interpreted as a continuation record and
the continuation character is kept as a part of the input
line. If a line ends with '>>', then
the nex line is also continuation line, and  '>>' is ignored. All continuation
records together form one input line. In previous version input programming functions operated on input lines
but now they operate on recors. One input record can contain 4096 characters, and an input line can contain also 4096 characters
(this can be increased if needed).
The continuation line cannot start with ‘*’ or ‘\textcolor{green}{!’\,because\,these\,are\,reserved}
to indicate comments.  Note: '/' (division)cannot be used as last character indicating the continuation of the line because it can
be legal last character indicating the end of an input paragraph.

When entering input lines from the keyboard, the previous lines given from the keyboard can no more be
accessed and edited using the arrow keys owing to MSYS2 MSYS environment used to build the exe-file.
To copy text from the \textbf{J} window into the clipboard right-click the upper left icon, select Edit,
and then select Mark. Next click and drag the cursor to select the text you want to copy and
finally press Enter (or right-click the title bar, select Edit, and in the context menu click Copy).
To paste text from the clipboard into the \textbf{J} command line right-click the title bar, select Edit,
and in the context menu click Paste. Console applications of Intel Fortran do not provide copy
and paste using <cntrl>c and <cntrl>v. An annoying feature of the current command window is that it is possible
All input lines starting with '*' will be comments, and in each line text starting with '!' will also
be interpreted as comment (\textcolor{green}{!debug\,will\,put\,a\,debugging\,mode\,on\,for\,interpretation\,of\,the}
line, but this debug information can be understood only by the author). If a comment line starts
with '*\textcolor{green}{!',\,it\,will\,be\,printed.}
\subsection{Input Paragraph}
\label{inpupara}
Many \textbf{J} functions interpreted and executed at the command level need or can use a group of
text lines as input. In these cases the additional input lines are immediately after the function.
This group of lines is called input paragraph. The input paragraph ends with '/', except the
input paragraph of text function ends with '//' as a text object can contain ordinary input
paragraphs. It may be default for the function that there is input paragraph following. When
it is not a default, then the existence of the input paragraph is indicated with option \textcolor{blue}{in->}
without any value. An input paragraph can contain input programming commands; the
resulting text lines are transmitted to the \textbf{J} function which interprets the input paragraph
\begin{example}[inpuparag]Example of inputparagraph\\
\label{inpuparag}
tr=\textcolor{VioletRed}{trans}()\\
a=\textcolor{VioletRed}{log}(b)\\
\textcolor{VioletRed}{write}(\$,'($\sim$sinlog\,is=$\sim$,f4.0)',\textcolor{VioletRed}{sin}(a))\\
/\\
b=\textcolor{VioletRed}{matrix}(2,3,\textcolor{blue}{in->})\\
1,2,3\\
5,6,7\\
/
\end{example}
\subsection{Command shortcuts}
\label{short}
Command shortcuts are defined by defining character variables. When entering the
name of a character variable at \textcolor{Red}{sit>} prompt or from an include file, \textbf{J} excutes the command.
The command can be either input programming command or ??? command. The file jexamples.inc
shows an useful way to organize shortcuts and include files.
\begin{example}[shortex]Example of using shortcuts and include files\\
\label{shortex}
short1='\textcolor{VioletRed}{sin}(Pi)+\textcolor{VioletRed}{cos}(Pi);'\\
short1\\
te=\textcolor{VioletRed}{text}()\\
this=\textcolor{VioletRed}{thisfile}()\\
ju1='\textcolor{Red}{;incl}(this,\textcolor{blue}{from->}a1)'\\
ju2='\textcolor{Red}{;incl}(this,\textcolor{blue}{from->}a2)'\\
\textcolor{Red}{;return}\\
;a1:\\
'greetings from a1'\\
\textcolor{Red}{;return}\\
;a2:\\
'here, jump to a1';\\
ju1\\
'back here, return to \textcolor{Red}{sit>}'\\
\textcolor{Red}{;return}\\
//\\
\textcolor{VioletRed}{write}('shortex.txt',\$,te)\\
\textcolor{Red}{;incl}(shortex.txt)\\
ju1\\
ju2\\
delete\_f('shortext.txt')\\
te=0 \textcolor{green}{!delete\,also\,text\,object\,te}
\end{example}
\subsection{Input programming}
\label{inpuprog}
The purpose of the input programming is to read or generate \textbf{J} commands or input lines
needed by \textbf{J} functions. The names of input programming commands start with semicolon ';'.
There can be no space between ';' and the following input programming function. The syntax
of input programming commands is the same as in \textbf{J} functions, but the input programming
functions cannot have an output. There are also controls structures in the input programming.
An input paragraph can contain input programming structures.
\subsubsection{Labels in input programming}
\label{inpuad}

The included text files can contain labels. Labels define possible starting points for the
inclusion or jump labels within an include file. A label starts with semicolon (;) and
ends with colon (:). There cannot be other text but not commands on the label line.

\\
;ad1:  At this point we are doing thit and that
\\


\begin{note}
The definition of a transformations object can also contain labels. These labels start
with a letter and end also with colon (:). When defining a transformation object with \textcolor{VioletRed}{trans}() function,
the input paragraph can contain input programming addresses and code labels. It is up to
input programming what code alabels become part of the transformation object.
\end{note}


\subsubsection{Changing “i” sequences}
\label{inpureplace}
If an original input line contains text within quotation marks, then the sequence will be replaced
as follows. If a character variable is enclosed, then the value of the character variable is
substituted: E.g.
directory='D:$\backslash$j$\backslash$'
name='area1'
extension='svs'
then
\textcolor{blue}{in->}'"directory""name"."extension"'
is equivalent to
\textcolor{blue}{in->}'D:$\backslash$j$\backslash$area1.svs'
If the "-expression is not a character variable then \textbf{J} interprets the sequence as an arithmetic
expression and computes its value. Then the value is converted to character string and
substituted into the place. E.g. if nper is variable having value 10, then lines
\color{Green}
\begin{verbatim}
x#"nper+1"#"nper" = 56
chv = 'code"nper"'
\end{verbatim}
\color{Black}
are translated into
\color{Green}
\begin{verbatim}
x#11#10 = 56
chv = 'code10'
\end{verbatim}
\color{Black}

With " " substitution one can define general macros which will get specific interpretation by
giving values for character and numeric parameters, and numeric parameters can be utilized in
variable names or other character strings. In transformation sets one can shorten computation
time by calculating values of expressions in the interpretation time instead of doing
computations repeatedly. E.g. if there is in a data set transformation
x3 = "\textcolor{VioletRed}{sin}(Pi/4)"*x5
Then evaluation of \textcolor{VioletRed}{sin}(Pi/4) is done immediately, and the value is transmitted to the
transformation set as a real constant.
If value of the expression within a “” sequence is an integer then the value is dropped in the
place without the decimal point and without any spaces, otherwise its value is presented in
form which is dependent on magnitude of the value. After J3.0 the format can be explicitly
specified within [] before the numeric value. Eg. text can be put into a figure as
fig =
\textcolor{VioletRed}{drawline}(5,5,\textcolor{blue}{mark->}’y=”[f5.2]coef(reg,x1)“*x1+”[f5.2]coef(reg,1)“‘)
See file jex.txt and Chapter 8 for an ex
\subsubsection{\textcolor{Red}{;incl}()}\index{;incl}
\label{incl}
Includes lines from a file or from a text object. Using the \textcolor{blue}{from->}
option the include file can contain sections which start with adresses like
;ad: \\
and end with \\
\textcolor{Red}{;return}

\vspace{0.3cm}
\hline
\vspace{0.3cm}
\noindent Args  \tabto{3cm}  0|1  \tabto{5cm}   Ch|Tx  \tabto{7cm}
\begin{changemargin}{3cm}{0cm}
\noindent   file name. Default: the same file is used as in the previous \textcolor{Red}{;incl}().he
\end{changemargin}
\vspace{0.3cm}
\hline
\vspace{0.3cm}
\noindent \textcolor{blue}{from}  \tabto{3cm}  N|1  \tabto{5cm}   Ch  \tabto{7cm}
\begin{changemargin}{3cm}{0cm}
\noindent gives the starting \textcolor{blue}{in->} label for the inclusion, label is given without starting ';'
and ending ':'.
\end{changemargin}
\vspace{0.3cm}
\hline
\vspace{0.3cm}
\noindent \textcolor{blue}{wait} \tabto{3cm}  N|0  \tabto{5cm}    \tabto{7cm}
\begin{changemargin}{3cm}{0cm}
\noindent  \textbf{J} waits until the include file can be opened. Useful in client server applications.
See chapter \textbf{J} as a server.
\end {changemargin}
\hline
\vspace{0.2cm}
\begin{note}
Include files can be nested up to 4 levels.evels
\end{note}
\begin{note}
See Chapter Defining a text object with text function and using it in \textcolor{Red}{;incl} how to include
commands from a text object.
\end{note}
\begin{note}
When editing the include file with Notepad ++, it is reasonable to set the language as Fortran (free form).
\end{note}
\begin{example}[inpuincl]Example of \textcolor{Red}{;incl}()\\
\label{inpuincl}
file=\textcolor{VioletRed}{text}()\\
i=1;\\
\textcolor{VioletRed}{goto}(ad1)\\
i=2;\\
ad1:i=66;\\
\textcolor{VioletRed}{goto}(ad2,ad3,2) \,\textcolor{green}{!select\,label\,from\,a\,label\,list}\\
ad2:\\
i=3;\\
ad3:i=4;\\
\textcolor{VioletRed}{goto}(5) \,\textcolor{green}{!select\,label\,from\,the\,list\,of\,all\,labels}\\
ad4:i=5;\\
ad5:i=6;\\
//\\
\textcolor{VioletRed}{write}('file.txt',file)\\
\textcolor{VioletRed}{close}('file.txt')\\
\textcolor{Red}{;incl}(file.txt)\\
\textcolor{Red}{;incl}(file.txt,\textcolor{blue}{from->}ad2)
\end{example}
\begin{note}
The adress line can contain comment starting with '!'.
\end{note}
\subsubsection{\textcolor{Red}{;goto}()}\index{;goto}
\label{inpugoto}
Go to different adress in \textcolor{Red}{;incl}() file.
\vspace{0.3cm}
\hline
\vspace{0.3cm}
\noindent Args \tabto{3cm} 1 \tabto{5cm}  CHAR \tabto{7cm}
\begin{changemargin}{3cm}{0cm}
\noindent  The label from which the reading continues. With \textcolor{Red}{;goto}('adr1')
the adress line starts ;adr1:
\end {changemargin}
\hline
\vspace{0.2cm}
\begin{example}[inpugotoex]Example of \textcolor{Red}{;goto}() and \textcolor{Red}{;incl}()\\
\label{inpugotoex}
gototxt=\textcolor{VioletRed}{text}()\\
'Start jumping';\\
\textcolor{Red}{;goto}(ad2)\\
;ad1:\\
'Greetings from ad1';\\
\textcolor{Red}{;return}\\
;ad2:\\
'Greetings from ad2';\\
\textcolor{Red}{;goto}(ad1)\\
//\\
\textcolor{VioletRed}{print}(gototxt)\\
\textcolor{VioletRed}{if}(exist\_f('goto.txt')delete\_f('goto.txt')\\
\textcolor{VioletRed}{write}('goto.txt',gototxt)\\
\textcolor{VioletRed}{close}('goto.txt')\\
\textcolor{VioletRed}{print}('goto.txt')\\
\textcolor{Red}{;incl}(goto.txt)\\
\textcolor{Red}{;incl}(goto.txt,\textcolor{blue}{from->}ad1)\\
delete\_f('goto.txt')
\end{example}
\subsubsection{\textcolor{Red}{;return} returns from include file}\index{;return}
\label{inpureturn}
\textcolor{Red}{;return} in an input file means that the control returns to the point where a
jumpt to an label was found. Two different cases need to be separated:
\begin{itemize}
\item[\textbf{J}] The conrol came to the starting address or to the beginning of the include file
from outside the current include file using a \textcolor{Red}{;incl} command. Then \textcolor{Red}{;return} returns the control to upper level include
file or to the \textcolor{Red}{sit>} prompt.
\item[\textbf{J}] The control came to the starting label from within the same include file using
either an explicit \textcolor{Red}{;incl} or \textcolor{Red}{;goto} command or generating these commands commands
with command shortcut.
\end{itemize}
\subsubsection{\textcolor{Red}{;do}()}\index{;do}
\label{inpudo}
Generates new input records and replaces text with other text
using " "  to generate numbers, @list to generate lists of object names
and @\textcolor{VioletRed}{list}(elem) to pick the names of the elements of a list, or
\textcolor{Red}{;sum}() to generate sums and \textcolor{Red}{;dif}() to generate differences.
\vspace{0.3cm}
\hline
\vspace{0.3cm}
\noindent Args \tabto{3cm} 3|4 \tabto{5cm}  Var,Num.. \tabto{7cm}
\begin{changemargin}{3cm}{0cm}
\noindent Arguments are: iteration index, starting limit,
final limit and step. First argument must be a variable name and others
can be REAL variables or numeric constants.
\end {changemargin}
\hline
\vspace{0.2cm}
\begin{example}[inpudoex]Examples of \textcolor{Red}{;do}()\\
\label{inpudoex}
\textcolor{Red}{;do}(i,1,2)\\
x"i"="i"*10\\
\textcolor{VioletRed}{print}('Greetings from iteration "i"')\\
\textcolor{Red}{;enddo}\\
\textcolor{VioletRed}{print}(x1,x2)
\color{Green}
\begin{verbatim}
!After dropping out extra text about the processing we get:
<print('Greetings from iteration 1')
'Greetings from iteration 1'
<print('Greetings from iteration 2')
'Greetings from iteration 2'
sit< print(x1,x2)
<print(x1,x2)
x1=   10.000000000000000
x2=   20.000000000000000
\end{verbatim}
\color{Black}
\end{example}
\subsubsection{\textcolor{Red}{;sum}()}\index{;sum}
\label{inpusum}
\textbf{J} can generate text of form part1+part2+...partn into input line using
input programming function ;isum(). The syntax of the function is as follows:\\
\textcolor{Red}{;sum}(i,low,up,step)(text)\\
or \\
\textcolor{Red}{;sum}(i,low,up)\\
Arguments low, up and step must be integers (actually from nonintger values, the
integer part is used) or REAL variables. Thus te valuse cannot be obtained
from arithmetic operations. Sum is useful at least in \textcolor{VioletRed}{problem}() function.
\begin{example}[inpusumex]Example of \textcolor{Red}{;sum}()\\
\label{inpusumex}
su='\textcolor{Red}{;sum}(i,1,5)(a"i"*x")'\\
\textcolor{VioletRed}{print}(su)
\color{Green}
\begin{verbatim}
<print(su)
'a1*x1+a2*x2+a3*x3+a4*x4'
\end{verbatim}
\color{Black}
prob=\textcolor{VioletRed}{problem}()\\
\textcolor{Red}{;sum}(i,1,5)(a"i"*x"i")==max
\color{Green}
\begin{verbatim}
<prob=problem()
prob< a1*x1+a2*x2+a3*x3+a4*x4+a5*x5==max
\end{verbatim}
\color{Black}
\end{example}
\begin{note}
\textcolor{Red}{;dif}()() works similarly for minus
\end{note}
\subsubsection{\textcolor{Red}{;dif}()}\index{;dif}
\label{inpudif}
\textbf{J} can generate text of form part1-part2-...partn into input line using
input programming function \textcolor{Red}{;dif}(). The syntax of the function is as follows:\\
\textcolor{Red}{;dif}(i,low,up,step)(text)\\
or \\
\textcolor{Red}{;dif}(i,low,up)\\
Arguments low, up and step must be integers (actually from nonintger values, the
integer part is used) or REAL variables. Thus te valuse cannot be obtained
from arithmetic operations. \textcolor{Red}{;dif}() is useful at least in \textcolor{VioletRed}{problem}() function.

\begin{note}
\textcolor{Red}{;sum}()() works similarly for plus. See \textcolor{Red}{;sum}() for examples.
\end{note}
\subsubsection{\textcolor{Red}{;pause}}\index{;pause}
\label{inpupause}
Including input from an include file can be interrupted using an input programming
command \textcolor{Red}{;pause} promt or the \textbf{J} function \textcolor{VioletRed}{pause}('<prompt>'). In both cases
the user can give \textbf{J} commands, e.g., print objects, change the value of Printdebug etc.
The difference is that  \textcolor{VioletRed}{pause}('<prompt>') goes first through the interpreted and the interptreted
code is transmitted to the \textbf{J} function driver. In the \textcolor{Red}{;pause}- pause it is possible to
use input programming commands while in \textcolor{VioletRed}{pause}()- pause it is not possible. In both cases, when
an error occurs, the control remains at the pause prompt. If the user is pressing
<return> \textbf{J} continues in the include file. If \textcolor{VioletRed}{pause}() is part of a transformation object,
pressing <return>, the function driver continues in the transformation object.
If the user gives command 'e' or 'end', then \textbf{J} procees similarly as if an error had occured,
i.e. print error messages and returns control to \textcolor{Red}{sit>} -promt.
\subsubsection{\textcolor{Red}{;return} returns from include file}\index{;return}
\label{inpureturn}
\textcolor{Red}{;return} in an input file means that the control returns to the point where a
jumpt to an label was found. Two different cases need to be separated:
\begin{itemize}
\item[\textbf{J}] The conrol came to the starting address or to the beginning of the include file
from outside the current include file using a \textcolor{Red}{;incl} command. Then \textcolor{Red}{;return} returns the control to upper level include
file or to the \textcolor{Red}{sit>} prompt.
\item[\textbf{J}] The control came to the starting label from within the same include file using
either an explicit \textcolor{Red}{;incl} or \textcolor{Red}{;goto} command or generating these commands commands
with command shortcut.
\end{itemize}
\subsection{Generating sequences with ... -construct}
\label{dots}
It is often natural to index object names, and often we need to refer object
names having consecutive index numbers or index letters. In \textbf{J} versions before version 3.0 it
was possible to generate object lists using ... -construct which replaced part of
the input line with the names of objects being between the the object
name before ... and after ... . Now the dots construct is no more done as
part of the input programming but in the interpret subroutine which interprets the
input line and generates the integer vector for function and argument indices.
But as dots work as if it would be part of the input programming, it is presented in this
section. Currently also sequences of integer constants can be generated with dots and
sequences can be from larger to smaller.
\begin{example}[dotsex]Example of dots construct\\
\label{dotsex}
dat=\textcolor{VioletRed}{data}(\textcolor{blue}{read->}(x4...x7),\textcolor{blue}{in->})\\
1,2,3,4\\
11,12,13,14\\
/\\
\textcolor{VioletRed}{stat}(\textcolor{blue}{min->},\textcolor{blue}{max->},\textcolor{blue}{mean->})\\
x3\%mean...x7\%mean;\\
A...D=4...1;\\
Continue=1 \,\textcolor{green}{!demo\,of\,error\,in\,\textcolor{VioletRed}{data}()}\\
dat=\textcolor{VioletRed}{data}(\textcolor{blue}{read->}(x3...x7),\textcolor{blue}{in->})\\
1,2,3,4\\
11,12,13,14\\
/\\
Continue=0
\end{example}
\section{\textbf{J} functions: an overview}
\label{jfuncs}
\textbf{J} functions are organized into 21 groups. Some of of the functions are such that the
interpreter converts the code given by the user into new function names. For instance
the code 1+2 is first changed into \textcolor{VioletRed}{PLUS}(1,2), and code a(3) is changed
into form \textcolor{VioletRed}{getelem}(a,3). The following list contains also such implicit
function names to support also those users who are interested to start looking at
the open source code.
The ordering of functions in this manual is the same as in the file jmodules.f90
in order to make it easier to start looking the source code, even if the order is
not the most logical from the application point of view. When adding new functions into
any group is easy when the new function is put to the end of group so new functions are not inserted
in the middle of group even if it would more logical.
!
\begin{itemize}
\item Special functions
\begin{itemize}
\item setoption
\item getelem
\item setelem
\item list2
\item o1\_funcs
\item o2\_funcs
\item o3\_funcs
\item setcodeopt
\end{itemize}
\item  Objects
\begin{itemize}

\item type
\item delete\_o
\item exist\_o
\item name
\end{itemize}
\item 	Transformations
\begin{itemize}
\item trans
\item call
\item pause
\item noptions
\item R
\end{itemize}

\item  Loops and controls structures
\begin{itemize}
\item do
\item if
\item ASSIGN
\item sit
\item which
\item errexit
\item goto
\item itrace
\item trace
\item tracenow
\item itraceoff
\item traceoff
\item tracetest
\item assone
\item enddo
\item assmany
\item goto2
\item goto3
\end{itemize}

\item Arithmetic and logical operations after converting to the polish notation
\begin{itemize}
\item HMULT
\item HDIV
\item IPOWER
\item MULT
\item DIV
\item PLUS
\item MINUS
\item EQ
\item NE
\item LE
\item LT
\item GE
\item GT
\item NOT
\item AND
\item OR
\item EQV
\item NEQV
\item POWER
\end{itemize}
\item Arithemetic functions which can operate on scalars or on matrices
\begin{itemize}
\item max
\item sign
\item mod
\item nint
\item int
\item ceiling
\item floor
\item sqrt
\item sqrt2
\item log10
\item exp
\item sin
\item sind argument is in degrees, also in the following
\item cos
\item cosd
\item tan
\item tand
\item cotan
\item cotand
\item asin
\item asind
\item acos
\item acosd
\item atan
\item atand
\item acotan
\item acotand
\item sinh
\item cosh
\item tanh
\item fraction
\item abs
\end{itemize}

\item  Special arithemetic functions
\begin{itemize}
\item der
\item gamma
\item loggamma
\item logistic
\item npv
\end{itemize}

\item  Probability distributions
\begin{itemize}
\item pdf
\item cdf
\item bin
\item negbin
\item density
\end{itemize}
\item Random numbers
\begin{itemize}
\item ran
\item rann
\item ranpoi
\item ranbin
\item rannegbin
\item select
\item random
\end{itemize}
\item Interpolation
\begin{itemize}
\item interpolate
\item plane
\item bilin
\end{itemize}

\item  List functions
\begin{itemize}
\item  list
\item merge
\item difference
\item index
\item index\_v
\item len
\item ilist
\item putlist
\end{itemize}
\item  Creating a text  object
\begin{itemize}
\item text
\item txt
\end{itemize}
\item File handling
\begin{itemize}
\item exist\_f
\item delete\_f
\item close
\item showdir
\item setdir
\item thisfile
\item filestat
\item read
\item write
\item print
\item ask
\item askc
\item printresult
\item printresult2
\end{itemize}

\item  Matrices
\begin{itemize}
\item matrix
\item nrows
\item ncols
\item t
\item inverse
\item solve
\item qr
\item eigen
\item sort
\item envelope
\item ind
\item mean
\item sum
\item var
\item sd
\item minloc
\item maxloc
\item cumsum
\item corrmatrix
\end{itemize}

\item  Data functions
\begin{itemize}
\item data
\item newdata
\item exceldata
\item linkdata
\item getobs
\item nobs
\item classvector
\item values
\item transdata
\end{itemize}

\item  Statistical functions
\begin{itemize}
\item stat
\item cov
\item corr
\item regr
\item mse
\item rmse
\item coef
\item r2
\item se
\item nonlin
\item varcomp
\item classify
\item class
\end{itemize}

\item  Linear programming
\begin{itemize}
\item problem
\item jlp
\item weights
\item unit
\item schedcum
\item schedw
\item weight
\item partweights
\item partunit
\item partschedcum
\item partschedw
\item partweight
\item priceunit
\item weightschedcum
\item priceschedcum
\item priceschedw
\item weightschedw
\item integerschedw
\item xkf
\end{itemize}

\item Plotting figures
\begin{itemize}
\item plotyx
\item draw
\item drawclass
\item drawline
\item show
\item plot3d
\end{itemize}
\item Splines, stem splines,  and volume functions
\begin{itemize}
\item tautspline
\item stemspline
\item stempolar
\item laasvol
\item laaspoly
\item integrate
\end{itemize}

\item Bit functions
\begin{itemize}
\item setbits
\item clearbits
\item getbit
\item getbitch
\item bitmatrix
\item setvalue
\item closures
\end{itemize}

\item  Misc. utility functions
\begin{itemize}
\item value
\item properties
\item cpu
\item seconds
\end{itemize}
\end{itemize}

\section{Special functions}
\label{special}
The special functions are such that the interpreter uses these functions for special operations.
Only \textcolor{VioletRed}{list2}() is function which also the user can use, but the interpreter is
is using it implicitly.
\subsection{\textcolor{VioletRed}{setoption}(): set option on}\index{setoption()}
\label{setoption}
When a function has an option then the interpreter generates first code
\textcolor{VioletRed}{setoption}(...) where the arguments of the option are interpred in the similar way as
arguments of all functions. Then the interpreter genarates the code for
\textcolor{VioletRed}{setoption}() function in a special way.
\subsection{\textcolor{VioletRed}{getelem}(): extracting information from an object}\index{getelem()}
\label{getelem0}
The origin of this function is the function which was used in previous versions
to take an matrix element, which explains the name. Now it is used to extract
also submatrices (e.g. a(1,-3,All)) , or to get value of an regression
function or to compute
a transforamtion and then take argument object as the result. E.g. if \textcolor{teal}{tr} is a transformation
then the result of  \textcolor{teal}{tr}(\textcolor{teal}{a}) is object \textcolor{teal}{a} after calling \textcolor{teal}{tr}.
\begin{note}
If someone starts to use the own function property of the open source J, she/he
probably would like to get the possiblity to extract information from her/his object types
also. To implement this property requires some co-operation from my side.
\end{note}
\subsubsection{Get or set a matrix element or submratrix}
\label{getelem}
Matrix elements or submatrices can be accessed using the same syntax as
accesing \textbf{J} functions.

One can get or set matrix elements and submatrices as follows. If the expression
is on the right side of '=' then \textbf{J} gets a REAL value or submatrix, if the expression
is on the left side of '=', the \textbf{J} sets new values for a matrix element or a submatrix.
In the following formulas \textcolor{teal}{C} is a column vector, \textcolor{teal}{R} a row vector, and \textcolor{teal}{M} is
a general matrix with m rows and n columns.
If \textcolor{teal}{C} is actually REAL it can
be used as if it would 1 x 1 MATRIX. This can be useful when working with
matrices whose dimensions can vary starting from 1 x 1. Symbol \textcolor{teal}{r} refers to
row index, \textcolor{teal}{r1} to first row in a row range, \textcolor{teal}{r2} to the last row. The rows and
columns can be specifiel using ILIST objects \textcolor{teal}{il1} and \textcolor{teal}{il2} to specify noncontiguous ranges.
It is currently not possible to mix ILIST range and contiguous range, so if ILIST
is needed for rows (columns), it must be used also for columns (rows). ILIST can be
sepcified using explicitly \textcolor{VioletRed}{ilist}() function or using {} construction.
Similarly columns are indicated  with \textcolor{teal}{c}. It is always legal to refer to
vectors by using the \textcolor{teal}{M} formulation and giving \textcolor{teal}{c} with value 1 for column vectors and
and \textcolor{teal}{r} with value 1 for row vectors.
\vspace{0.3cm}
\hline
\vspace{0.3cm}
\noindent Args \tabto{3cm} 0-4 \tabto{5cm}  REAL | ILIST \tabto{7cm}
\begin{changemargin}{3cm}{0cm}
\noindent row and column range as explained below.
\end{changemargin}
\vspace{0.3cm}
\hline
\vspace{0.3cm}
\noindent \textcolor{blue}{diag} \tabto{3cm} N| \tabto{5cm}    \tabto{7cm}
\begin{changemargin}{3cm}{0cm}
\noindent Get or set diagonal elements
\end{changemargin}
\vspace{0.3cm}
\hline
\vspace{0.3cm}
\noindent \textcolor{blue}{sum} \tabto{3cm} N|0|1 \tabto{5cm}    \tabto{7cm}
\begin{changemargin}{3cm}{0cm}
\noindent  When setting elements, the right side is added to the current elements. If the
\textcolor{blue}{sum->} option has argument, the right side is multiplied with the argument when adding to the curretn elements.
\end {changemargin}
\hline
\vspace{0.2cm}

Row and column ranges can be spcidie as follows.
\begin{itemize}
\item[\textbf{J}\.] \textcolor{teal}{M}(r,c) \tabto{5cm} Get or set single element.
\item[\textbf{J}\.] \textcolor{teal}{C}(r)  \tabto{5cm} Get or set single element in column vector.
\item[\textbf{J}\.] \textcolor{teal}{R}(c) \tabto{5cm} Get or set single element in column vector.
\item[\textbf{J}\.] \textcolor{teal}{M}(RANGES) \tabto{5cm} Get or set a submatrix, where RANGES
can be. For a column vector, the column range need not to specified.
\begin{itemize}
\item[\textbf{J}\.] r1,-r2,c1,-c2
\item[\textbf{J}\.] r,c1,-c2  \tabto{5cm} part of row r
\item[\textbf{J}\.] r1,-r2,c	\tabto{5cm} part of column c
\item[\textbf{J}\.] r1,-r2,All  \tabto{5cm} All columns of the row range
\item[\textbf{J}\.] All,c1,-c2  \tabto{5cm} All rows of the column range
\item[\textbf{J}\.] {r1,....rm},{c1,...,cn}  \tabto{5cm} Given rows and columns
\item[\textbf{J}\.] {r1,....rm}  \tabto{5cm} Given rows for column vector
\item[\textbf{J}\.] il1,il2   \tabto{5cm} for matrix with several columns
\item[\textbf{J}\.] il1   \tabto{5cm} for column vector
\end{itemize}
\item[\textbf{J}\.] When r2= m, then -r2 can be replaced with \textcolor{teal}{Tolast}.
\item[\textbf{J}\.] When c2= n, then -c2 can be replaced with \textcolor{teal}{Tolast}.
\end{itemize}

If option \textcolor{blue}{diag->} is present then
\begin{itemize}
\item[\textbf{J}\.] \textcolor{teal}{M}(\textcolor{blue}{diag->}) Get or set the diagonal. If \textcolor{teal}{M} is not square matrix, and error
occurs.
\item[\textbf{J}\.] \textcolor{teal}{M}(r1,-r2,\textcolor{blue}{diag->}) (Again -r2 can be \textcolor{teal}{Tolast}.
\end{itemize}
\begin{note}
Note
When setting values to a submatrix the the the values given in input matrix
are put into the outputmatrix in row order, and the shape of the input and output matrices need
not be the same. An error occurs only if input and output contain different number of
elements.
\end{note}
\begin{example}[getset]Get or set submatrices\\
\label{getset}
A=\textcolor{VioletRed}{matrix}(3,4,\textcolor{blue}{do->});\\
B=A(1,-2,3,-4);\\
A(1,-2,3,-4)=B+3;\\
A(1,-2,3,-4,\textcolor{blue}{sum->})=-5;\\
A(1,-2,3,-4,\textcolor{blue}{sum->}2)=A(2,-3,1,-2,\textcolor{blue}{sum->}2);\\
C=A({1,3},{4...2});\\
H=\textcolor{VioletRed}{matrix}(4,4,\textcolor{blue}{diag->},\textcolor{blue}{do->}3);\\
H(3,-4,\textcolor{blue}{diag->})=\textcolor{VioletRed}{matrix}(2,\textcolor{blue}{values->}(4,7));
\end{example}
\begin{note}
When giving range, the lower and upper limit can be equal.qual
\end{note}
\subsection{\textcolor{VioletRed}{setelem}(): Putting something  into an object.}\index{setelem()}
\label{setelem}
The origin of this function is the function which was used in previous versions
to set an matrix element, which explains the name. Now it is used to replace values
of submatrices when submatrix expression is on the
output side (e.g. a(1,-3,All)=..).
\begin{note}
If someone starts to use the own function property of the open source J, she/he
probably would like to get the possiblity to put information into from her/his object types
also. To implement this property requires some co-operation from my side.
\end{note}
\begin{note}
In effect the \textcolor{VioletRed}{getelem}() and \textcolor{VioletRed}{setelem}() functions are excuted in the same
\textcolor{VioletRed}{getelem}() subroutine, because bot functions can utilize the same code.
\end{note}
\subsection{\textcolor{VioletRed}{list2}()}\index{list2()}
\label{list20}
The interpreted utilizes this to separate when spearating output and input objects.
See Section \ref{list2} hw user can use this function
\subsection{setcode(): Initialization of a code option}
\label{setcode}
This function initializes an code option for a function which has the option.
\subsection{Own functions:  o1\_o1\_funcs(), o2\_funcs() and o3\_funcs()}
\label{own}
The users of open \textbf{J} can define their own functions using three
available own-function sets. In addition to own functions open \textbf{J} is ready to
recognize also own object types and options which are defined in the source files
controlled by the users. The main \textbf{J} does not know what to do with these own
object types and options, they are just transmitted to the own-functions.
In the main \textbf{J} the control is transmitted to the own functions using
\textbf{J} functions o1\_o1\_funcs(), o2\_funcs() and o3\_funcs().
\section{Functions for handling objects}
\label{objects}
The following functions can handle objects.
\subsection{Get type of an object: \textcolor{VioletRed}{type}()}\index{type()}
\label{type}
\textcolor{green}{!\,The\,index\,of\,any\,object\,can\,be\,access\,by\,\textcolor{VioletRed}{type}(\textcolor{teal}{object}).}
\textcolor{green}{!\,If\,the\,argument\,is\,a\,character\,variable\,or\,character\,constant\,referring\,to}
\textcolor{green}{!\,a\,character\,constant,\,and\,there\,is\,\textcolor{blue}{content->}\,option,\,and\,the\,character\,is\,the\,name\,of}
an object , the \textcolor{VioletRed}{type}() returns the type of the object having the name given in the argument.
If there is no object having taht name, then \textcolor{VioletRed}{type}() returns -1 and no error is generated.
\begin{example}[typeex]Example of type\\
\label{typeex}
ttt=8; \,\,\textcolor{green}{!REAL}\\
\textcolor{VioletRed}{type}(ttt);\\
\textcolor{VioletRed}{type}('ttt'); \textcolor{green}{!type\,is\,CHAR}\\
\textcolor{VioletRed}{type}('ttt',\textcolor{blue}{content->});\\
cttt='ttt'\\
\textcolor{VioletRed}{type}(cttt);\\
\textcolor{VioletRed}{type}(cttt,\textcolor{blue}{content->});
\end{example}
\subsection{Deleting objects: delete\_o()}
\label{delete}
When an object with a given name is created, the name cannot be removed. With
delete\_0() function one can free all memory allocated for data structures needed
by general objects:
delete\_o(obj1,…,objn)
After deleting an object, the name refers to a real variable (which is
initialized by the delete\_o() function into zero).
\begin{note}
Other objects except matrices can equivalently be deleted by giving
command
obj1,…,objn = 0
This is because the output objects of any functions are first deleted before
defining them anew. Usually an object is automatically deleted if the object
name is as an output object for other functions.
\end{note}
\begin{note}
One can see how much memory each object is using \textcolor{VioletRed}{print}(Names).mes
\end{note}
\begin{note}
Deleting a compound object deletes also such
subobjects which have no meaning when the main object is deleted. But e.g. if a
data object is deleted then the as-sociated transformation object is not
deleted as the transformation can be used independently.
\end{note}
\begin{note}
Files can be deleted with delete\_f(file). See IO-functions for
details.
\end{note}
\begin{note}
If the user has defined own new compound objects in the open source
\textbf{J} software she/he needs to define the associated delete function.
\end{note}
\subsection{_o exist\_o(): does an object exist}
\label{exist}
looks whether an object with the name given in
the character constant  argument exists.
\begin{note}
In the prvious versions of \textbf{J} same function was used for files and objects.
\end{note}
\subsection{\textcolor{VioletRed}{name}(): writes the name of an object}\index{name()}
\label{name}
The argument gives the index of the object. This function
may useful if \textbf{J} prints in problem cases the object indices.
\section{Transformation objects}
\label{Transformations}
The code lines generated by the input programming can be either
executed directly after interpretation, or the interpreted code lines
are packed into a transformation object, which can be excuted with \textcolor{VioletRed}{call}()
which is either in the code generated with the input programming
or inside the same or other transformation object. Recursive
calling a transformation is thus also possible. Different functions
related to transformation object are described in this section.
\subsection{Generating a transformation object \textcolor{VioletRed}{trans}()}\index{trans()}
\label{trans}
\textcolor{VioletRed}{trans}() function interprets lines from input paragraph following the \textcolor{VioletRed}{trans}() command and puts the
interpreted code into an integer vector, which can be excuted in several places.
If there are no arguments in the function, the all objected used within the
transforamations are global. This may cause conflicts if there are several recursive
functions operating at the same time with same objects. \textbf{J} checks some of
these conflict situations, but not all.  These conflicts can be avoided by giving
intended global arguments  in the list of arguments.
Then an object 'ob' created e.g. with transformation object \textcolor{teal}{tr} have prefix
]tr$\backslash$[ yelding ]tr$\backslash$ob[. Actually also these objects are global, but their prefix
protects them so that they do not intervene with objects having the same name in the
calling transformation objec.

Each line in the input paragraph is read and interpreted and packed into a transformation
object, and associated tr\%input and tr\%output lists are created for input and output
variables. Objects can be in both lists. Objects having names starting
with '\$' are not put into the input or output lists. The source code is saved in a text object
tr\%source. List tr\%arg contains all arguments.
!
If a semicolon ';'  is at the end of an input line, then
the output is printed if REAL variable Prindebug has value 1 or value>2 at
the execution time. If the double semicolon ';;' is at the end then the output is
printed if Printresult>1. If there is no output, but just list of objects, then these
objects will be printed with semicolns.

!
\vspace{0.3cm}
\hline
\vspace{0.3cm}
\noindent Output \tabto{3cm} 1 \tabto{5cm}  Data \tabto{7cm}
\begin{changemargin}{3cm}{0cm}
\noindent The TRANS object generated.
\end{changemargin}
\vspace{0.3cm}
\hline
\vspace{0.3cm}
\noindent Args \tabto{3cm} N|1- \tabto{5cm}    \tabto{7cm}
\begin{changemargin}{3cm}{0cm}
\noindent  Global objects.
\end {changemargin}
\hline
\vspace{0.2cm}
\begin{note}
Options input->, \textcolor{blue}{local->}, \textcolor{blue}{matrix->}, \textcolor{blue}{arg->}, result->, \textcolor{blue}{source->} of previous
versions are obsolte.
\end{note}
\begin{note}
The user can intervene the execution from console if the code calls \textcolor{VioletRed}{read}(\$,),
\textcolor{VioletRed}{ask}(), \textcolor{VioletRed}{askc}() or \textcolor{VioletRed}{pause}() functions. During the pause one can give any command excepts
such input programming command as \textcolor{Red}{;incl}.
\end{note}
\begin{note}
The value of Printresult can be changed in other parts of the transformation, or
in other transforamations called or during execution of \textcolor{VioletRed}{pause}().
\end{note}

\begin{note}
Output variables in \textcolor{blue}{maketrans->} transformations whose name start with \$ are not put into the new data object.
\end{note}
\begin{example}[transex]Demonstrates also error handling\\
\label{transex}
tr=\textcolor{VioletRed}{trans}()\\
\$x3=x1+3\\
x2=2/\$x3;\\
/\\
tr\%input,tr\%output,tr\%source;\\
x1=8\\
\textcolor{VioletRed}{call}(tr)\\
tr2=\textcolor{VioletRed}{trans}(x1,x2)\\
\$x3=x1+3\\
x2=2/\$x3;\\
x3=x1+x2+\$x3;\\
/\\
tr2\%input,tr2\%output,tr2\%source;\\
\textcolor{VioletRed}{call}(tr2)\\
tr2\x3; \,\,\textcolor{green}{!x3\,is\,now\,local}\\
tr3=\textcolor{VioletRed}{trans}()\\
x1=-3\\
\textcolor{VioletRed}{call}(tr) \textcolor{green}{!this\,is\,causing\,division\,by\,zero}\\
/\\
Continue=1 \,\,\textcolor{green}{!\,continue\,after\,error}\\
\textcolor{VioletRed}{call}(tr3)
\end{example}
\color{Green}
\begin{verbatim}
sit>transex
<;incl(exfile,from->transex)


<tr=trans()
<$x3=x1+3
<x2=2/$x3;
</
<tr%input,tr%output,tr%source;
tr%input is list with            2  elements:
x1 $x3
tr%output is list with            1  elements:
x2
tr%source is text object:
1 $x3=x1+3
2 x2=2/$x3;
3 /
///end of text object

<x1=8


<call(tr)
x2=0.18181818

<tr2=trans(x1,x2)
<$x3=x1+3
<x2=2/$x3;
<x3=x1+x2+$x3;
</
<tr2%input,tr2%output,tr2%source;
tr2%input is list with            1  elements:
x1
tr2%output is list with            1  elements:
x2
tr2%source is text object:
1 $x3=x1+3
2 x2=2/$x3;
3 x3=x1+x2+$x3;
4 /
///end of text object

<call(tr2)
x2=0.18181818

tr2\x3=19.1818181

<tr2\x3;
tr2\x3=19.1818181

<tr3=trans()
<x1=-3
<call(tr)
</
<Continue=1
<call(tr3)
*division by zero
*****error on row            2  in tr%source
x2=2/$x3;
recursion level set to    3.0000000000000000

*****error on row            2  in tr3%source
call(tr)
recursion level set to    2.0000000000000000

*err* transformation set=$Cursor$
recursion level set to    1.0000000000000000
****cleaned input
call(tr3)
*Continue even if error has occured
<;return
\end{verbatim}
\color{Black}
\subsection{Excuting transformation object explicitly \textcolor{VioletRed}{call}()}\index{call()}
\label{call}

Interpreted transformations in a transformation object can be automatically executed by other \textbf{J}
functions or they can be executed explicitly using \textcolor{VioletRed}{call}() function.

\vspace{0.3cm}
\hline
\vspace{0.3cm}
\noindent Arg \tabto{3cm} 1 \tabto{5cm}  TRANS \tabto{7cm}
\begin{changemargin}{3cm}{0cm}
\noindent  The transformation object executed.
\end {changemargin}
\hline
\vspace{0.2cm}

\begin{note}
A transformation objects can be used recursively, i.e. a transformation can be called from
itself. The depth of recursion is not controlled by J, so going too deep in recursion will
eventually lead to a system error.
\end{note}
\begin{example}[recursion]Recursion produces system crash.\\
\label{recursion}
tr=\textcolor{VioletRed}{trans}() \textcolor{green}{!level\,will\,be\,initialized\,as\,zero}\\
level;\\
level=level+1\\
\textcolor{VioletRed}{call}(tr)\\
/\\
Continue=1 \,\textcolor{green}{!error\,is\,produced}\\
\textcolor{VioletRed}{call}(tr)\\
Continue=0
\end{example}
\subsection{Pause: \textcolor{VioletRed}{pause}()}\index{pause()}
\label{pause}
Function \textcolor{VioletRed}{pause}() stops the execution of \textbf{J} commands which are either in an
include file or in TRANS object. The user can give any commands during the
\textcolor{VioletRed}{pause}() except input programming commands.
If the user presses <return> the excution continues. If the user gives 'e', then
an error condition is generated, and \textbf{J} comes to the \textcolor{Red}{sit>} promt, except
if \textcolor{teal}{Continue} has values 1, in which case the control returns one level above
the \textcolor{Red}{sit>} promt.
If \textcolor{VioletRed}{pause}()  has a CHAR argument, then that character constant is used as
the prompt.
\begin{note}
When reading commands from an include file, the a pause can be generated also
with \textcolor{Red}{;pause}, which works similarly as \textcolor{VioletRed}{pause}(), but during \textcolor{Red}{;pause} also input programming
commands can be given.
\end{note}
\subsection{Number of options in the current function: \textcolor{VioletRed}{noptions}()}\index{noptions()}
\label{noptions}
\textcolor{VioletRed}{noptions}() returns as a REAL value the number of options currently
active. This may be used for testing purposes.
\begin{note}
When \textbf{J} processes options in different funtions, it varies at which point
options are cleared and number of options decreases
\end{note}
\section{Co-operation between \textbf{J} and R}
\label{JR}
Is is possible to run R scripst from \textbf{J} and \textbf{J} scripts from R.
\subsection{\textcolor{VioletRed}{R}() executes an R-script}\index{R()}
\label{R}
An R script can be executed with \textcolor{VioletRed}{R}(script) where script is CHAR object
defining the script text file. The function is calling //
call execute\_command\_line('Rscript.lnk '//j\_filename(1:le), wait=.false.)//
Thus a shortcut for the Rscript program needs to be available.
\begin{example}[Rex]Example of Rscript\\
\label{Rex}
rscript=\textcolor{VioletRed}{text}()\\
\# A simple R-script that generates a small data to file mydat.txt\\
wd<-"C:/j3/"\\
x<-runif(10,0,10)\\
y<-cbind(1,x)\%*\%c(1,2)+rnorm(10)\\
mydat<-data.frame(y,x)\\
write.table(mydat,file=paste(wd,"/mydat.txt",sep=""))\\
//\\
\textcolor{VioletRed}{write}('miniscript'.r',rscript)\\
\textcolor{VioletRed}{close}('miniscript.r)\\
\textcolor{VioletRed}{R}('miniscript.r')\\
\textcolor{VioletRed}{print}('mydat.txt')
\end{example}
\subsection{Calling J-scripts from R}
\label{Rcalls}
File JR\_0.0.tar.gz in the folder J\_R contains R tarball for taking \textbf{J}  subroutines into R.
With R command JR(”testr.inc”) the example jp problem can be solved from R.
Later lauri Mehtat\"alo will develop this co-operation further so that R can directly access also
matrices in the \textbf{J} memory.
For further information contakt lauri.mehtatalo@luke.fi
\section{Loops and control strucures}
\label{loops}
This section describes nonstadard functions.
\subsection{\textcolor{VioletRed}{do}() loops}\index{do()}
\label{do}

The loop construction in \textbf{J} looks as follows:
\\
\textcolor{VioletRed}{do}(i,start,end[,step])
\\
enddo
\\
\begin{note}
cycle and exit are implemented in the current \textbf{J} version with \textcolor{VioletRed}{goto}()oto
Within a do–loop there can be cycleand exitdostatements
\end{note}
\begin{note}
There can be 8 nested loops. do-loop is not allowed at command level.evel
\end{note}
\begin{example}[doex]do-loop\\
\label{doex}
'begin';\\
tr=\textcolor{VioletRed}{trans}()\\
\textcolor{VioletRed}{do}(i,1,5)\\
i;\\
ad1: \textcolor{VioletRed}{if}(i.eq.3)\textcolor{VioletRed}{goto}(cycle)\\
i;\\
\textcolor{VioletRed}{if}(i.eq.4)\textcolor{VioletRed}{goto}(jump)\\
cycle:enddo\\
jump:i;\\
\textcolor{green}{!\textcolor{VioletRed}{goto}(ad1)\,\,!\,it\,is\,not\,allowed\,to\,jump\,into\,a\,loop}\\
/\\
\textcolor{VioletRed}{call}(tr)
\end{example}
\subsection{\textcolor{VioletRed}{if}()}\index{if()}
\label{if}

\textcolor{VioletRed}{if}()j\_statement… \newline
The one line if-statement.
\subsection{assignment: \textcolor{teal}{output}=\textcolor{teal}{input}}
\label{ASSIGN}
There are two assignment functions generated by '=', when the line is of
form \textcolor{teal}{output}=func[]input[), then the output is directly
put to the output position of the function
without explicitly generating assignment.
When the codeline is in form \textcolor{teal}{output}=\textcolor{teal}{input} then the following cases can occur

\begin{itemize}
\item[\textbf{J}]  \textcolor{teal}{output} is MATRIX
and \textcolor{teal}{input} is scalar, then each element of MATRIX is replaced with the \textcolor{teal}{input}
in \textcolor{VioletRed}{assone}() function.
\item[\textbf{J}]  \textcolor{teal}{output} is submatrix expression, then the elements of the
submatrix are asigned in \textcolor{VioletRed}{setelem}() funtion whether \textcolor{teal}{input} is MATRIX or submatrix
expression, scalar or LIST.

\item[\textbf{J}]  \textcolor{teal}{output} is MATRIX  and on input side is a random number
generation function, the random numbers are put to all elements of the matrix.

\item[\textbf{J}] If on output side are many object names, and input side is
one REAL value, this is put to all variables.

\item[\textbf{J}] If on output side are many object names, and input there are several
variables then both sides should have equal numbers of object names, then
then copies of the input objects are put into output objects.


\end{itemize}
\begin{example}[assignex]Examples of assignments\\
\label{assignex}
a=\textcolor{VioletRed}{matrix}(2,3);\\
a=4;\\
a=\textcolor{VioletRed}{rann}();\\
v1...5=2...6;\\
v1...5=77;\\
Continue=1 \,\textcolor{green}{!\,ERROR}\\
v1..3=1,5;\\
Continue=0
\end{example}
\subsection{Selecting a value based on conditions.}
\label{which}

Usage://

output=\textcolor{VioletRed}{which}(condition1,value1,...,conditionn,valuen) //
or//
output=\textcolor{VioletRed}{which}(condition1,value1,...,conditionn,valuen,valuedefault)
Where conditionx is a REAL value, nonzero	value indicating TRUE. Output will get first value for which
the condition is TRUE. When the number of arguments is not even, the the last value
is the default value.
\begin{example}[whichex]Example of \textcolor{VioletRed}{which}()\\
\label{whichex}
c=9\\
\textcolor{VioletRed}{which}(a.eg.3.or.c.gt.8,5,a.eq.7,55);\\
a=7\\
\textcolor{VioletRed}{which}(a.eg.3.or.c.gt.8,5,a.eq.7,55);\\
a=5\\
\textcolor{VioletRed}{which}(a.eg.3.or.c.gt.8,5,a.eq.7,55);\\
\textcolor{VioletRed}{which}(a.eg.3.or.c.gt.8,5,a.eq.7,55,108);
\end{example}
\subsection{\textcolor{VioletRed}{errexit}()}\index{errexit()}
\label{errexit}
Function \textcolor{VioletRed}{errexit}() returns the control to \textcolor{Red}{sit>} prompt with a message similarly
as when an error occurs.

\begin{example}[errexitex]itex\\
\label{errexitex}
tr=\textcolor{VioletRed}{trans}()\\
\textcolor{VioletRed}{if}(a.eq.0)\textcolor{VioletRed}{errexit}('illegal value ',a)\\
s=3/a; \textcolor{green}{!\,division\,with\,zero\,is\,teste\,automatically}\\
/\\
a=3.7\\
\textcolor{VioletRed}{call}(tr)\\
tr(s); \textcolor{green}{!tr\,can\,also\,be\,used\,as\,a\,function}\\
a=0\\
Continue=1 \,\textcolor{green}{!Do\,not\,stop\,in\,thsi\,seflmade\,error}\\
\textcolor{VioletRed}{call}(tr)\\
Continue=0
\end{example}
\subsection{\textcolor{VioletRed}{goto}()}\index{goto()}
\label{goto}
Control can be transfered to a line in a transformation set with \textcolor{VioletRed}{goto}().
\begin{example}[gotoex]\\
\label{gotoex}
tr=\textcolor{VioletRed}{trans}()\\
i=0\\
\textcolor{VioletRed}{if}(i.eq.0)\textcolor{VioletRed}{goto}(koe)\\
'here';\\
koe:ch='here2';\\
/\\
\textcolor{VioletRed}{call}(tr)\\
ch;

\end{example}

\begin{note}
It is not allowed to jump in to a loop or into if -then structure. This is
checked already in in the interpreter.
\end{note}
\begin{note}
It is not yet possible to continue within an include file using Continue=1.
\end{note}
\begin{note}
It is not recommended to use \textcolor{VioletRed}{goto}() according to modern computation practices.
However, it was easier to implement cycle and exitdo with \textcolor{VioletRed}{goto}(), especially if
cycle or exitdo does not apply hte innermost do-loop.
\end{note}
\section{Arithmetic and logical operations}
\label{arit}
\textcolor{green}{!The\,arithmetic\,and\,logical\,operations\,are\,first\,converted\,into\,the\,polish\,notation.}
The logical operations follow the same ruels as addition +. The following rules,
extending the standard matrix computaion rules apply. The same rules aplly if the
order of arguments is changed,
\begin{itemize}
\item[\textbf{J}] MATRIX + REAL : REAL is added to each element

\item[\textbf{J}] MATRIX1+MATRIX2 :: elementwise addition, if matrices hav comptaible dimensions
\item[\textbf{J}]  MATRIX+ column vector: column vector is added to each column of MATRIX
if the numbers of rows agrees.
\item[\textbf{J}]  MATRIX+ row vector: row vector is added to each
row of MATRIX
if the numbers of columns agree.

\end{itemize}
The same rules apply for the lelmentwise multiplication *. and elementwise
division /. as for addition +.
\subsection{Minimum and maximum: \textcolor{VioletRed}{min}() and \textcolor{VioletRed}{max}}\index{min()max()}
\label{minmax}
Functions \textcolor{VioletRed}{min}() and max ()
behave in a special way, \textcolor{VioletRed}{max}() behaves similarly as \textcolor{VioletRed}{min}() here:
\begin{itemize}
\item[\textbf{J}] \textcolor{VioletRed}{min}(x1,x2):: minimum of two REAL
\item[\textbf{J}] \textcolor{VioletRed}{min}(MATRIX,REAL):: each element is \textcolor{VioletRed}{min}(elem,REAL)
\item[\textbf{J}] \textcolor{VioletRed}{min}(MATRIX):: row vector having minimums of all columns
\item[\textbf{J}] \textcolor{VioletRed}{min}(MATRIX,\textcolor{blue}{any->}):: minimum over the whole amtrix
\end{itemize}
\section{Statistical functions for matrices}
\label{matrixstat}
Functions \textcolor{VioletRed}{mean}(), \textcolor{VioletRed}{sd}(), \textcolor{VioletRed}{var}(), \textcolor{VioletRed}{sum}(), \textcolor{VioletRed}{min}() and \textcolor{VioletRed}{max}()
can be used used to compute stastics from a matrix. Let \textcolor{VioletRed}{mean}() here present
any of thes functions. The following rules apply:
\begin{itemize}
[\textbf{J}\.] \textcolor{VioletRed}{mean}(VECTOR) computes the mean of the vector, output is REAL
[\textbf{J}\.] \textcolor{VioletRed}{mean}(MATRIX) computes the mean of the each column. Result is row vector.
[\textbf{J}\.] \textcolor{VioletRed}{mean}(VECTOR,\textcolor{blue}{weight->}wvector) computes the
weigted mean of the vector,weights being in vector wvector.
[\textbf{J}\.] \textcolor{VioletRed}{mean}(MATRIX,\textcolor{blue}{weight->}wvector) computes the
weighted mean of each column, weights being in vector wvector, result is row vector.

\end{itemize}
\subsection{\textcolor{VioletRed}{mean}(): means or weighted means of matrix columns}\index{mean()}
\label{mean}
See section matrixstat for details
\subsection{\textcolor{VioletRed}{sd}(): sd's or weighted sd's of matrix columns}\index{sd()}
\label{sd}
See section matrixstat for details
\subsection{\textcolor{VioletRed}{var}(): Sample variances or weighted variances of matrix c}\index{var()}
\label{var}
See section matrixstat for details
\subsection{\textcolor{VioletRed}{sum}(): sums or weighted sums of matrix columns}\index{sum()}
\label{sum}
See section matrixstat for details
\section{Special arithmetic functions}
\label{Special}
\textbf{J} has the following arithmetic functions producing REAL values.These functions cannot yet have
matrix arguments.
\subsection{\textcolor{VioletRed}{der}(): derivatives}\index{der()}
\label{der}
Derivates of a function with respect to any of its arguments can be
computed using the derivation rules by using \textcolor{VioletRed}{der}() function in the previous line. The funcion must be expressed with
one-line statement. The function can call other functions using the standard way
to obtain objects from transformations, but these functions cannot
contain variables for which derivatives are obtained.
Nonlinear regression needs the derivatives with respect to the parameters.
\vspace{0.3cm}
\hline
\vspace{0.3cm}
\noindent Output  \tabto{3cm}    \tabto{5cm}     \tabto{7cm}
\begin{changemargin}{3cm}{0cm}
\noindent  The \textcolor{VioletRed}{der}() function does not have an explicit output, but
\textcolor{VioletRed}{der}() accompanied with the function produces REAL ]d[[] variable for each of the
argument variables.
\end{changemargin}
\vspace{0.3cm}
\hline
\vspace{0.3cm}
\noindent Args  \tabto{3cm} 1-  \tabto{5cm}   REAL  \tabto{7cm}
\begin{changemargin}{3cm}{0cm}
\noindent   ]d[[Argi] variable will get the value of the derivative wiht
respect to the argument \textcolor{teal}{Argi}.
\end {changemargin}
\hline
\vspace{0.2cm}
\begin{example}[derex]Derivatives with \textcolor{VioletRed}{der}()\\
\label{derex}
tr=\textcolor{VioletRed}{trans}()\\
\textcolor{VioletRed}{der}(x)\\
f=(1+x)*cos(x)\\
/\\
fi=\textcolor{VioletRed}{draw}(\textcolor{blue}{func->}tr(d[x]),\textcolor{blue}{x->}x,\textcolor{blue}{xrange->}(0,10),\textcolor{blue}{color->}Blue,\textcolor{blue}{continue->})\\
fi=\textcolor{VioletRed}{draw}(\textcolor{blue}{func->}tr(f),\textcolor{blue}{x->}x,\textcolor{blue}{xrange->}(0,10),\textcolor{blue}{color->}Cyan,\textcolor{blue}{append->},\textcolor{blue}{continue->}fcont)
\end{example}
\begin{example}[derex2]2\\
\label{derex2}
X=\textcolor{VioletRed}{matrix}(\textcolor{blue}{do->}(0,1000,10))\\
e=\textcolor{VioletRed}{matrix}(\textcolor{VioletRed}{nrows}(X))\\
e=\textcolor{VioletRed}{rann}(0,2);\\
A,Pmax,R=0.1,20,2\\
A*Pmax*1000/(A*1000+Pmax);\\
Y=A*Pmax*X/.(A*X+Pmax)-R+e \,\textcolor{green}{!rectangular\,hyperbola\,used\,often\,for\,photosynthesis}
\\
rect=\textcolor{VioletRed}{trans}()\\
\textcolor{VioletRed}{der}(A,Pmax,R)\\
f=A*Pmax*I/(A*I+Pmax)-R\\
/
\\
fi=\textcolor{VioletRed}{draw}(\textcolor{blue}{func->}(rect(f)),\textcolor{blue}{x->}I,\textcolor{blue}{xrange->}(0,1000),\textcolor{blue}{color->}Orange,\textcolor{blue}{width->}2,\textcolor{blue}{continue->},\textcolor{blue}{show->}0)\\
da=\textcolor{VioletRed}{newdata}(X,Y,e,\textcolor{blue}{extra->}(Regf,Resid),\textcolor{blue}{read->}(I,P,er))\\
\textcolor{VioletRed}{stat}()\\
fi=\textcolor{VioletRed}{plotyx}(P,I,\textcolor{blue}{append->},\textcolor{blue}{show->}0,\textcolor{blue}{continue->}fcont)
\\
A,Pmax,R=0.07,17,3 \textcolor{green}{!initial\,values}
\\
fi=\textcolor{VioletRed}{draw}(\textcolor{blue}{func->}(rect(f)),\textcolor{blue}{x->}I,\textcolor{blue}{xrange->}(0,1000),\textcolor{blue}{color->}Green,\textcolor{blue}{width->}2,\textcolor{blue}{append->},\textcolor{blue}{show->}0,\textcolor{blue}{continue->})\\
reg=\textcolor{VioletRed}{nonlin}(P,f,\textcolor{blue}{par->}(A,Pmax,R),\textcolor{blue}{var->},\textcolor{blue}{corr->},\textcolor{blue}{data->}da,\textcolor{blue}{trans->}rect)\\
reg\%var;\\
reg\%corr;\\
\textcolor{VioletRed}{corrmatrix}(reg\%var);\\
fi=\textcolor{VioletRed}{draw}(\textcolor{blue}{func->}(rect(f)),\textcolor{blue}{x->}I,\textcolor{blue}{xrange->}(0,1000),\textcolor{blue}{color->}Violet,\textcolor{blue}{append->},\textcolor{blue}{continue->}fcont)
\end{example}
\subsection{gamma function: \textcolor{VioletRed}{gamma}()}\index{gamma()}
\label{gamma}
Function \textcolor{VioletRed}{gamma}() produses the value of gamma funtion for a positive argument.
The function utilises gamma subroutine from
library dcdflib in Netlib.
\subsection{\textcolor{VioletRed}{logistic}(): value of the logistic function}\index{logistic()}
\label{logistic}
Returns the value of the logistic function 1/(1+\textcolor{VioletRed}{exp}(-x)). This can in principle computed by the
transformation, but the transformation will produce an error condition when the argument -x
of the exp-function is large. Because the logistic function is symmetric, these cases are
computed as 1-1/(1+\textcolor{VioletRed}{exp}(x)). Because the logistic function can be needed in the nonlinear
regression, also the derivatives are implemented.  Note, to utilse derivatives
the function needs to be in a TRANS object.
Eg when f=\textcolor{VioletRed}{logistic}(a*(x-x0)), then
the derivatives can be obtained with respect to the parameters a and x0 by
\begin{example}[logisticex]Example of logistic function\\
\label{logisticex}
tr=\textcolor{VioletRed}{trans}()\\
\textcolor{VioletRed}{der}(a,x0)\\
f=\textcolor{VioletRed}{logistic}(a*(x-x0));\\
/\\
x,x0,a=10,5,0.1\\
d[a],d[x],tr(d[x0]);
\end{example}
\begin{note}
In the previous example tr(d[x0] ahs the effect that TRANS tr is first
called, which makes that also d[a] and d[x] have been computed. Remember that
the parse tree is computed from right to left.
\end{note}
\subsection{\textcolor{VioletRed}{npv}(): net present value}\index{npv()}
\label{npv}

\textcolor{VioletRed}{npv}(]interest,income1,…,incomen,time1,…,timen[)//
Returns net present value for income sequence income1,...,incomen, occurring at times
time1,…,timen when the interest percentage is \textcolor{teal}{interest}.
\section{Functions for probality distributions}
\label{dist}
There are currently the following functions relate to probality distributions.
\begin{note}
function \textcolor{VioletRed}{density}() can be used define density or probality
function for any continuous or discrete distribution which can then be used
to generate random numbers with \textcolor{VioletRed}{random}() function.
\end{note}
\subsection{Density for normal distribution: \textcolor{VioletRed}{pdf}()}\index{pdf()}
\label{pdf}
\vspace{0.3cm}
\hline
\vspace{0.3cm}
\noindent Output  \tabto{3cm}  1  \tabto{5cm}   REAL  \tabto{7cm}
\begin{changemargin}{3cm}{0cm}
\noindent  the value of the density.
\end{changemargin}
\vspace{0.3cm}
\hline
\vspace{0.3cm}
\noindent Args  \tabto{3cm} 0-2  \tabto{5cm}   REAL  \tabto{7cm}
\begin{changemargin}{3cm}{0cm}
\noindent  \textcolor{teal}{Arg1} is the mean (default 0), \textcolor{teal}{Arg2} is the standard deviation
(default 1). If sd is given, the mean must be given explicitly as teh first argument.
\end {changemargin}
\hline
\vspace{0.2cm}
\begin{note}
See example drawclassex for an utilization of \textcolor{VioletRed}{pdf}()
\end{note}
\subsection{Cumulative distribution function for normal and chi2: \textcolor{VioletRed}{cdf}}\index{cdf()}
\label{cdf}
\vspace{0.3cm}
\hline
\vspace{0.3cm}
\noindent Output \tabto{3cm}  1  \tabto{5cm}   REAL  \tabto{7cm}
\begin{changemargin}{3cm}{0cm}
\noindent  The value of the cdf.
\end{changemargin}
\vspace{0.3cm}
\hline
\vspace{0.3cm}
\noindent Args  \tabto{3cm}  1-3  \tabto{5cm}   REAL \tabto{7cm}
\begin{changemargin}{3cm}{0cm}
\noindent  \textcolor{teal}{Arg1} the upper limit of the integral. When \textcolor{blue}{chi2->} is not present, then
\textcolor{teal}{Arg2}, if present is the mean of the normal distribution (defaul 0), and \textcolor{teal}{Arg3}, if present,
is the sd of the ditribution. If \textcolor{blue}{chi2->} is present, then oblicatory \textcolor{teal}{Arg2} is
ifs the number of degrees of freedom for chi2-distribution.
\end{changemargin}
\vspace{0.3cm}
\hline
\vspace{0.3cm}
\noindent \textcolor{blue}{chi2}  \tabto{3cm}  N |0  \tabto{5cm}    \tabto{7cm}
\begin{changemargin}{3cm}{0cm}
\noindent \noindent chi2  \tabto{3cm}  N |0  \tabto{5cm}    \tabto{7cm}
\end {changemargin}
\hline
\vspace{0.2cm}
\subsection{\textcolor{VioletRed}{bin}(): binomial probability}\index{bin()}
\label{bin}
\textcolor{VioletRed}{bin}(\textcolor{teal}{k},\textcolor{teal}{n},\textcolor{teal}{p})//
Gives the binomial probability that there will be \textcolor{teal}{k} successes
in \textcolor{teal}{n} independent trials when in a
single trial the probability of success is \textcolor{teal}{p}.
\subsection{\textcolor{VioletRed}{negbin}():: negative binomila distribution}\index{negbin()}
\label{negbin}
\textcolor{VioletRed}{negbin}(\textcolor{teal}{k},\textcolor{teal}{myy},\textcolor{teal}{theta})//
Gives the probability that a negative binomial random variable
has value \textcolor{teal}{k} when the variable
has mean \textcolor{teal}{myy} and variance \textcolor{teal}{myy}+]theta[*]myy[**2.
\begin{note}
\textcolor{VioletRed}{negbin}(k,n*p,0)=
\textcolor{VioletRed}{bin}(k,n*p).
\end{note}
\begin{note}
Sorry for the parameter inconsistency with \textcolor{VioletRed}{rannegbin}().
\end{note}
\subsection{\textcolor{VioletRed}{density}() define}\index{density()}
\label{density}
any discrete or continues distribution for random numbers either with a function
or histogram generated with \textcolor{VioletRed}{classify}()
\vspace{0.3cm}
\hline
\vspace{0.3cm}
\noindent Args \tabto{3cm} 0-1 \tabto{5cm}  MATRIX  \tabto{7cm}
\begin{changemargin}{3cm}{0cm}
\noindent  MATRIX generated with \textcolor{VioletRed}{classify}()
\end{changemargin}
\vspace{0.3cm}
\hline
\vspace{0.3cm}
\noindent \textcolor{blue}{func} \tabto{3cm}  N|1 \tabto{5cm}    \tabto{7cm}
\begin{changemargin}{3cm}{0cm}
\noindent codeoption defining the density. The x-varaible is \$.
\end{changemargin}
\vspace{0.3cm}
\hline
\vspace{0.3cm}
\noindent \textcolor{blue}{xrange} \tabto{3cm} 0|2 \tabto{5cm}   REAL \tabto{7cm}
\begin{changemargin}{3cm}{0cm}
\noindent  Range of x-values
\end{changemargin}
\vspace{0.3cm}
\hline
\vspace{0.3cm}
\noindent discrete \tabto{3cm} -1|0 \tabto{5cm}    \tabto{7cm}
\begin{changemargin}{3cm}{0cm}
\noindent  Presence implies the the distribution is discrete
\end {changemargin}
\hline
\vspace{0.2cm}
\begin{note}
Actually the function generates a matrix having towo rows which
has values for the cumulative distribution function.
\end{note}
\begin{note}
When defining the density function, the user need not care about
the scaling constant which makes the integral to integrate up to 1.
\end{note}
\begin{example}[densityex]Example of distributions\\
\label{densityex}
ber=\textcolor{VioletRed}{density}(\textcolor{blue}{func->}(1-p+(2*p-1)*\$),\textcolor{blue}{xrange->}(0,1),discrete->); Bernoully\\
bim=\textcolor{VioletRed}{matrix}(100)\\
bim=\textcolor{VioletRed}{random}(ber)\\
\textcolor{VioletRed}{mean}(bim);\\
p*(1-p); \,\textcolor{green}{!theoretical\,variance}\\
\textcolor{VioletRed}{var}(bim);\\
pd=\textcolor{VioletRed}{density}(\textcolor{blue}{func->}\textcolor{VioletRed}{exp}(-0.5*\$*\$),\textcolor{blue}{xrange->}(-3,3)) \,\textcolor{green}{!Normal\,distribution}
\\
ra=\textcolor{VioletRed}{random}(pd);\\
f=\textcolor{VioletRed}{matrix}(1000)\\
f=\textcolor{VioletRed}{random}(pd)\\
da=\textcolor{VioletRed}{newdata}(f,\textcolor{blue}{read->}x)\\
\textcolor{VioletRed}{stat}(\textcolor{blue}{min->},\textcolor{blue}{max->})\\
cl=\textcolor{VioletRed}{classify}(\textcolor{blue}{x->}x,\textcolor{blue}{xrange->});\\
fi=\textcolor{VioletRed}{drawclass}(cl)\\
fi=\textcolor{VioletRed}{drawclass}(cl,\textcolor{blue}{area->})

\\
fi=\textcolor{VioletRed}{draw}(\textcolor{blue}{func->}\textcolor{VioletRed}{pdf}(x),\textcolor{blue}{x->}x,\textcolor{blue}{xrange->},\textcolor{blue}{append->})\\
f=\textcolor{VioletRed}{matrix}(1000)\\
f=\textcolor{VioletRed}{rann}()\\
da=\textcolor{VioletRed}{newdata}(f,\textcolor{blue}{read->}x)\\
\textcolor{VioletRed}{stat}(\textcolor{blue}{min->},\textcolor{blue}{max->})\\
cl=\textcolor{VioletRed}{classify}(\textcolor{blue}{x->}x,\textcolor{blue}{xrange->})\\
fi=\textcolor{VioletRed}{drawclass}(cl,\textcolor{blue}{histogram->},\textcolor{blue}{classes->}20)\\
den=\textcolor{VioletRed}{density}(cl);\\
fi=\textcolor{VioletRed}{drawline}(den)
\end{example}
\section{Random number generators}
\label{randomgen}
Random number generators are taken from Ranlib library of Netlib.
They can produce single REAL variables or random MATRIX objects.
Random matrices are produced by defining first a matrix with \textcolor{VioletRed}{matrix}()
funtion and putting that as the output.
\subsection{\textcolor{VioletRed}{ran}(): uniform random number}\index{ran()}
\label{ran}
Uniform random numbers between 0 and 1 are generating usig Netlib function ranf.
\vspace{0.3cm}
\hline
\vspace{0.3cm}
\noindent Output \tabto{3cm}  1 \tabto{5cm}   REAL |MATRIX   \tabto{7cm}
\begin{changemargin}{3cm}{0cm}
\noindent  The generated REAL value or MATRIX.
Random matrix cab generated by defining first the matrix with \textcolor{VioletRed}{matrix}().
\end {changemargin}
\hline
\vspace{0.2cm}
\begin{example}[ranex]\\
\label{ranex}
\textcolor{VioletRed}{ran}();\\
\textcolor{VioletRed}{ran}();\\
cpu0=cpu()\\
A=\textcolor{VioletRed}{matrix}(10000,5)\\
A=\textcolor{VioletRed}{ran}()\\
\textcolor{VioletRed}{mean}(A);\\
\textcolor{VioletRed}{mean}(A,\textcolor{blue}{any->}) \textcolor{green}{!mean\,over\,all\,elements}\\
\textcolor{VioletRed}{mean}(A(All,2));\\
\textcolor{VioletRed}{sd}(A);\\
\textcolor{VioletRed}{sd}(A,\textcolor{blue}{any->});\\
\textcolor{VioletRed}{min}(A);\\
\textcolor{VioletRed}{min}(A,\textcolor{blue}{any->});\\
\textcolor{VioletRed}{max}(A);\\
cpu()-cpu0;

\end{example}

\subsection{\textcolor{VioletRed}{rann}(): normal random variate}\index{rann()}
\label{rann}
Computes normally distributed pseudo random numbers into a REAL variable or
into MATRIX.
\vspace{0.3cm}
\hline
\vspace{0.3cm}
\noindent Output \tabto{3cm} 1 \tabto{5cm}  REAL|MATRIX \tabto{7cm}
\begin{changemargin}{3cm}{0cm}
\noindent The matrix to be generated bus be defined earlier with \textcolor{VioletRed}{matrix}().cm} er
\end{changemargin}
\vspace{0.3cm}
\hline
\vspace{0.3cm}
\noindent Args \tabto{3cm} 0-2 \tabto{5cm}  num \tabto{7cm}
\begin{changemargin}{3cm}{0cm}
\noindent  rannn() produces N(0,1) variables, \textcolor{VioletRed}{rann}(mean) will produce
N(mean,1) variables and \textcolor{VioletRed}{rann}(mean,sd) procuses N(mean,sd) variables.
\end {changemargin}
\hline
\vspace{0.2cm}
\begin{example}[rannex]Random normal variates, illustrating also find\\
\label{rannex}
rx=\textcolor{VioletRed}{rann}() \,\textcolor{green}{!Output\,is\,REAL}\\
rm=\textcolor{VioletRed}{matrix}(100)\\
\textcolor{VioletRed}{print}(\textcolor{VioletRed}{mean}(rm),\textcolor{VioletRed}{sd}(rm),\textcolor{VioletRed}{min}(rm),\textcolor{VioletRed}{max}(rm))\\
Continue=1 \textcolor{green}{!an\,error}\\
large=\textcolor{VioletRed}{find}(rm,\textcolor{blue}{filter->}(\$.ge.2),any)\\
Continue=0\\
large=\textcolor{VioletRed}{find}(rm,\textcolor{blue}{filter->}(\$.ge.2),\textcolor{blue}{any->})\\
\textcolor{VioletRed}{print}(100*nrows(large)/\textcolor{VioletRed}{nrows}(rm))\\
cpu0=cpu()\\
rm2=\textcolor{VioletRed}{matrix}(1000000)\\
rm2=\textcolor{VioletRed}{rann}(10,2) \,\textcolor{green}{!there\,cannot\,be\,arithmetix\,opreations\,in\,the\,right\,side}\\
cpu()-cpu0;\\
\textcolor{VioletRed}{mean}(rm2),\textcolor{VioletRed}{sd}(rm2),\textcolor{VioletRed}{min}(rm2),\textcolor{VioletRed}{max}(rm2);\\
large=\textcolor{VioletRed}{find}(rm,\textcolor{blue}{filter->}(\$.ge.14),\textcolor{blue}{any->})\\
\textcolor{VioletRed}{print}(100*nrows(large)/\textcolor{VioletRed}{nrows}(rm))\\
!
\end{example}
\begin{note}
When generating a matrix with random numbers, there cannot be
arithmetic operations on the right side.That means that code:\newline
rm=\textcolor{VioletRed}{matrix}(100)\newline
rm=2*rann()\newline
would produce a REAL value rm.
\end{note}

\subsection{\textcolor{VioletRed}{ranpoi}(): random Poisson variables}\index{ranpoi()}
\label{ranpoi}
\textcolor{VioletRed}{ranpoi}(\textcolor{teal}{myy}//
returns a random Poisson variable with expected value and variance \textcolor{teal}{myy}
\subsection{\textcolor{VioletRed}{ranbin}(): random binomial values}\index{ranbin()}
\label{ranbin}

Binomial random numbers between 0 and n are generating usig Netlib
ignbin(n,p).Random matrix can generated by defining first
the matrix with \textcolor{VioletRed}{matrix}().
\vspace{0.3cm}
\hline
\vspace{0.3cm}
\noindent Output \tabto{3cm}  1 \tabto{5cm}   REAL |MATRIX   \tabto{7cm}
\begin{changemargin}{3cm}{0cm}
\noindent  The generated REAL value or MATRIX with
number of successes. (J does not have explicit integer type object).
\end{changemargin}
\vspace{0.3cm}
\hline
\vspace{0.3cm}
\noindent Args  \tabto{3cm} 2 \tabto{5cm}  REAL \tabto{7cm}
\begin{changemargin}{3cm}{0cm}
\noindent  \textcolor{teal}{Arg1} is the number of trials (n) and \textcolor{teal}{Arg2} is the probability
of succes in one trial.
!
\end {changemargin}
\hline
\vspace{0.2cm}
\begin{example}[ranbinex]x\\
\label{ranbinex}
\textcolor{VioletRed}{ranbin}(10,0.1);
\\
\textcolor{VioletRed}{ranbin}(10,0.1);\\
!\\
A=\textcolor{VioletRed}{matrix}(1000,2)\\
A(All,1)=\textcolor{VioletRed}{ranbin}(20,0.2)\\
A(All,2)=\textcolor{VioletRed}{ranbin}(20,0.2)\\
da=\textcolor{VioletRed}{newdata}(A,\textcolor{blue}{read->}(s1,s2))\\
\textcolor{VioletRed}{stat}()\\
cl=\textcolor{VioletRed}{classify}(1,\textcolor{blue}{x->}s1)\\
fi=\textcolor{VioletRed}{drawclass}(cl,\textcolor{blue}{histogram->},\textcolor{blue}{color->}Blue,\textcolor{blue}{continue->}fcont)\\
cl=\textcolor{VioletRed}{classify}(1,\textcolor{blue}{x->}s2)\\
fi=\textcolor{VioletRed}{drawclass}(cl,\textcolor{blue}{histogram->},\textcolor{blue}{color->}Red,\textcolor{blue}{append->},\textcolor{blue}{continue->}fcont)
\end{example}

\subsection{\textcolor{VioletRed}{rannegbin}(): negative binomial variates}\index{rannegbin()}
\label{rannegbin}
The  function  returns  random  number  distributed  according to the
negative binomila distribution.
\vspace{0.3cm}
\hline
\vspace{0.3cm}
\noindent Output \tabto{3cm} 1 \tabto{5cm}  REAL | MATRIX  \tabto{7cm}
\begin{changemargin}{3cm}{0cm}
\noindent  the number of successes in
independent Bernoul trials before r’th failure when
p is the probability of success. \textcolor{VioletRed}{ranbin}(r,1)returns 1.7e37 and
\textcolor{VioletRed}{ranbin}(r,0)returns 0.
\end{changemargin}
\vspace{0.3cm}
\hline
\vspace{0.3cm}
\noindent \textcolor{green}{!Args\,\tabto{3cm}\,\,\tabto{5cm}\,\,REAL\,\tabto{7cm}\,}
\begin{changemargin}{3cm}{0cm}
\noindent  r=\textcolor{teal}{Arg1} and p=\textcolor{teal}{Arg1}
\end {changemargin}
\hline
\vspace{0.2cm}

\begin{note}
there are different ways to define the negative binomial distribution. In this definition
a Poisson random variable with mean \$\lamda\$ is obtained by letting r go
to infinity and defining p= \$\lamda\$/( \$\lamda\$+r)
The mean E(x) of this definition is p*r/(1-p) and the variance is V=p*r/(1-p)2. Thus given
E(x) and V, r and p can be obtained as follows: p=1- E(x) /V and r= E(x)**2/(V- E(x)) . This is useful when
simulating ‘overdispersed Poisson’ variables. Sorry for the (temporary) inconsistency of parameters with
function \textcolor{VioletRed}{negbin}().
\end{note}
\begin{note}
can also have a noninteger values. This is not in accordance with the above
interpretation of the distribution, but it is compatible with interpreting negative binomial
distribution as a compound gamma-Poisson distribution and it is useful when simulating
overdispersed Poisson distributions.
\end{note}
\subsection{\textcolor{VioletRed}{select}(): Random selection of elements}\index{select()}
\label{select}
\vspace{0.3cm}
\hline
\vspace{0.3cm}
\noindent Output \tabto{3cm} 1 \tabto{5cm}  MATRIX  \tabto{7cm}
\begin{changemargin}{3cm}{0cm}
\noindent column vector with n elements indicating random
selection of k
elements out of n elements. The selection is with without replacement,
thus elements of the output are 1 or 0..
\end{changemargin}
\vspace{0.3cm}
\hline
\vspace{0.3cm}
\noindent Args \tabto{3cm} 2 \tabto{5cm}  REAL \tabto{7cm}
\begin{changemargin}{3cm}{0cm}
\noindent  k=\textcolor{teal}{Arg1} and n=\textcolor{teal}{Arg2}.
\end {changemargin}
\hline
\vspace{0.2cm}
\begin{example}[selectex]Random selection\\
\label{selectex}
S=\textcolor{VioletRed}{select}(500,10000)\\
\textcolor{VioletRed}{mean}(S),\textcolor{VioletRed}{sum}(S),500/10000;
\end{example}
\subsection{\textcolor{VioletRed}{random}():  random variates from any distribution}\index{random()}
\label{random}
usage \textcolor{VioletRed}{random}(\textcolor{teal}{dist}) where \textcolor{teal}{dist} is the density defined in \textcolor{VioletRed}{density}().
See \textcolor{VioletRed}{density}() for examples.
\section{Functions for interpolation}
\label{inter}
The following functions can be used for interpolation
\subsection{\textcolor{VioletRed}{interpolate}(): linear interpolation}\index{interpolate()}
\label{interpolate}
Usage://
\textcolor{VioletRed}{interpolate}(x0,x1[,x2],y0,y1[,y2],x]//
If arguments x2 and y2 are given then computes the value of the quadratic function at value
x going through the three points, otherwise computes the value of the linear function at value
x going through the two points.
\begin{note}
The argument x need not be within the interval of given x values (thus the function also
extrapolates).
\end{note}
\subsection{\textcolor{VioletRed}{plane}() interpolation from a plane}\index{plane()}
\label{plane}
Usage://
\textcolor{VioletRed}{plane}(x1,x2,x3,y1,y2,y3,z1,z2,z3,x,y]//
The function computes the equation of plane going through the three points (x1,y1,z1), etc
and computes the value of the z-coordinate in point (x,y). The three points defining the plane
cannot be on single line.
\subsection{\textcolor{VioletRed}{bilin}(): bilinear interpolation}\index{bilin()}
\label{bilin}
Usage://
\textcolor{VioletRed}{bilin}(x1,x2,y1,y2,z1,z2,z3,z4,x,y]//
z1 is the value of function at point (x1,y1), z2 is the value at point (x1,y2), z3 is the value at
(x2,y1) and z4 is the value at (x2,y2): the function is using bilinear interpolation to compute
the value of the z-coordinate in point (x,y). The point (x,y) needs not be within the square
defined by the corner points, but it is good if it is. See Press et al. ? (or Google) for the principle
of bilinear interpolation
\section{List functions}
\label{lists}
The following list functions are available
\subsection{Object lists}
\label{listo}
An object list is a list of named \textbf{J} object. See Shortcuts for implicit object
lists and List functions for more details. Object lists can be used also as
pointers to objects, see e.g. the selector option of the simulate() function.
\subsection{\textcolor{VioletRed}{list}() generates a LIST object}\index{list()}
\label{list}
\vspace{0.3cm}
\hline
\vspace{0.3cm}
\noindent Output \tabto{3cm}  1 \tabto{5cm}   LIST  \tabto{7cm}
\begin{changemargin}{3cm}{0cm}
\noindent The generated LIST object.
\end{changemargin}
\vspace{0.3cm}
\hline
\vspace{0.3cm}
\noindent Args \tabto{3cm}  0-  \tabto{5cm}     \tabto{7cm}
\begin{changemargin}{3cm}{0cm}
\noindent  named objects. If an argument is LIST it is ex+panded
\end{changemargin}
\vspace{0.3cm}
\hline
\vspace{0.3cm}
\noindent \textcolor{blue}{mask} \tabto{3cm}  N|1-  \tabto{5cm}  REAl \tabto{7cm}
\begin{changemargin}{3cm}{0cm}
\noindent  Which object are picked from the list of arguments.\ta
value 0 indicates that
he object is dropped, positive value indicates how many variables are taken,
negative value how many objects are dropped (thus 0 is equivalent to -1). mask-
option is useful for creating sublists of long lists.
\end {changemargin}
\hline
\vspace{0.2cm}
\begin{note}
The same object may appear several times in the list. (see \textcolor{VioletRed}{merge}())ge
\end{note}
\begin{note}
There may be zero arguments, which result in an empty list
which can be updated later.
\end{note}
\begin{note}
The index of object in a LIST can be obtained using \textcolor{VioletRed}{index}().ex
\end{note}
\\
li=\textcolor{VioletRed}{list}(x1...x3);
\\
\textcolor{VioletRed}{index}(x2,li);
\\
Continue=1
\\
\textcolor{VioletRed}{index}(x4,li); \,\textcolor{green}{!\,error}
\\
Continue=0
\\


\begin{example}[list2ex]x\\
\label{list2ex}
all=\textcolor{VioletRed}{list}(); \,\textcolor{green}{!\,empty\,list}\\
sub=\textcolor{VioletRed}{list}();\\
nper=3\\
\textcolor{Red}{;do}(i,1,nper)\\
period\#"i"=\textcolor{VioletRed}{list}(ba\#"i",vol\#"i",age\#"i",harv\#"i")\\
sub\#"i"=\textcolor{VioletRed}{list}(@period\#"i",\textcolor{blue}{mask->}(-2,1,-1))\\
all=\textcolor{VioletRed}{list}(@all,@period\#"i") \textcolor{green}{!note\,that\,all\,is\,on\,both\,sides}\\
sub=\textcolor{VioletRed}{list}(@sub,@sub\#"i")\\
;end do
\end{example}
\subsection{\textcolor{VioletRed}{merge}()}\index{merge()}
\label{merge}
\textcolor{VioletRed}{merge}() will produce of list consisting of separate objects
in argument lists and argument objects.
\vspace{0.3cm}
\hline
\vspace{0.3cm}
\noindent Output \tabto{3cm} 1 \tabto{5cm}  LIST \tabto{7cm}
\begin{changemargin}{3cm}{0cm}
\noindent A list which is produced by first putting all elements of
argument lists and non-list arguments into a vector, and then duplicate objects are dropped.
\end{changemargin}
\vspace{0.3cm}
\hline
\vspace{0.3cm}
\noindent Args \tabto{3cm} 2- \tabto{5cm}  LIST|OBJ \tabto{7cm}
\begin{changemargin}{3cm}{0cm}
\noindent  LIST and separate non-list objects.
\end {changemargin}
\hline
\vspace{0.2cm}
\begin{example}[mergex]Merging list\\
\label{mergex}
x1...x3=1,2,3\\
mat=\textcolor{VioletRed}{matrix}(3,\textcolor{blue}{values->}(4,5,6))\\
lis0=\textcolor{VioletRed}{list}(x2,x1)\\
lis2=\textcolor{VioletRed}{merge}(x1,mat,lis0)\\
\textcolor{VioletRed}{print}(lis2)
\color{Green}
\begin{verbatim}
<print(lis2)
lis2 is list with            3  elements:
x1 mat x2
\end{verbatim}
\color{Black}
\end{example}
\subsection{Difference of LIST objects}
\label{difference}
\textcolor{VioletRed}{difference}() removes elements from a LIST
\vspace{0.3cm}
\hline
\vspace{0.3cm}
\noindent Output \tabto{3cm}  1 \tabto{5cm}  LIST  \tabto{7cm}
\begin{changemargin}{3cm}{0cm}
\noindent  the generated LIST.
\end{changemargin}
\vspace{0.3cm}
\hline
\vspace{0.3cm}
\noindent Args \tabto{3cm} 2 \tabto{5cm}  LIST|OBJ  \tabto{7cm}
\begin{changemargin}{3cm}{0cm}
\noindent  The first argument gives the LIST from which the elements of of the
are removed  If second argument is LIST then all of its eleemts are remove, other wise
it is assumed that the second argument is an object which is remode from the lisrt.
\end {changemargin}
\hline
\vspace{0.2cm}
\begin{example}[diffex]fex\\
\label{diffex}
lis=\textcolor{VioletRed}{list}(x1...x3,z3..z5);\\
lis2=\textcolor{VioletRed}{list}(x1,z5);\\
liso=\textcolor{VioletRed}{difference}(lis,lis2);\\
liso2=\textcolor{VioletRed}{difference}(liso,z3);\\
Continue=1\\
lisoer=\textcolor{VioletRed}{difference}(lis,z6); \textcolor{green}{!\,error\,occurs}\\
liser=\textcolor{VioletRed}{difference}(Lis,x3); \,\textcolor{green}{!error\,occurs}\\
Continue=0
\end{example}
\subsection{\textcolor{VioletRed}{index}(): index of a variable in a data object}\index{index()}
\label{index}
To be documented later
\subsection{\textcolor{VioletRed}{index}(): index of a variable in a data object}\index{index()}
\label{index}
To be documented later
\subsection{\textcolor{VioletRed}{len}() lengths of different lists or vectors}\index{len()}
\label{len}
\textcolor{VioletRed}{len}(\textcolor{teal}{arg}) gives the following lenghts for different argument types
\begin{itemize}
\item[\textbf{J}\.] \textcolor{teal}{arg} is MATRIX => len=the size of the matrix, i.e.
\textcolor{VioletRed}{nrows}(\textcolor{teal}{arg})*ncols(\textcolor{teal}{arg})

\item[\textbf{J}\.] \textcolor{teal}{arg} is TEXT => len=the number of chracter in TEXT object

\item[\textbf{J}\.] \textcolor{teal}{arg} is LIST => len=the number of elements in LIST
\item[\textbf{J}\.] \textcolor{teal}{arg} is ILIST => len=the number of elements in ILIST
\end{itemize}
If \textcolor{teal}{arg} does not have a legal type for \textcolor{VioletRed}{len}(), then \textcolor{VioletRed}{len}(\textcolor{teal}{arg})=-1 if \textcolor{VioletRed}{len}() has
option \textcolor{blue}{any->}, otherwise an error is produced.
\subsection{\textcolor{VioletRed}{ilist}(): list of integers}\index{ilist()}
\label{ilist}
Generates a list of integers which can be used as indexes. This function
is used implicitly with {}.
\vspace{0.3cm}
\hline
\vspace{0.3cm}
\noindent Output \tabto{3cm}  1 \tabto{5cm}   ILIST  \tabto{7cm}
\begin{changemargin}{3cm}{0cm}
\noindent The generated ILIST.
\end{changemargin}
\vspace{0.3cm}
\hline
\vspace{0.3cm}
\noindent Args \tabto{3cm}  0-  \tabto{5cm}  REAL  \tabto{7cm}
\begin{changemargin}{3cm}{0cm}
\noindent  Values to be put into ILIST, or the dimesion
of the ILIST when values are given in \textcolor{blue}{values->},  or variables whose indeces
in in the data are put into the ILIST.
\end{changemargin}
\vspace{0.3cm}
\hline
\vspace{0.3cm}
\noindent \textcolor{blue}{data} \tabto{3cm} N|1 \tabto{5cm}   DATA \tabto{7cm}
\begin{changemargin}{3cm}{0cm}
\noindent  The DATA object from whose variable indeces are obtained.
\end{changemargin}
\vspace{0.3cm}
\hline
\vspace{0.3cm}
\noindent \textcolor{blue}{extra}  \tabto{3cm}  1 \tabto{5cm}  REAL  \tabto{7cm}
\begin{changemargin}{3cm}{0cm}
\noindent  Extra space reserved for later updates of the ILIST.
\end{changemargin}
\vspace{0.3cm}
\hline
\vspace{0.3cm}
\noindent \textcolor{blue}{values}  \tabto{3cm} N|1- \tabto{5cm}   REAL \tabto{7cm}
\begin{changemargin}{3cm}{0cm}
\noindent  Values to be put into ILIST when dimesnion is determined as the
only argument
\end {changemargin}
\hline
\vspace{0.2cm}
\begin{note}
ILIST is a new object whose all utilization possiblities are not yet explored.
It will be used e.g. when developing factory optimization.
\end{note}
\begin{note}
eote
Using \textcolor{VioletRed}{ilist}() by giving the dimension as argument and values with \textcolor{blue}{values->} option
imitates the definition of a matrix (column vector). The structure of ILIST object
is the same as LIST object which can be used in matrix computations.
\end{note}
\begin{example}[ilistex]ILIST examples\\
\label{ilistex}
{1,4,5};\\
{4...1};\\
A=\textcolor{VioletRed}{matrix}(4,4)\\
A({1,5},{3}=
\end{example}
\subsection{\textcolor{VioletRed}{putlist}()}\index{putlist()}
\label{putlist}
Usage://
\textcolor{VioletRed}{putlist}(LIST,OBJ)//
put OBJ into LIST
\section{Creating two types of text objects}
\label{texts}
The are now two object types for text.
\subsection{\textcolor{VioletRed}{text}() creates the old TEXT object}\index{text()}
\label{text}
Text objects are created as a side product by many \textbf{J} functions. Text objects can be created
directly by the \textcolor{VioletRed}{text}() function which works in a nonstandard way. The syntax is:
output=\textcolor{VioletRed}{text}()//
…
//

The input paragraph ends exceptionally with '//' and not with '/'. The lines within the input
paragraph of text are put literally into the text object (i.e. even if there would be input
programming functions or structures included)
\subsection{\textcolor{VioletRed}{txt}() generates the new TXT object.}\index{txt()}
\label{txt}
Works as \textcolor{VioletRed}{text}(), to be documented later. The new TXT object is used
to implement \textcolor{Red}{;incl} and Gnuplot -figures.
\section{File handling}
\label{file}
The following function can handle files.
\subsection{_o exist\_o(): does an object exist}
\label{exist}
looks whether an object with the name given in
the character constant  argument exists.
\begin{note}
In the prvious versions of \textbf{J} same function was used for files and objects.
\end{note}
\subsection{Deleting objects: delete\_o()}
\label{delete}
When an object with a given name is created, the name cannot be removed. With
delete\_0() function one can free all memory allocated for data structures needed
by general objects:
delete\_o(obj1,…,objn)
After deleting an object, the name refers to a real variable (which is
initialized by the delete\_o() function into zero).
\begin{note}
Other objects except matrices can equivalently be deleted by giving
command
obj1,…,objn = 0
This is because the output objects of any functions are first deleted before
defining them anew. Usually an object is automatically deleted if the object
name is as an output object for other functions.
\end{note}
\begin{note}
One can see how much memory each object is using \textcolor{VioletRed}{print}(Names).mes
\end{note}
\begin{note}
Deleting a compound object deletes also such
subobjects which have no meaning when the main object is deleted. But e.g. if a
data object is deleted then the as-sociated transformation object is not
deleted as the transformation can be used independently.
\end{note}
\begin{note}
Files can be deleted with delete\_f(file). See IO-functions for
details.
\end{note}
\begin{note}
If the user has defined own new compound objects in the open source
\textbf{J} software she/he needs to define the associated delete function.
\end{note}
\subsection{\textcolor{VioletRed}{close}() closes a file}\index{close()}
\label{close}
\textcolor{VioletRed}{close}(file) closes an open file where file is either a character constant
or character variable associated with a file.
\begin{note}
No open(9 function is needed. An file is opened when it is first time in \textcolor{VioletRed}{write}().te
if the file exists, it is asked whether the old file is dleted.
\end{note}
\subsection{\textcolor{VioletRed}{showdir}() shows the current directory}\index{showdir()}
\label{showdir}
\begin{note}
showdir is defined in the system dependent file jsysdep\_gfortran.f90.
Using other compliers it may be neccessary to change the definition
\end{note}
\subsection{\textcolor{VioletRed}{setdir}() sets the current directory}\index{setdir()}
\label{setdir}
\begin{note}
setdir is defined in the system dependent file jsysdep\_gfortran.f90.
Using other complilers it may be neccessary to change the definition
\end{note}
\subsection{\textcolor{VioletRed}{thisfile}() returns the name of the current include file}\index{thisfile()}
\label{thisfile}
The name of the current include file is returned as a character variable by:
out=\textcolor{VioletRed}{thisfile}()
This is useful when defining shortcuts for commands that include sections from an include file.
Using this function the shortcuts work even if the name of the include file is changed. See file
jexamples.inc for an application
\subsection{\textcolor{VioletRed}{filestat}() gives information of a file}\index{filestat()}
\label{filestat}
Function \textcolor{VioletRed}{filestat}(filename) prints the size of the
file in bytes (if available) and the time the file was last accessed
\section{Io-functions}
\label{io}
Theree are following io functions
\subsection{\textcolor{VioletRed}{read}() read from a file}\index{read()}
\label{read}
\textcolor{VioletRed}{read}(file,format[,obj1,…,objn][,\textcolor{blue}{eof->}var] [,\textcolor{blue}{wait->}])//
Reads real variables or matrices from a file. If there are no objects
to be read, then a record is
by-passed.//
Arguments:
file the file name as a character variable or a character constant//
format//
b' unformatted (binary) data //
'bn' unformatted, but for each record there is integer for the size of the record. Does
not work when reading matrices.
'bis' binary data consisting of bytes, each value is converted to real value (the only
numeric data type in J). This works only when reading matrices.//
'(….)' a Fortran format. Does not work when reading matrices.
\$ the * format of Fortran//
obj1,…,objn
\textbf{J} objects//
Options://
eof Defines the variable which indicates the end of file condition of the file. If the end
of the file is not reached the variable gets the value 0, and when the end of file is
reached then the variable gets value 1 and the file is closed without extra notice.

When \textcolor{blue}{eof->} option is not present and the file ends then an error
condition occurs and the file is closed.//
wait \textbf{J} is waiting until the file can be opened. Useful in client-server applications. See
chapter \textbf{J} as a server.
\begin{note}
Use \textcolor{VioletRed}{ask}() or \textcolor{VioletRed}{askc}() to read values from the terminal when reading lines from an
include file.
\end{note}
\begin{note}
When reading matrices, their shapes need to
be defined earlier with \textcolor{VioletRed}{matrix}()
hfunction.
\end{note}
\subsection{\textcolor{VioletRed}{write}()}\index{write()}
\label{write}
\textcolor{VioletRed}{write}(file,format,val1,…,valn[,\textcolor{blue}{tab->}][,\textcolor{blue}{rows->}])\textcolor{green}{!\,case[1/6]}
Writes real values to a file or to the console. If val1 is a matrix then this matrix (or usually
vector) is written or at most as many values as given in the \textcolor{blue}{values->} option.
Arguments:
file variable \$ (indicating the console), or the name of the file as a character variable
or a character constant, or variable \$Buffer
format
\$ indicates the '*' format of Fortran, works only for numeric values.
A character expression, with the following possibilities:
A format starting with 'b' will indicate binary file. Now 'b' indicates ordinary
unformatted write, later there will be other binary formats
A Fortran format statement, e.g. ($\sim$the\,values\,were\,$\sim$,4f6.0), with this
format pure text can be written by having no object to write (e.g.
\textcolor{VioletRed}{write}('out.txt','($\sim$kukuu$\sim$)')).
For these formats, other arguments are supposed to be real variables or numeric
expressions or there is a matrix argument. If they are not, then just the real value
which is anyhow associated with each \textbf{J} object is printed (usually it will be zero).
If the val1 argument is a matrix, then all values are printed.
val1,…,valn
real values
Options:
tab if format is a Fortran format then, \textcolor{blue}{tab->} option indicates that sequences of
spaces are replaced by tab character so that written text can be easily converted
to Ms Word tables. If there are no decimals after the decimal point also the
decimal point is dropped.
rows If val1 is a matrix or a vector and \textcolor{blue}{rows->} has one argument then at most as
many values are written as given in this option, if there are two arguments then
the option gives the range of written rows in the form \textcolor{blue}{rows->}(row1,-row2). If the
upper limit is greater than the number of rows, no error is produced, all available
rows are just written.
\textcolor{VioletRed}{write}(file,'t',t1,val1,t2,val2,…,tn,valn[,\textcolor{blue}{tab->}]) \textcolor{green}{!\,case[2/6]}
Tabulation format. positive tab position values indicate that the value is written starting from
that position, negative tab positions indicate that the value is written up to that position. The
values can be either numeric expressions or character variables or character constants. Tab
positions can be in any order.
Arguments:
file variable \$ (indicating the console), or the name of the file as a character variable
or a character constant, or variable \$Buffer
't' tabulation format
t1,val1,t2,val2,…,tn,valn
KUVAUS
Options:
tab option indicates that sequences of spaces are replaced by tab character so that
writ-ten text can be easily converted to Ms Word tables.

\subsection{\textcolor{VioletRed}{print}() prints objects}\index{print()}
\label{print}
\textcolor{VioletRed}{print}(arg1,…,argn[,\textcolor{blue}{maxlines->}][,\textcolor{blue}{data->}][,\textcolor{blue}{row->}]
[,\textcolor{blue}{file->}][,\textcolor{blue}{func->}][,\textcolor{blue}{debug->}])//
Print values of variables or information about objects.//
Arguments:
arg1,…,argn
arguments can be any \textbf{J} objects or values of arithmetic or logical expressions
Options:
maxlines the maximum number of lines printed for matrices, default 100.
data data sets. If \textcolor{blue}{data->} option is given then arguments must be ordinary real
variables obtained from data.
row if a text object is printed, then the first value given in the \textcolor{blue}{row->} option gives the
first line to be printed. If a range of lines is printed, then the second argument
must be the negative of the last line to be printed (e.q. \textcolor{blue}{row->}(10,-15)).Note
that \textcolor{VioletRed}{nrows}() function can be used to get the number of rows.
file the file name as a character variable or a character constant. Redirects the output
of the \textcolor{VioletRed}{print}() function to given file. After printing to the file, the file remains
open and must be explicitly closed (\textcolor{VioletRed}{close}(‘file’)) if it should be opened in
a different application.
form when a matrix is printed, the format for a row can be given as a Fortran format,
e.g. form ‘(15f6.2)’ may be useful when printing a correlation matrix.
debug the associated real variable part is first printed, and thereafter the tow associated
two integer vectors, the real vector and the double precision vector
func all functions available are printed
\begin{note}
For simple objects, all the object content is printed, for complicates objects only
summary information is printed. \textcolor{VioletRed}{print}(Names) will list the names, types and sizes of all
named \textbf{J} objects. The printing format is dependent on the object type.
\end{note}
\begin{note}
\textcolor{VioletRed}{print}() function can be executed for the output of a \textbf{J} command
by writing ';' or ';;' at the end of the line. The excution of implied \textcolor{VioletRed}{print}()
is dependent on the value of \textcolor{teal}{Printoutput}. If \textcolor{teal}{printoutput} =0,
then the output is not printed, If \textcolor{teal}{printoutput} =1, then ';' is
causing printing, if \textcolor{teal}{Printoutput} =2 then only ';;'-outputs are
printed, and if \textcolor{teal}{Printoutput} =3, the bot ';' and ';;' outputs are printed.
\end{note}


\subsection{\textcolor{VioletRed}{ask}() asks a value for REAL}\index{ask()}
\label{ask}
\textcolor{VioletRed}{ask}([var][,\textcolor{blue}{default->}][,\textcolor{blue}{q->}][,\textcolor{blue}{exit->}])//
Ask values for a variable while reading commands from an include file.//
Argument://
var 0 or one real variable (need not exist before)
Options:
default default values for the asked variables
q text used in asking
exit if the value given in this option is read, then the control returns to command level
similarly as if an error would occur. If there is no value given in this option, then
the exit takes place if the text given as answer is not a number.
\begin{note}
If there are no arguments, then the value is asked for the output variable, otherwise for
the argument. The value is interpreted, so it can be defined using transformations.
Response with carriage return indicates that the variables get the default values. If there is no
\textcolor{blue}{default->} option, then the previous value of the variable is maintained (which is also printed
as the \textcolor{blue}{default->} value in asking)
\end{note}

\begin{example}[askex]Examples for \textcolor{VioletRed}{ask}()\\
\label{askex}
a=\textcolor{VioletRed}{ask}(\textcolor{blue}{default->}8)\\
\textcolor{VioletRed}{ask}(a,\textcolor{blue}{default->}8)\\
\textcolor{VioletRed}{print}(\textcolor{VioletRed}{ask}()+\textcolor{VioletRed}{ask}()) \textcolor{green}{!\,ask\,without\,argument\,is\,a\,numeric\,function}\\
\textcolor{VioletRed}{ask}(v,\textcolor{blue}{q->}'Give v>')
\end{example}
\subsection{\textcolor{VioletRed}{askc}() asks a value for a character variable}\index{askc()}
\label{askc}
Usage ://
\textcolor{VioletRed}{askc}(chvar1[,\textcolor{blue}{default->}][,\textcolor{blue}{q->}][,\textcolor{blue}{exit->}])
Asks values for character variables when reading commands from an include file.

\vspace{0.3cm}
\hline
\vspace{0.3cm}
\noindent Args \tabto{3cm} 0-4 \tabto{5cm}  REAL | ILIST \tabto{7cm}
\begin{changemargin}{3cm}{0cm}
\noindent row and column range as explained below.

\end{changemargin}
\vspace{0.3cm}
\hline
\vspace{0.3cm}
\noindent Args  \tabto{3cm}  0|1  \tabto{5cm}  CHAR  \tabto{7cm}
\begin{changemargin}{3cm}{0cm}
\noindent  character variable (need not exist before)

\end{changemargin}
\vspace{0.3cm}
\hline
\vspace{0.3cm}
\noindent \textcolor{blue}{default} \tabto{3cm} 0|1 \tabto{5cm}  CHAR  \tabto{7cm}
\begin{changemargin}{3cm}{0cm}
\noindent  default character stings
\end{changemargin}
\vspace{0.3cm}
\hline
\vspace{0.3cm}
\noindent \textcolor{blue}{q}  \tabto{3cm}  0|1  \tabto{5cm}  CHAR  \tabto{7cm}
\begin{changemargin}{3cm}{0cm}
\noindent text used in asking
\end{changemargin}
\vspace{0.3cm}
\hline
\vspace{0.3cm}
\noindent \textcolor{blue}{exit} \tabto{3cm}  -1|0 \tabto{5cm}    \tabto{7cm}
\begin{changemargin}{3cm}{0cm}
\noindent  if the character constant or variable given in this option is read, then the control
return to command level similarly as if an error would occur.
\end {changemargin}
\hline
\vspace{0.2cm}
\begin{note}
Note
Response with carriage return indicates that the variable gets the default value. If there is no
\textcolor{blue}{default->} option, then the variable will be unchanged (i.e. it may remain also as another
object type than character variable).
\end{note}
\begin{note}
If there are no arguments, then the value is asked for the output variable, otherwise for
the arguments.
\end{note}
\subsection{\textcolor{VioletRed}{printresult}() and \textcolor{VioletRed}{printresult2}() e}\index{printresult()printresult2()}
\label{printresult}
The \textbf{J} interpreter translates ';'  at the end of the
line to a call to \textcolor{VioletRed}{printresult}() function and ';;' to a call to \textcolor{VioletRed}{printresult2}()
The output of a function is printed by writing ';' or ';;' at the end of the line. The excution of implied \textcolor{VioletRed}{print}()
is dependent on the value of \textcolor{teal}{Printoutput}. If \textcolor{teal}{printoutput} =0,
then the output is not printed, If \textcolor{teal}{printoutput} =1, then ';' is
causing printing, if \textcolor{teal}{Printoutput} =2 then only ';;'-outputs are
printed, and if \textcolor{teal}{Printoutput} =3, the bot ';' and ';;' outputs are printed.
\begin{note}
\textcolor{VioletRed}{printresult}() and \textcolor{VioletRed}{printresult2}() are  simple functions
which just test the value of
\textcolor{teal}{Printoutput} and then call the  printing subroutine, if needed.
\end{note}
\section{Matrix functions}
\label{matrixs}
\textbf{J} contains now the following matrix functions.
\subsection{Matrices and vectors}
\label{matrixo}
Matrices and vectors are generated with the \textcolor{VioletRed}{matrix}() function or they are
produced by matrix operations, matrix functions or by other \textbf{J} functions. E.g.
the \textcolor{VioletRed}{data}() function is producing a data matrix as a part of the compound data
object. Matrix elements can be used in arithmetic operations as input or output
in similar way as real variables.
See Matrix computations.
\subsection{\textcolor{VioletRed}{matrix}(): create a matrix:}\index{matrix()}
\label{matrix}
Function \textcolor{VioletRed}{matrix}() creates a matrix and puts REAL values to the elements. Element values
can be read from the input paragraph, file, or the values can be generated
using \textcolor{blue}{values->} option, or sequential values can be generated
using \textcolor{blue}{do->} option. Function \textcolor{VioletRed}{matrix}() can generate a diagonal and block diagonal matrix.
A matrix can be generated from submatrices by using matrices as arguments
of the  \textcolor{blue}{values->} option. It should be noted that matrices are stored in row order.
\vspace{0.3cm}
\hline
\vspace{0.3cm}
\noindent Output \tabto{3cm}  1 \tabto{5cm}   MATRIX | REAL \tabto{7cm}
\begin{changemargin}{3cm}{0cm}
\noindent  If a 1x1 matrix is defined, the output will be REAL.
The output can be a temporary matrix without name, if \textcolor{VioletRed}{matrix}() is an argument
of an arithmetic function  or matrix function. If no element values are
given in \textcolor{blue}{values->} or obtained from \textcolor{blue}{in->} input, all elemets get value zero.
\end{changemargin}
\vspace{0.3cm}
\hline
\vspace{0.3cm}
\noindent Args \tabto{3cm} 0-2 \tabto{5cm}  REAL \tabto{7cm}
\begin{changemargin}{3cm}{0cm}
\noindent  The dimension of the matrix. The first argument gives the number of rows,
the second argument, if present, the number of columns.  If the matrix is generated from submatrices given in \textcolor{blue}{values->}, then the dimensions
refer to the submatrix rows and submatrix columns. If there are no arguments, then the
it should be possible to infer the dimensions from \textcolor{blue}{values->} option. If the
first argument is \textcolor{teal}{Inf}, the the number of rows is determined by the number
number of lines in source determined by \textcolor{blue}{in->}.
\end{changemargin}
\vspace{0.3cm}
\hline
\vspace{0.3cm}
\noindent \textcolor{blue}{in} \tabto{3cm} N|0|1 \tabto{5cm}  CHAR \tabto{7cm}
\begin{changemargin}{3cm}{0cm}
\noindent  The input for values. \textcolor{blue}{in->} means that values are read in from
the following input paragraps, \textcolor{blue}{in->}\textcolor{teal}{file} means that the values are read from file.
in both cases a record must contain one row for the matrix.
If there is reading error and values are read from the terminal, \textbf{J} gives
possibility to continue with better luck, otherwise an error occurs.
\end{changemargin}
\vspace{0.3cm}
\hline
\vspace{0.3cm}
\noindent \textcolor{blue}{values} \tabto{3cm} N|1- \tabto{5cm}   REAL  \tabto{7cm}
\begin{changemargin}{3cm}{0cm}
\noindent values or MATRIX objects put to the matrix. The argumenst of
\textcolor{blue}{values->} option go in the regular way
through the interpreter, so the values can be obtained by computations. If only one REAL value is
given then all diagonal elements are put equal to the value (ohers will be zero),
if \textcolor{blue}{diag->} option is present, otherwise all elements are put equal to this value. If matrix dimensions
are given, and there are fewer values than is the size the matrix, matrix is
filled row by row using all values given in
\textcolor{blue}{values->}. If there are more values as is the size, an error occurs unless there is
\textcolor{blue}{any->} option present.
Thus \textcolor{VioletRed}{matrix}(N,N,\textcolor{blue}{values->}1) generates the identity matrix.
If value-> refers to one MATRIX,and \textcolor{blue}{diag->} is present then a block diagonal
matrix is generated. Without \textcolor{blue}{diag->}, a partitioned matrix is generated having all
submatrices equal
\end{changemargin}
\vspace{0.3cm}
\hline
\vspace{0.3cm}
\noindent \textcolor{blue}{do}  \tabto{3cm} N|0-3 \tabto{5cm}  REAL   \tabto{7cm}
\begin{changemargin}{3cm}{0cm}
\noindent  A matrix of number sequences is generated, as followsws: \newline
\textcolor{blue}{do->} Values 1,2,...,\textcolor{teal}{arg1} x \textcolor{teal}{arg2} are put into the matrix in the row order. \newline
\textcolor{blue}{do->}5 Values 5,6,...,\textcolor{teal}{arg1} x \textcolor{teal}{arg2}+4 are put into the matrix \newline
\textcolor{blue}{do->}
\end {changemargin}
\hline
\vspace{0.2cm}
\begin{example}[matrixex]Example of generating matrices\\
\label{matrixex}
A=\textcolor{VioletRed}{matrix}(3,

\end{example}

\subsection{\textcolor{VioletRed}{nrows}(): number of rows in MATRIX, TEXT or BITMATRIX}\index{nrows()}
\label{nrows}
can be used as:
\begin{itemize}
\item[\textbf{J}\.] \textcolor{VioletRed}{nrows}(MATRIX)
\item[\textbf{J}\.] \textcolor{VioletRed}{nrows}(TEXT)
\item[\textbf{J}\.] \textcolor{VioletRed}{nrows}(BITMATRIX)
\end{itemize}
\begin{note}
If the argument has another object type, and error occurs
\end{note}
\subsection{\textcolor{VioletRed}{ncols}(): number of columns in MATRIX or BITMATRIX}\index{ncols()}
\label{ncols}
can be used as:
\begin{itemize}
\item[\textbf{J}\.] \textcolor{VioletRed}{nrows}(MATRIX)
\item[\textbf{J}\.] \textcolor{VioletRed}{nrows}(BITMATRIX)
\end{itemize}
\begin{note}
If the argument has another object type, and error occurs
\end{note}
\subsection{\textcolor{VioletRed}{t}() gives transpose of a MATRIX or a LIST}\index{t()}
\label{t}
\textcolor{VioletRed}{t}(MATRIX) is the transpose of a MATRIX. As LIST objects can now
be used in matrix computations, \textcolor{VioletRed}{t}(LIST) is also available.
\begin{note}
Multiplying a matrix by the transpose of a matrix can be made by
making new operation '*.
\end{note}
\begin{note}
The argument matrix can also be a submatrix expression.
\end{note}
\subsection{\textcolor{VioletRed}{inverse}()}\index{inverse()}
\label{inverse}
\textcolor{VioletRed}{inverse}(A) computes the inverse of a square MATRIX A. The function utilized dgesv funtion
of netlib. If the argument has type REAL, then the reciprocal is computed,
and the output will also have type REAL. An error occurs, if A is not
a square matrix or REAL, or A is singular according to dgesv.
\begin{note}
instead of writing c=\textcolor{VioletRed}{inverse}(a)*b, it is faster and more accurate to
write c=\textcolor{VioletRed}{solve}(a,b)
\end{note}
\subsection{\textcolor{VioletRed}{solve}() solves a linear equation A*x=b}\index{solve()}
\label{solve}
A linear matrix eqaution A*x=b can be solved for x with code//
x=\textcolor{VioletRed}{solve}(A,b)
\begin{note}
x=\textcolor{VioletRed}{solve}(A,b) is faster and more accurate than x=\textcolor{VioletRed}{inverse}(A)*b
\end{note}
\begin{note}
solve works also if A and b are scalars. This is useful when
working with linear systems which start to grow from scalars.
\end{note}
\subsection{\textcolor{VioletRed}{qr}()}\index{qr()}
\label{qr}
Makes QR decomposition of a MATRIX This can be used to study if columns of a are linearly
dependent. \textbf{J} prints a matrix which indicates the structure of the upper diagonal matrix R in the
qr decomposition. If column k is linearly dependent on previous columns the k’th diagonal
element is zero. If output is given, then it will be the r matrix. Due to rounding errors diagonal
elements which are interpreted to be zero are not exactly zero. Explicit r matrix is useful if user
thinks that \textbf{J} has not properly interpreted which diagonal elements are zero.
In \textbf{J}  \textcolor{VioletRed}{qr}() may be useful when it is studied why a matrix which shoudl
be nonsingular turns out to be singular in \textcolor{VioletRed}{inverse}() or \textcolor{VioletRed}{solve}().
\textcolor{VioletRed}{qr}() is using  the subroutine dgeqrf from Netlib.
An error occurs if the argument is not MATRIX or if dgeqrf produces
error code, which is just printed.
Now the function just shows the linear dependencies, as sho in the examples.
\vspace{0.3cm}
\hline
\vspace{0.3cm}
\noindent Args \tabto{3cm} 1 \tabto{5cm}  MATRIX \tabto{7cm}
\begin{changemargin}{3cm}{0cm}
\noindent A m-by-n MATRIX.
\end {changemargin}
\hline
\vspace{0.2cm}
\subsection{\textcolor{VioletRed}{eigen}()}\index{eigen()}
\label{eigen}
Computes eigenvectors and eigenvalues of a square matrix. The eigenvectors are stored as
columns in matrix output\%matrix and the eigenvalues are stored as a row vector
output\%values. The eigenvalues and eigenvectors are sorted from smallest to larges eigenvalue.
Netlib subroutines DLASCL, DORGTR, DSCAL, DSTEQR, DSTERF, DSYTRD,
XERBLA, DLANSY and DLASCL are used.
\vspace{0.3cm}
\hline
\vspace{0.3cm}
\noindent Args \tabto{3cm} 1 \tabto{5cm}  MAT \tabto{7cm}
\begin{changemargin}{3cm}{0cm}
\noindent  A square MATRIX.
\end {changemargin}
\hline
\vspace{0.2cm}
\subsection{\textcolor{VioletRed}{sort}() sorts a matrix}\index{sort()}
\label{sort}
Usage://
\textcolor{VioletRed}{sort}(a,\textcolor{blue}{key->}(key1[,key2]))//
Makes a new matrix obtained by sorting all matrix columns of MATRIX a according to one or two columns.
Absolute value of key1 and the value of key2 must be legal column numbers.
If key1 is
positive then the columns are sorted in ascending order,
if key1 is negative then the columns
are sorted in descending order. If two keys are given, then first key dominates.
\begin{note}
It is currently
assumed that if there are two keys then the values in first key column have integer values.
\end{note}
\begin{note}
If key2 is not given and key1 is positive, then the syntax is: \textcolor{VioletRed}{sort}(a,\textcolor{blue}{key->}key1).
\end{note}
\begin{note}
If there is no output, then the argument matrix is sorted in place.lace
\end{note}
\begin{note}
The argument can be the data matrix of a data object. The data object will remain a
valid data object.
\end{note}
\subsection{\textcolor{VioletRed}{envelope}() computes the convex hull of point}\index{envelope()}
\label{envelope}
\vspace{0.3cm}
\hline
\vspace{0.3cm}
\noindent Output \tabto{3cm}  1 \tabto{5cm}  MATRIX  \tabto{7cm}
\begin{changemargin}{3cm}{0cm}
\noindent  (nvertex+1, 2) matrix of the coordinates of the convex hull, where nvertex is the number of
verteces. The last point is the same as the first point
\end{changemargin}
\vspace{0.3cm}
\hline
\vspace{0.3cm}
\noindent \textcolor{blue}{arg}  \tabto{3cm} 1 \tabto{5cm}  MATRIX  \tabto{7cm}
\begin{changemargin}{3cm}{0cm}
\noindent  (n,2) matrix of point coordinates
\end{changemargin}
\vspace{0.3cm}
\hline
\vspace{0.3cm}
\noindent \textcolor{blue}{nobs} \tabto{3cm} -1|1 \tabto{5cm}   REAL \tabto{7cm}
\begin{changemargin}{3cm}{0cm}
\noindent  gives the number of points if not all
points of the input matrix are used
\end {changemargin}
\hline
\vspace{0.2cm}
\begin{note}
The transpose of the output can be directly used in frawline() function
to draw the envelope
\end{note}
\begin{note}
The function is using a subroutine made by Alan Miller and found in Netlib
\end{note}
\subsection{\textcolor{VioletRed}{find}()}\index{find()}
\label{find}
Function \textcolor{VioletRed}{find}() can be used to find the first matrix element satisfying a given condition, or
all matrix elements satifying the conditon, and in that case the found
elements can be put to a vector containg element numbers or to a
vector which has equal size as the input matrix and where 1 indicates that
the element satifies the condition..
Remember that matrices are stored in row order. If a given column or row should be seaeched,
use submatrix() to extract that row or column.
\vspace{0.3cm}
\hline
\vspace{0.3cm}
\noindent Output \tabto{3cm}  1 \tabto{5cm}   REAL|MATRIX \tabto{7cm}
\begin{changemargin}{3cm}{0cm}
\noindent Without \textcolor{blue}{any->} or \textcolor{blue}{expand->} the first element found in row order.
With \textcolor{blue}{any->}, the vector of element numbers satisfying the conditon. If nothing found
the output will be REAL with value zero.
With \textcolor{blue}{expand->}, the matrix of the same dimensions as the input matrix where
hits are marked with 1.
\end{changemargin}
\vspace{0.3cm}
\hline
\vspace{0.3cm}
\noindent Args \tabto{3cm} 1 \tabto{5cm}  Matrix \tabto{7cm}
\begin{changemargin}{3cm}{0cm}
\noindent  The matrix searched.
\end{changemargin}
\vspace{0.3cm}
\hline
\vspace{0.3cm}
\noindent \textcolor{blue}{filter} \tabto{3cm} 1 \tabto{5cm}  Code \tabto{7cm}
\begin{changemargin}{3cm}{0cm}
\noindent Gives the condition which the matrix element should be satisfied. The
values of the matrix elements are put to the variable \$.
\end{changemargin}
\vspace{0.3cm}
\hline
\vspace{0.3cm}
\noindent \textcolor{blue}{any} \tabto{3cm} -1|0 \tabto{5cm}    \tabto{7cm}
\begin{changemargin}{3cm}{0cm}
\noindent  The filtered element numbers are put to the output vector.
\end{changemargin}
\vspace{0.3cm}
\hline
\vspace{0.3cm}
\noindent \textcolor{blue}{expand} \tabto{3cm} -1|0 \tabto{5cm}   \tabto{7cm}
\begin{changemargin}{3cm}{0cm}
\noindent  The filtered elements are put the output matrix
\end {changemargin}
\hline
\vspace{0.2cm}
\begin{example}[findex]Example of find, illustrating also \textcolor{VioletRed}{rann}()\\
\label{findex}
\textcolor{green}{!Repeating\,the\,example,\,different\,results\,will\,be\,obtained}\\
rm=\textcolor{VioletRed}{matrix}(100)\\
m,s=2,3\\
rm=\textcolor{VioletRed}{rann}(m,s)\\
\textcolor{VioletRed}{print}(\textcolor{VioletRed}{mean}(rm),\textcolor{VioletRed}{sd}(rm),\textcolor{VioletRed}{min}(rm),\textcolor{VioletRed}{max}(rm))
\color{Green}
\begin{verbatim}
=   2.4564691829681395
=   3.2549002852383477
=  -5.6481685638427734
=   10.714715003967285
\end{verbatim}
\color{Black}
first=\textcolor{VioletRed}{find}(rm,\textcolor{blue}{filter->}(\$.ge.m+1.96*s))\\
large=\textcolor{VioletRed}{find}(rm,\textcolor{blue}{filter->}(\$.ge.m+1.96*s),\textcolor{blue}{any->})\\
large2=\textcolor{VioletRed}{find}(rm,\textcolor{blue}{filter->}(\$.ge.m+1.96*s),\textcolor{blue}{expand->})\\
\textcolor{VioletRed}{print}(first,100*nrows(large)/\textcolor{VioletRed}{nrows}(rm),100*sum(large2)/\textcolor{VioletRed}{nrows}(rm))
\color{Green}
\begin{verbatim}
first=   12.000000000000000
=   4.0000000000000000
=   4.0000000000000000
\end{verbatim}
\color{Black}
\end{example}
\subsection{\textcolor{VioletRed}{mean}(): means or weighted means of matrix columns}\index{mean()}
\label{mean}
See section matrixstat for details
\subsection{\textcolor{VioletRed}{sum}(): sums or weighted sums of matrix columns}\index{sum()}
\label{sum}
See section matrixstat for details
\subsection{\textcolor{VioletRed}{var}(): Sample variances or weighted variances of matrix c}\index{var()}
\label{var}
See section matrixstat for details
\subsection{\textcolor{VioletRed}{sd}(): sd's or weighted sd's of matrix columns}\index{sd()}
\label{sd}
See section matrixstat for details
\subsection{\textcolor{VioletRed}{minloc}() : locations of the minimum values in columns}\index{minloc()}
\label{minloc}
\textcolor{VioletRed}{minloc}(MATRIX) generates a row vector containing the locations of the  minimum
values in each column. \textcolor{VioletRed}{minloc}(VECTOR) is the REAL scalar telling
the location of the minimum value. Thus the VECTOR can also be a row vector.
\subsection{\textcolor{VioletRed}{maxloc}() : locations of the minimum values in columns}\index{maxloc()}
\label{maxloc}
\textcolor{VioletRed}{maxloc}(MATRIX) generates a row vector containing the locations of the  minimum
values in each column. \textcolor{VioletRed}{maxloc}(VECTOR) is the REAL scalar telling
the location of the maxim value whether VECTOR is a row vector or column vector.
\subsection{\textcolor{VioletRed}{cumsum}(): cumulative sums of matrix columns}\index{cumsum()}
\label{cumsum}
\textcolor{VioletRed}{cumsum}(MATRIX) generates a MATRIX with the same dimesnions as the argument,
and puts the cumulative sums of the columsn into the output matrix.
\begin{note}
If the argument is vector, the cumsum makes a vector having the same
form as the argument.
\end{note}












\subsection{Making correaltion matrix from variance-covariance amtrix: \textcolor{VioletRed}}\index{corrmatrix()}
\label{corrmatrix}
This simple function is sometimes needed. The function does not test wether the input matrix is symmetric.
Negative diagonal eleemnt produces error, value zero correaltion 9,99.
\vspace{0.3cm}
\hline
\vspace{0.3cm}
\noindent Output \tabto{3cm}  1 \tabto{5cm}   MATRIX \tabto{7cm}
\begin{changemargin}{3cm}{0cm}
\noindent  matrix having nondiagonal values \textcolor{teal}{Out}(i,j)=\textcolor{teal}{arg}(i,j)=
\textcolor{teal}{arg}(i,j)/\textcolor{VioletRed}{sqrt}(\textcolor{teal}{arg}(i,i)*]arg[(j,j)).
\end{changemargin}
\vspace{0.3cm}
\hline
\vspace{0.3cm}
\noindent Args  \tabto{3cm} 1 \tabto{5cm}   MATRIX \tabto{7cm}
\begin{changemargin}{3cm}{0cm}
\noindent  symmetric matrix
\end{changemargin}
\vspace{0.3cm}
\hline
\vspace{0.3cm}
\noindent \textcolor{blue}{sd} \tabto{3cm} N|0  \tabto{5cm}    \tabto{7cm}
\begin{changemargin}{3cm}{0cm}
\noindent If \textcolor{blue}{sd->} is given, then diagonal elements will be equal to \textcolor{VioletRed}{sqrt}(\textcolor{teal}{arg}(i,i)
\end {changemargin}
\hline
\vspace{0.2cm}
\section{Working with DATA objects}
\label{datahead}
Data can be analyzed and processed either using matrix computations
or using DATA objects. A DATA object is compound object
linked to a data MATRIX and LIST object containing variable (column) names,
some other information.
When data ere used via DATA object in statistical or linear programming
functions, the data are processed observarion by observation.
It is possible to work using DATA object or using directly the data matrix, wharever is more
convenient. It is posible make new data objects or new matrices
by to extracting  columns of data matrix, computing matrices with matrix computations.
It is possible to use data in hierarcchila way, This property is inherited fro JLP.
There are two \textbf{J} functions which create DATA objects from files, \textcolor{VioletRed}{data}() and
exceldta(). \textcolor{VioletRed}{data}() can create hierarchical data objects. Function \textcolor{VioletRed}{newdata}() creates  DATA object from matrices, which temselfs can be
picked from data objects. Function \textcolor{VioletRed}{linkdata}() can link two data sets to make a hierarchical data.
\begin{note}
If a data file contains columns which are referred with variable names and some vectors,
the it is practical to read data first into a matrix using \textcolor{VioletRed}{matrix}() function and then
use matrix operations and \textcolor{VioletRed}{newdata}() to make DATA object with variable names and matrices.
See Simulator section for an example.
\end{note}
\begin{note}
\textcolor{VioletRed}{transdata}() function goes through DATA object similarly as statitical functions, but
does not serve a specific purpose, just transforamtions defined in the TRANS object refreed with
\textcolor{blue}{trans->} option are computed. See again the simulator section.
\end{note}
\begin{note}
In earlier versions it was possible to give several data sets as arguments for \textcolor{blue}{data->} option.
This festure is now deleted as it is possible to stack several data matrices and then use \textcolor{VioletRed}{newdata}() function to create a single data set.
\end{note}
\subsection{\textcolor{VioletRed}{data}()}\index{data()}
\label{data}
Data objects are created with the \textcolor{VioletRed}{data}() function. Two linked data objects can be created with the
same function call (using option \textcolor{blue}{subdata->} and options thereafter in the following
description). It is recommended that two linked data objects are created with one \textcolor{VioletRed}{data}()
function call only in case the data is read from a single file where subdata observations are
stored immediately after the upper data observation.  Data objects can be linked also afterwards with the \textcolor{VioletRed}{linkdata}() function.
A data object can created by a \textcolor{VioletRed}{data}() function when data are read from files
or data are created using transformation objects. New data objects can
be created with \textcolor{VioletRed}{newdata}() function from previous data objects and/or matrices.
If data objects can created using transformation objects either with \textcolor{VioletRed}{data}() function
or by creating first data matrix by transformation and then using \textcolor{VioletRed}{newdata}() to
create data object.
\vspace{0.3cm}
\hline
\vspace{0.3cm}
\noindent Output  \tabto{3cm} 0|1 \tabto{5cm}  Data \tabto{7cm}
\begin{changemargin}{3cm}{0cm}
\noindent \noindent Output  \tabto{3cm} 0|1 \tabto{5cm}  Data \tabto{7cm}
Data object to be created. If there is no output then the default is \$Data\$.
It is recommended that this default is used only when only one data object
is used in the analysis.
!
\end{changemargin}
\vspace{0.3cm}
\hline
\vspace{0.3cm}
\noindent \textcolor{blue}{read}  \tabto{3cm} 0|1- \tabto{5cm}  REAL|List \tabto{7cm}
\begin{changemargin}{3cm}{0cm}
\noindent  Variables read from the input files or the name of the list containing all variables
to be read in. If no arguments are given and there is no readfirst-> option
then the variables to read in are stored in the first line of the data file separated
with commas.?? Also the … -shortcut can be used to define the varaible list. If no arguments are given and
there is readfirst-> option then the variable names are read from the second
line.
\end{changemargin}
\vspace{0.3cm}
\hline
\vspace{0.3cm}
\noindent \textcolor{blue}{in}  \tabto{3cm} 0- \tabto{5cm}  Char \tabto{7cm}
\begin{changemargin}{3cm}{0cm}
\noindent input file or list of input files. If no files are given, data is read from the following input
paragraph. If either of \textcolor{blue}{read->} or \textcolor{blue}{in->} option is given, then both options must
be present.
\end{changemargin}
\vspace{0.3cm}
\hline
\vspace{0.3cm}
\noindent \textcolor{blue}{form}  \tabto{3cm} -1|1 \tabto{5cm}  Char \tabto{7cm}
\begin{changemargin}{3cm}{0cm}
\noindent  Format of the data as follows \newline
\$  Fortran format '*', the default \newline
b    Single precison binary \newline
bs  Single precison binary opened with access='stream'
Needed for Pascal files in Windows. \newline
B  Double precison binary.\newline
Char giving a Fortran format, e.g. '(4f4.1,1x,f4.3)' \newline
d4 Single precison direct access for Gfortran files.\newline
d1 Single precison direct acces for Intel Fortran files.
\end{changemargin}
\vspace{0.3cm}
\hline
\vspace{0.3cm}
\noindent \textcolor{blue}{maketrans} \tabto{3cm} -1|1 \tabto{5cm}  TRANS  \tabto{7cm}
\begin{changemargin}{3cm}{0cm}
\noindent Transformations computed for each observation when reading the data


\end{changemargin}
\vspace{0.3cm}
\hline
\vspace{0.3cm}
\noindent \textcolor{blue}{keep} \tabto{3cm} -1|1- \tabto{5cm}  REAL \tabto{7cm}
\begin{changemargin}{3cm}{0cm}
\noindent  variables kept in the data object, default: all \textcolor{blue}{read->} variables plus the output
variables of \textcolor{blue}{maketrans->} transformations.

\end{changemargin}
\vspace{0.3cm}
\hline
\vspace{0.3cm}
\noindent \textcolor{blue}{obs} \tabto{3cm} -1|1 \tabto{5cm}  REAL \tabto{7cm}
\begin{changemargin}{3cm}{0cm}
\noindent  Variable which gets automatically the observation number when working with the
data, variable is not stored in the data matrix, default: Obs. When working with
hierarchical data it is reasonable to give obs variable for each data object.

\end{changemargin}
\vspace{0.3cm}
\hline
\vspace{0.3cm}
\noindent \textcolor{blue}{filter} \tabto{3cm} -1|1 \tabto{5cm}  Code \tabto{7cm}
\begin{changemargin}{3cm}{0cm}
\noindent  logical or arithmetic statement (nonzero value indicating True) describing which
observations will be accepted to the data object. \textcolor{blue}{maketrans->}-transformations are
computed before using filter. Option \textcolor{blue}{filter->} can utilize automatically created
variable Record which tells which input record has been just read. If observations
are rejected, then the Obs-variable has as its value number of already accepted
observations+1.

\end{changemargin}
\vspace{0.3cm}
\hline
\vspace{0.3cm}
\noindent \textcolor{blue}{reject}  \tabto{3cm} -1|1 \tabto{5cm}  Code \tabto{7cm}
\begin{changemargin}{3cm}{0cm}
\noindent  Logical or arithmetic statement (nonzero value indicating True) describing which
observations will be rejected from the data object. If \textcolor{blue}{filter->} option is given then
reject statement is checked for observations which have passed the filter. Option
\textcolor{blue}{reject->} can utilize automatically created variable Record which tells which
input record has been just read. If observations are rejected, then the Obsvariable has as its value number of already accepted observations+1.
subdata the name of the lower level data object to be created. This option is not allowed, if
there are multiple input files defined in option \textcolor{blue}{in->}.

subread,…,subobs sub data options similar as \textcolor{blue}{read->}…obs-> for the upper level data.
(\textcolor{blue}{subform->}'bgaya' is the format for the Gaya system). The following options
can be used only if \textcolor{blue}{subdata->} is present

\end{changemargin}
\vspace{0.3cm}
\hline
\vspace{0.3cm}
\noindent \textcolor{blue}{nobsw}  \tabto{3cm}  -1|1  \tabto{5cm}  REAL \tabto{7cm}
\begin{changemargin}{3cm}{0cm}
\noindent  A variable in the upper data telling how many subdata observations there is under
each upper level observation, necessary if \textcolor{blue}{subdata->} option is present.

\end{changemargin}
\vspace{0.3cm}
\hline
\vspace{0.3cm}
\noindent \textcolor{blue}{nobswcum} \tabto{3cm} -1|1 \tabto{5cm}  REAL \tabto{7cm}
\begin{changemargin}{3cm}{0cm}
\noindent  A variable telling the cumulative number of subdata observations up to the
current upper data observation but not including it. This is useful when accessing
the data matrix one upper level unit by time, i.e., the observation numbers within
upper level observation are nobswcum+1,…,nobswcum+nobsw

\end{changemargin}
\vspace{0.3cm}
\hline
\vspace{0.3cm}
\noindent \textcolor{blue}{obsw} \tabto{3cm} -1|1 \tabto{5cm}   REAL  \tabto{7cm}
\begin{changemargin}{3cm}{0cm}
\noindent  A variable in the subdata which automatically will get the number of observation
within the current upper level observation, i.e. obsw variable gets values from 1
to the value of nobsw-variable, default is 'obs\_variable\%obsw'.

\end{changemargin}
\vspace{0.3cm}
\hline
\vspace{0.3cm}
\noindent \textcolor{blue}{duplicate} \tabto{3cm} -1|2 \tabto{5cm}   TRANS  \tabto{7cm}
\begin{changemargin}{3cm}{0cm}
\noindent \noindent duplicate \tabto{3cm} -1|2 \tabto{5cm}   TRANS  \tabto{7cm}
The two transformation object arguments describe how observations in the subdata
will be duplicated. The first transformation object should have Duplicates as an
output variable so that the value of Duplicates tells how many duplicates ar
made (0= no duplication). The second transformation object defines how the values
of subdata variables are determined for each duplicate. The number of duplicate
is transmitted to the variable Duplicate. These transformations are called also
when Duplicate=0. This means that when there is the \textcolor{blue}{duplicate->} option,
then all transformations for the subdata can be defined in the duplicate
transformation object, and \textcolor{blue}{submaketrans->} is not necessary.

\end{changemargin}
\vspace{0.3cm}
\hline
\vspace{0.3cm}
\noindent \textcolor{blue}{oldsubobs} \tabto{3cm} -1|1 \tabto{5cm}   REAL \tabto{7cm}
\begin{changemargin}{3cm}{0cm}
\noindent  If there are duplications of sub-observations, then this option gives the variable
into which the original observation number is put. This can be stored in the
subdata by putting it into \textcolor{blue}{subkeep->} list, or, if \textcolor{blue}{subkeep->} option is not given
then this variable is automatically put into the \textcolor{blue}{keep->} list of the subdata.

\end{changemargin}
\vspace{0.3cm}
\hline
\vspace{0.3cm}
\noindent \textcolor{blue}{oldobsw} \tabto{3cm} -1|1 \tabto{5cm}  REAL \tabto{7cm}
\begin{changemargin}{3cm}{0cm}
\noindent  This works similarly with respect to the initial obsw variable as \textcolor{blue}{oldsubobs->}
works for initial obs variable.

\end{changemargin}
\vspace{0.3cm}
\hline
\vspace{0.3cm}
\noindent \textcolor{blue}{nobs} \tabto{3cm} -1|1 \tabto{5cm}  Real \tabto{7cm}
\begin{changemargin}{3cm}{0cm}
\noindent  There are two uses of this option. First, a data object can be created without reading
from a file or from the following input paragraph by using \textcolor{blue}{nobs->} option and
\textcolor{blue}{maketrans->} transformation, which can use Obs variable as argument. Creation
of data object this way is indicated by the presence of \textcolor{blue}{nobs->} option and absence
of \textcolor{blue}{in->} and \textcolor{blue}{read->} options. Second, if \textcolor{blue}{read->} option is present \textcolor{blue}{nobs->}
option can be used to indicate how many records are read from a file and what
will be the number of observations. Currently \textcolor{blue}{reject->} or \textcolor{blue}{filter->} can not
be used to reject records (consult authors if this would be needed). If there are
fewer records in file as given in \textcolor{blue}{nobs->} option, an error occurs. There are three
reasons for using \textcolor{blue}{nobs->} option this way. First, one can read a small sample
from a large file for testing purposes. Second, the reading is slightly faster as the
data can be read directly into proper memory area without using linked buffers.
Third, if the data file is so large that a virtual memory overflow occurs, then it may
be possible to read data in as linked buffers are not needed.
In case \textcolor{blue}{nobs->} option is present and \textcolor{blue}{read->} option is absent either
\textcolor{blue}{maketrans->} or \textcolor{blue}{keep->} option (or both) is required.
\end{changemargin}
\vspace{0.3cm}
\hline
\vspace{0.3cm}
\noindent \textcolor{blue}{buffersize} \tabto{3cm} -1|1 \tabto{5cm}  Real \tabto{7cm}
\begin{changemargin}{3cm}{0cm}
\noindent \noindent buffersize \tabto{3cm} -1|1 \tabto{5cm}  Real \tabto{7cm}
The number of observations put into one temporary working buffer. The default
is 10000. Experimentation with different values of \textcolor{blue}{buffersize->} in huge data
objects may result in more efficient \textcolor{blue}{buffersize->} than is the default (or perhaps
not). Note that the buffers are not needed if number of observations is given in
\textcolor{blue}{nobs->}.

\end{changemargin}
\vspace{0.3cm}
\hline
\vspace{0.3cm}
\noindent \textcolor{blue}{par} \tabto{3cm} -1|1- \tabto{5cm}   Real \tabto{7cm}
\begin{changemargin}{3cm}{0cm}
\noindent additional parameters for reading. If \textcolor{blue}{subform->} option is 'bgaya' then par
option can be given in form \textcolor{blue}{par->}(ngvar,npvar) where ngvar is the number
of nonperiodic x-variables and npvar is the number of period specific x-variables
for each period. Default values are \textcolor{blue}{par->}(8,93).

\end{changemargin}
\vspace{0.3cm}
\hline
\vspace{0.3cm}
\noindent \textcolor{blue}{rfhead}  \tabto{3cm} -1|0 \tabto{5cm}    \tabto{7cm}
\begin{changemargin}{3cm}{0cm}
\noindent  When reading data from a text file, the first line can contain a header which is
printed but othewise ignored

\end{changemargin}
\vspace{0.3cm}
\hline
\vspace{0.3cm}
\noindent \textcolor{blue}{rfcode}  \tabto{3cm} -1|0 \tabto{5cm}    \tabto{7cm}
\begin{changemargin}{3cm}{0cm}
\noindent The data file can contain also J-code which is first executed. Note the code can
be like var1,var,x1...x5=1,2,3,4,5,6,7, which give the possibility to
define variables which describe the \textcolor{blue}{in->} file.


rfsubhead-> works for subdata similarly as \textcolor{blue}{rfhead->} for data.
rfsubcode works for subdata similarly as \textcolor{blue}{rfcode->} for data

If there are both \textcolor{blue}{rfhead->} and \textcolor{blue}{rfcode->} then \textcolor{blue}{rfhead->} is excuted first.
\textcolor{blue}{rfhead->} and \textcolor{blue}{rfcode->} replace readfirst-> option  of previous versions which was too
complicated.
\end {changemargin}
\hline
\vspace{0.2cm}

\begin{note}
\textcolor{VioletRed}{data}() function will create a data object object, which is a compound object consisting
of links to data matrix, etc. see Data object object. If Data is the output
of the function, the function creates the list Data\%keep telling the
variables in the data and
Data\%matrix containg the data as a single precision matrix. The number of observations can be obtained by \textcolor{VioletRed}{nobs}(Data) or by
\textcolor{VioletRed}{nrows}(Data\%matrix).
\end{note}
\begin{note}
See common options section for how data objects used in other \textbf{J} functions will be defined.
\end{note}
\begin{note}
The \textcolor{blue}{in->} and \textcolor{blue}{subin->} can refer to the same file, or if both are without arguments
then data are in the following input paragraph. In this case \textcolor{VioletRed}{data}() function read first one
upper level record and then \textcolor{blue}{nobsw->} lower level records.
\end{note}
\begin{note}
When reading the data the \textcolor{blue}{obs->}variable (default Obs) can be used in maketrans-
> transformation and in \textcolor{blue}{reject->} option and \textcolor{blue}{filter->} option, and the variable refers to
the number of observation in resulting data object. The variable Record gets the number of
the read record in the input file, and can be used in \textcolor{blue}{maketrans->} transformations and in
\textcolor{blue}{reject->} and \textcolor{blue}{filter->} options. If \textcolor{blue}{subdata->} option is given, variable Subreject gets
the number of record in the sub file, and it can be used in \textcolor{blue}{submaketrans->} transformations
and in \textcolor{blue}{subreject->} option and in \textcolor{blue}{subfilter->} option.
\end{note}
\begin{note}
Options \textcolor{blue}{nobs->}100, \textcolor{blue}{reject->}(Record.gt.100) and \textcolor{blue}{filter->}
(Record.le.100) result in the same data object, but when reading a large file, the \textcolor{blue}{nobs->}
option is faster as the whole input file is not read.
\end{note}
\begin{note}
If no observations are rejected, obs variable and Record variable get the same values.
\end{note}
\begin{note}
If virtual memory overflow occurs, see \textcolor{blue}{nobs->} optio. This should not happen easily with the currrent
64-bit application.
\end{note}
\begin{note}
Earlier versions contained \textcolor{blue}{trans->} and \textcolor{blue}{subtrans->}options which associated
a permanent transformation object with the data object. This feature is now deleted because
it may confuse and is not really needed. If tranformations are needed in functions
they can always be included using \textcolor{blue}{trans->} .
\end{note}
\begin{example}[dataex]data() generates a new data object by reading data.\\
\label{dataex}
data1=\textcolor{VioletRed}{data}(\textcolor{blue}{read->}(x1...x3),\textcolor{blue}{in->})\\
1,2,3\\
4,5,6\\
7,8,9\\
/
\end{example}
\subsection{\textcolor{VioletRed}{newdata}()}\index{newdata()}
\label{newdata}
Function \textcolor{VioletRed}{newdata}() generates a new data object from existing data objects and/or
matrices possibly using transformations to generate new variables.
\vspace{0.3cm}
\hline
\vspace{0.3cm}
\noindent Output \tabto{3cm} 1 \tabto{5cm}  Data \tabto{7cm}
\begin{changemargin}{3cm}{0cm}
\noindent The data object generated.
\end{changemargin}
\vspace{0.3cm}
\hline
\vspace{0.3cm}
\noindent Args \tabto{3cm} 1- \tabto{5cm}  Data|Matrix \tabto{7cm}
\begin{changemargin}{3cm}{0cm}
\noindent  Input matrices and data objects.
\end{changemargin}
\vspace{0.3cm}
\hline
\vspace{0.3cm}
\noindent \textcolor{blue}{read} \tabto{3cm} N|1- \tabto{5cm}  REAL \tabto{7cm}
\begin{changemargin}{3cm}{0cm}
\noindent  Variable names for columns of matrices in the order of
matrices.
\end{changemargin}
\vspace{0.3cm}
\hline
\vspace{0.3cm}
\noindent \textcolor{blue}{maketrans} \tabto{3cm} N|1 \tabto{5cm}   TRANS  \tabto{7cm}
\begin{changemargin}{3cm}{0cm}
\noindent  A predefined ransformation object computed for each observation.
\end {changemargin}
\hline
\vspace{0.2cm}
\begin{note}
It is not yet possible to drop variables.
\end{note}
\begin{note}
An error occurs if the same variable is several times in the variable list obtained
by combining variables in data sets and \textcolor{blue}{read->} variables.
\end{note}
\begin{note}
An error occurs if the numbers of rows of matrices and observations in data sets
are not compatible.
\end{note}
\begin{note}
Output variables in \textcolor{blue}{maketrans->} transformations whose name start with \$ are not put into the new data object.
\end{note}
\begin{example}[newdataex]newdata() generates a new data object.\\
\label{newdataex}
data1=\textcolor{VioletRed}{data}(\textcolor{blue}{read->}(x1...x3),\textcolor{blue}{in->})\\
1,2,3\\
4,5,6\\
7,8,9\\
/\\
matrix1=\textcolor{VioletRed}{matrix}(3,2,\textcolor{blue}{in->})\\
10,20\\
30,40\\
50,60\\
/\\
newtr=\textcolor{VioletRed}{trans}()\\
\textcolor{Red}{;do}(i,1,3)\\
\textcolor{Red}{;do}(j,1,2)\\
x"i"\#z"j"=x"i"*z"j"\\
\textcolor{Red}{;enddo}\\
\textcolor{Red}{;enddo}\\
/\\
new=\textcolor{VioletRed}{newdata}(data1,matrix1,\textcolor{blue}{read->}(z1,z2),\textcolor{blue}{maketrans->}newtr)\\
\textcolor{VioletRed}{print}(new)
\end{example}
\subsection{\textcolor{VioletRed}{exceldata}()}\index{exceldata()}
\label{exceldata}
Generates data object from csv data generated with excel. It is assumed that ';' is used as column separator,
and first is the header line generated with excel and containing column names.
The second line contains information for \textbf{J} how to read the data.
First the first line is copied and pasted as the second line. To the beginning of the second line is put
'@\#'. Then each entry separated by ';' is edited as follows. If the column is just ignored, then
put '!' to the beginning of the entry. If all characters in the column are read in as
a numeric variable, change the name to accpetable variable name in J.
If the column is read in but it is just used as an input variable fot
\textcolor{blue}{maketrans->} trasformations, then start the name with '\$' so the variable is not put to
the list of \textcolor{blue}{keep->} variables. If a contains only character values then it must
be ignored using '!'. If the contains numeric values surrounded by characters, the the numeric value can be picked
as follows. Put '?' to the end of entry. Put the variable name to the beginning of the entry.
then put the the number of characters to be ignored by two digits, inserting
aleading zero if needed. The given the length of the numeric field to be read in as a numeric value.
For instance, if the header line in the excel file is
\color{Green}
\begin{verbatim}
Block;Contract;Starting time;Name of municipality;Number of stem;Species code
\end{verbatim}
\color{Black}
and the first data line could be
\color{Green}
\begin{verbatim}
MG_H100097362501;20111001;7.5.2021 9:37;Akaa;20;103;1;FI2_Spruce
\end{verbatim}
\color{Black}
then the second line before the first data line could be
\color{Green}
\begin{verbatim}
##block0808?;!Contract;!Starting time;!Name of municipality;stem;species0201?
\end{verbatim}
\color{Black}
therafter the first observation would get values block=97362501,stem=1, and
species=2.

If there are several input files, the header line of later input lines is ignored, and
also if the second line of later files starts with '\#\#', then it is ignored.
if any later lines in any input files start with 'jcode:', then the code is computed.
This way variables decribing the whole input file can be transmitted to the data.
Currently jcode-output variables can be transmitted to data matrix only by using the as pseudo
outputvariables in maketrans-transformations, e.g., filevar1=filevar1, if
filevar1 is generated in jcode transformation.
If there are several input files the file number is put into variable In before computing maketrans transformations
and this variable is automatically stored in the data matrix.
\vspace{0.3cm}
\hline
\vspace{0.3cm}
\noindent Output \tabto{3cm} 1 \tabto{5cm}  Data \tabto{7cm}
\begin{changemargin}{3cm}{0cm}
\noindent  Data object generated
\end{changemargin}
\vspace{0.3cm}
\hline
\vspace{0.3cm}
\noindent \textcolor{blue}{in} \tabto{3cm} 1- \tabto{5cm}  Char \tabto{7cm}
\begin{changemargin}{3cm}{0cm}
\noindent Files to read in.
\end{changemargin}
\vspace{0.3cm}
\hline
\vspace{0.3cm}
\noindent \textcolor{blue}{maketrans} \tabto{3cm} N|1 \tabto{5cm}  trans \tabto{7cm}
\begin{changemargin}{3cm}{0cm}
\noindent  Transformations used to compute new variables to be stored
in the data.
\end {changemargin}
\hline
\vspace{0.2cm}
\subsection{\textcolor{VioletRed}{linkdata}() links hierarchical data sets}\index{linkdata()}
\label{linkdata}
\textcolor{VioletRed}{linkdata}(\textcolor{blue}{data->},\textcolor{blue}{subdata->},\textcolor{blue}{nobsw->}[,\textcolor{blue}{obsw->}])//
links hierarchical data sets.
\vspace{0.3cm}
\hline
\vspace{0.3cm}
\noindent \textcolor{blue}{data} \tabto{3cm} 1 \tabto{5cm}  DATA \tabto{7cm}
\begin{changemargin}{3cm}{0cm}
\noindent  the upper level data set object
\end{changemargin}
\vspace{0.3cm}
\hline
\vspace{0.3cm}
\noindent \textcolor{blue}{subdata}  \tabto{3cm} 1 \tabto{5cm}  DATA \tabto{7cm}
\begin{changemargin}{3cm}{0cm}
\noindent  the lower data set object
\end{changemargin}
\vspace{0.3cm}
\hline
\vspace{0.3cm}
\noindent \textcolor{blue}{nobsw} \tabto{3cm} 1 \tabto{5cm}  REAL  \tabto{7cm}
\begin{changemargin}{3cm}{0cm}
\noindent  the name of variable telling the number of lower level observations for each
\end{changemargin}
\vspace{0.3cm}
\hline
\vspace{0.3cm}
\noindent \textcolor{blue}{obsw}  \tabto{3cm}  0|1  \tabto{5cm}  REALV  \tabto{7cm}
\begin{changemargin}{3cm}{0cm}
\noindent variable which will automatically get the number of lower level observation within
each upper level observation. If not given, then this variable will be
the Obs-variable of the upper level data.
\end {changemargin}
\hline
\vspace{0.2cm}
\begin{note}
In most cases links between data sets can be either made using sub-options of \textcolor{VioletRed}{data}()
function or \textcolor{VioletRed}{linkdata}() function. If there is need to duplicate lower level observations, then
this can be currently made only in \textcolor{VioletRed}{data}() function. Also when the data for both the upper
level and lower level data are read from the same file, then \textcolor{VioletRed}{data}() function must be used.
\end{note}
\begin{note}
When using linked data in other functions, the values of the upper level variables are
automatically obtained when accessing lower level observations. Which is the observational
unit in each function is determined which data set is given in \textcolor{blue}{data->} option or defined using
Data list.
\end{note}
\begin{note}
In the current version of \textbf{J} it is no more necassary to use linked data sets in
\textcolor{VioletRed}{jlp}() function, as the treatment unit index in data containing both
stand and schdedule data can be given in unit-> option
\end{note}
\subsection{\textcolor{VioletRed}{getobs}() loads an obsevarion from  DATA}\index{getobs()}
\label{getobs}
Getting an observation from a data set: //
\textcolor{VioletRed}{getobs}(dataset,obs[,\textcolor{blue}{trans->}])//
Get the values of all variables associated with observation obs in data object dataset. First all the
variables stored in row obs in the data matrix are put into the corresponding real variables. If
a transformation set is permanently associated with the data object, these transformations are
executed.
\vspace{0.3cm}
\hline
\vspace{0.3cm}
\noindent dataset \tabto{3cm} 1 \tabto{5cm}  DATA  \tabto{7cm}
\begin{changemargin}{3cm}{0cm}
\noindent  the DATA object
\end{changemargin}
\vspace{0.3cm}
\hline
\vspace{0.3cm}
\noindent \textcolor{blue}{obs}  \tabto{3cm} 1 \tabto{5cm}   REAL \tabto{7cm}
\begin{changemargin}{3cm}{0cm}
\noindent  row number in the data matrix of the dataset
\end{changemargin}
\vspace{0.3cm}
\hline
\vspace{0.3cm}
\noindent \textcolor{blue}{trans} \tabto{3cm} -1|1  \tabto{5cm}   TRANS  \tabto{7cm}
\begin{changemargin}{3cm}{0cm}
\noindent  these transformations are also executed.
\end {changemargin}
\hline
\vspace{0.2cm}
\subsection{\textcolor{VioletRed}{nobs}(): number of observations in DATA or REGR}\index{nobs()}
\label{nobs}
\textcolor{VioletRed}{nobs}(DATA) returns the number of rows in the data matrix of DATA//
\textcolor{VioletRed}{nobs}(REGR) returns the number of observations used to compute
the regression with \textcolor{VioletRed}{regr}().
\subsection{\textcolor{VioletRed}{classvector}()}\index{classvector()}
\label{classvector}
Function classvector computes vectors from data which extract information from grouped
data. These vectors can be used to generate new data object using \textcolor{VioletRed}{newdata}() function or
new matrices from submatrices using \textcolor{VioletRed}{matrix}() function with \textcolor{blue}{matrix->} option or
they can be used in transformation objects to compute class related things.
There is no explicit output for the function, but several output vectors can
be generated depending on the arguments and \textcolor{blue}{first->}, \textcolor{blue}{last->} and
\textcolor{blue}{expand->} options. The function prints the names of the output vectors generated.
\vspace{0.3cm}
\hline
\vspace{0.3cm}
\noindent Args \tabto{3cm} 0- \tabto{5cm}  REAL \tabto{7cm}
\begin{changemargin}{3cm}{0cm}
\noindent  The variables whose class information is computed. Arguments
are not necessary if \textcolor{blue}{first->} and/or \textcolor{blue}{last->} are present.
Let §Arg be the generic name for arguments.
\end{changemargin}
\vspace{0.3cm}
\hline
\vspace{0.3cm}
\noindent \textcolor{blue}{class} \tabto{3cm} 1 \tabto{5cm}  REAL \tabto{7cm}
\begin{changemargin}{3cm}{0cm}
\noindent  .oindent class \tabto{3cm} 1 \tabto{5cm}  REAL \tabto{7cm}
\end{changemargin}
\vspace{0.3cm}
\hline
\vspace{0.3cm}
\noindent \textcolor{blue}{class} \tabto{3cm} 1 \tabto{5cm}  REAL \tabto{7cm}
\begin{changemargin}{3cm}{0cm}
\noindent The variable indicating the class. The class variable which must be present in the data object or which is
an output variable of the \textcolor{blue}{trans->} transformations.
When the \textcolor{blue}{class->} variable, denoted as
as §Class changes,
the class changes.
\end{changemargin}
\vspace{0.3cm}
\hline
\vspace{0.3cm}
\noindent \textcolor{blue}{data} \tabto{3cm} 0|1 \tabto{5cm}  Data \tabto{7cm}
\begin{changemargin}{3cm}{0cm}
\noindent Data object used. Only one data object used; extra \textcolor{blue}{data->} objects just ignored. The default is the last
data object generated.
\end{changemargin}
\vspace{0.3cm}
\hline
\vspace{0.3cm}
\noindent \textcolor{blue}{expand} \tabto{3cm} -1|0 \tabto{5cm}    \tabto{7cm}
\begin{changemargin}{3cm}{0cm}
\noindent If \textcolor{blue}{expand->} is present then the lengths output vectors are equal
to the number of observations in the data object and the values of the class variables
are repeated as many times as there are observations in each class. If
\textcolor{blue}{expand->} is not present, the lengths of the output vectors are.
equal to the number of classes.
\end{changemargin}
\vspace{0.3cm}
\hline
\vspace{0.3cm}
\noindent \textcolor{blue}{first} \tabto{3cm} 0 \tabto{5cm}    \tabto{7cm}
\begin{changemargin}{3cm}{0cm}
\noindent The the number of first observation in class is stored in vector
§Class\%\%first if \textcolor{blue}{expand->} is present and §Class\%first if \textcolor{blue}{expand->} is not present.
\end{changemargin}
\vspace{0.3cm}
\hline
\vspace{0.3cm}
\noindent \textcolor{blue}{last} \tabto{3cm} 0 \tabto{5cm}    \tabto{7cm}
\begin{changemargin}{3cm}{0cm}
\noindent The the number of lastt observation in class is stored in vector
§Class\%\%last if \textcolor{blue}{expand->} is present and §Class\%last if \textcolor{blue}{expand->} is not present.
\end{changemargin}
\vspace{0.3cm}
\hline
\vspace{0.3cm}
\noindent \textcolor{blue}{obsw} \tabto{3cm} 0 \tabto{5cm}    \tabto{7cm}
\begin{changemargin}{3cm}{0cm}
\noindent If there axpnad-> option then vector Class\%\%obsw
\end{changemargin}
\vspace{0.3cm}
\hline
\vspace{0.3cm}
\noindent \textcolor{blue}{ext} \tabto{3cm} -1|1 \tabto{5cm}  Char \tabto{7cm}
\begin{changemargin}{3cm}{0cm}
\noindent The extension to the names of vectors generated for arguments. Let
Ext be denote the extension.
\end{changemargin}
\vspace{0.3cm}
\hline
\vspace{0.3cm}
\noindent \textcolor{blue}{mean} \tabto{3cm} -1|0 \tabto{5cm}    \tabto{7cm}
\begin{changemargin}{3cm}{0cm}
\noindent The class means are stored in the vectors \newline
§Arg\#Class\%\%mean with \textcolor{blue}{expand->} and without \textcolor{blue}{ext->}\newline
§Arg\#Class\%\%meanExt with \textcolor{blue}{expand->} and with \textcolor{blue}{ext->} are \newline
§Arg\#Class\%mean without \textcolor{blue}{expand->} and without \textcolor{blue}{ext->} \newline
§Arg\#Class\%meanExt without \textcolor{blue}{expand->} and with \textcolor{blue}{ext->}
\end{changemargin}
\vspace{0.3cm}
\hline
\vspace{0.3cm}
\noindent \textcolor{blue}{sd} \tabto{3cm} -1|0 \tabto{5cm}    \tabto{7cm}
\begin{changemargin}{3cm}{0cm}
\noindent  Class standard deviations are computed to sd vectors
\end{changemargin}
\vspace{0.3cm}
\hline
\vspace{0.3cm}
\noindent \textcolor{blue}{var} \tabto{3cm} -1|0 \tabto{5cm}    \tabto{7cm}
\begin{changemargin}{3cm}{0cm}
\noindent  Class variances are computed to var vectors
\end{changemargin}
\vspace{0.3cm}
\hline
\vspace{0.3cm}
\noindent \textcolor{blue}{min} \tabto{3cm} -1|0 \tabto{5cm}    \tabto{7cm}
\begin{changemargin}{3cm}{0cm}
\noindent  Class minimums are computed to min vectors
\end{changemargin}
\vspace{0.3cm}
\hline
\vspace{0.3cm}
\noindent \textcolor{blue}{max} \tabto{3cm} -1|0 \tabto{5cm}    \tabto{7cm}
\begin{changemargin}{3cm}{0cm}
\noindent  Class maximums are computed to max vectors.
\end {changemargin}
\hline
\vspace{0.2cm}
\begin{note}
Numbers of observations in each class can be obtained by \\
Class\%nobs=Class\%\%last-Class\%\%first+1 when \textcolor{blue}{expand->} is present, and \\
Class\%nobs=Class\%\%last-Class\%\%first+1
\end{note}
\begin{example}[newclassdata]Making class level data object\\
\label{newclassdata}
\textcolor{VioletRed}{classvector}(x1,x2,\textcolor{blue}{class->}stand,\textcolor{blue}{data->}treedata,\textcolor{blue}{mean->},\textcolor{blue}{min->})\\
standdata=\textcolor{VioletRed}{newdata}(x1\#stand\%mean,x2\#stand\%mean,x1\#stand\%min,x2\#stand\%min,\\
\textcolor{blue}{read->}(x1,x2,x1min,x2min))
\end{example}
\begin{example}[addingclass]Adding class means and deviations from class means\\
\label{addingclass}
\textcolor{VioletRed}{classvector}(x1,x2,\textcolor{blue}{class->}stand,\textcolor{blue}{data->}treedata,\textcolor{blue}{mean->},\textcolor{blue}{expand->})\\
tr=\textcolor{VioletRed}{trans}()\\
relx1=x1-x1mean\\
relx2=x2-x2mean\\
/\\
treedata=\textcolor{VioletRed}{newdata}(treedata,x1\#stand\%mean,x2\#stand\%mean,\textcolor{blue}{read->}(x1mean,x2mean),\\
\textcolor{blue}{maketrans->}tr)
\end{example}

\subsection{\textcolor{VioletRed}{values}() extracts values of class variables}\index{values()}
\label{values}
Extracting values of class variables: \textcolor{VioletRed}{values}( ).
\vspace{0.3cm}
\hline
\vspace{0.3cm}
\noindent Output \tabto{3cm} 1 \tabto{5cm}  VECTOR \tabto{7cm}
\begin{changemargin}{3cm}{0cm}
\noindent  the vector getting differen values
\end{changemargin}
\vspace{0.3cm}
\hline
\vspace{0.3cm}
\noindent \textcolor{blue}{arg} \tabto{3cm} 1 \tabto{5cm}  REALV \tabto{7cm}
\begin{changemargin}{3cm}{0cm}
\noindent  variables whose values obtained
\end{changemargin}
\vspace{0.3cm}
\hline
\vspace{0.3cm}
\noindent \textcolor{blue}{data} \tabto{3cm} 1 \tabto{5cm}  DATA \tabto{7cm}
\begin{changemargin}{3cm}{0cm}
\noindent  The data set.
\end {changemargin}
\hline
\vspace{0.2cm}
\begin{note}
The values found will be sorted in an increasing order.
\end{note}
\begin{note}
After getting the values into a vector,
the number of different values can be obtained
using \textcolor{VioletRed}{nrows}() function.
\end{note}

\begin{note}
\textcolor{VioletRed}{values}() function can be utilized e.g. in generating domains for all different
owners or regions found in data.
\end{note}
\subsection{\textcolor{VioletRed}{transdata}() own transformations for data}\index{transdata()}
\label{transdata}
\textcolor{VioletRed}{transdata}() is useful when all necassy computions are put into a TRANS
object, and a DATA object is gone throug obsevation by observation.
This is useful e.g. when simulating harvesting schdules using a simulator which is defined
as an ordinary TRANS object. The whole function is written below to indicate
how users' own functions dealing with data could be developped.
@@data
\hline
\vspace{0.2cm}

\color{Green}
\begin{verbatim}
subroutine transdata(iob,io)
call j_getdataobject(iob,io)
if(j_err)return
call j_clearoption(iob,io)  ! subroutine

do iobs=j_dfrom,j_duntil
call j_getobs(iobs)
if(j_err)return
end do !do iobs=j_dfrom,j_duntil

if(j_depilog.gt.0)call dotrans(j_depilog,1)

return
\end{verbatim}
\color{Black}
\section{Statistical functions}
\label{statistics}
There are several statistical functions which can be used to compute basic statistics
linear and and nonlinear regression, class means, standard deviations and standard errors
in one or two dimensional tables using data sets. There are also functions
which can be used to compute statistics from matrices, but these are described
in Section  \ref{matrix}
\subsection{\textcolor{VioletRed}{stat}()}\index{stat()}
\label{stat}
Computes and prints basic statistics from data objects.
\vspace{0.3cm}
\hline
\vspace{0.3cm}
\noindent Output \tabto{3cm} 0-1 \tabto{5cm}  REAL \tabto{7cm}
\begin{changemargin}{3cm}{0cm}
\noindent  kokopo
\end{changemargin}
\vspace{0.3cm}
\hline
\vspace{0.3cm}
\noindent Args  \tabto{3cm}  0-99 \tabto{5cm}  REAL \tabto{7cm}
\begin{changemargin}{3cm}{0cm}
\noindent variables for which the statistics are computed,
the default is all variables in the data (all variables in the data matrix plus the output variables of the associated transformation object) and all output variables of the tran

@@data
\end{changemargin}
\vspace{0.3cm}
\hline
\vspace{0.3cm}
\noindent \textcolor{blue}{data}  \tabto{3cm}  -1,99  \tabto{5cm}   Data  \tabto{7cm}
\begin{changemargin}{3cm}{0cm}
\noindent 	data objects , see section Common options for default\textcolor{green}{!\,weight\,	gives\,the\,weight\,of\,each\,observations\,if\,weighted\,means\,and\,variances\,are\,computed.\,The\,weigh}
transformation or it can be a variable in the data object
@@seecom
\end{changemargin}
\vspace{0.3cm}
\hline
\vspace{0.3cm}
\noindent \textcolor{blue}{min}  \tabto{3cm}  -1,99 \tabto{5cm}  REAL \tabto{7cm}
\begin{changemargin}{3cm}{0cm}
\noindent 	defines to which variables the minima are stored.
If the value is character constant or character variable,
then the name is formed by concatenating the character with the name of the argument
variable. E.g. \textcolor{VioletRed}{stat}(x1,x2,\textcolor{blue}{min->}'\%pien') stores minimums into variables
x1\%pien and x2\%pien. The default value for min  is '\%min'.
If the values of the \textcolor{blue}{min->} option are variables,
then the minima are stored into these variables.
\end{changemargin}
\vspace{0.3cm}
\hline
\vspace{0.3cm}
\noindent \textcolor{blue}{max}  \tabto{3cm} -1,99 \tabto{5cm}  REAL \tabto{7cm}
\begin{changemargin}{3cm}{0cm}
\noindent  maxima are stored, works as \textcolor{blue}{min->}
\end{changemargin}
\vspace{0.3cm}
\hline
\vspace{0.3cm}
\noindent \textcolor{blue}{mean}  \tabto{3cm} -1,99 \tabto{5cm}  REAL  \tabto{7cm}
\begin{changemargin}{3cm}{0cm}
\noindent  means are stored
\end{changemargin}
\vspace{0.3cm}
\hline
\vspace{0.3cm}
\noindent \textcolor{blue}{var}  \tabto{3cm} -1,99 \tabto{5cm}  REAL \tabto{7cm}
\begin{changemargin}{3cm}{0cm}
\noindent  variances are stored
\end{changemargin}
\vspace{0.3cm}
\hline
\vspace{0.3cm}
\noindent \textcolor{blue}{sd}  \tabto{3cm} -1,99 \tabto{5cm}  REAL \tabto{7cm}
\begin{changemargin}{3cm}{0cm}
\noindent  standard deviations are stored
\end{changemargin}
\vspace{0.3cm}
\hline
\vspace{0.3cm}
\noindent \textcolor{blue}{sum}  \tabto{3cm} -1,99 \tabto{5cm}  REAL \tabto{7cm}
\begin{changemargin}{3cm}{0cm}
\noindent 	sums are stored, (note that sums are not printed automatically)
\end{changemargin}
\vspace{0.3cm}
\hline
\vspace{0.3cm}
\noindent \textcolor{blue}{nobs}  \tabto{3cm} -1 | 1 \tabto{5cm}  REAL \tabto{7cm}
\begin{changemargin}{3cm}{0cm}
\noindent 	gives variable which will get the number of accepted observations, default is variable 'Nnobs'. If all observations are rejected due to \textcolor{blue}{filter->} or \textcolor{blue}{reject->} opt
\end{changemargin}
\vspace{0.3cm}
\hline
\vspace{0.3cm}
\noindent \textcolor{blue}{trans}  \tabto{3cm} -1 | 1 \tabto{5cm}  TRANS \tabto{7cm}
\begin{changemargin}{3cm}{0cm}
\noindent 	transformation object which is executed for each observation. If there is a transformation object associated with the data object, those transformations are
\end{changemargin}
\vspace{0.3cm}
\hline
\vspace{0.3cm}
\noindent \textcolor{blue}{filter}  \tabto{3cm} -1 | 1 \tabto{5cm}  Code \tabto{7cm}
\begin{changemargin}{3cm}{0cm}
\noindent  logical or arithmetic statement (nonzero value indicating True) describing which observations will be accepted. \textcolor{blue}{trans->} transformations are computed before u
\end{changemargin}
\vspace{0.3cm}
\hline
\vspace{0.3cm}
\noindent \textcolor{blue}{reject}  \tabto{3cm} -1 | 1 \tabto{5cm}  Code \tabto{7cm}
\begin{changemargin}{3cm}{0cm}
\noindent \noindent reject  \tabto{3cm} -1 | 1 \tabto{5cm}  Code \tabto{7cm}
\end{changemargin}
\vspace{0.3cm}
\hline
\vspace{0.3cm}
\noindent transafter  \tabto{3cm} -1 | 1 \tabto{5cm}   TRANS \tabto{7cm}
\begin{changemargin}{3cm}{0cm}
\noindent  transformation object which is executed for each observation which has passed the filter and is not rejected by the \textcolor{blue}{reject->}-option.
\end {changemargin}
\hline
\vspace{0.2cm}
\begin{note}
1: \textcolor{VioletRed}{stat}() function prints min, max, means, sd and sd of the mean computed
as sd/\textcolor{VioletRed}{sqrt}(number of observations)
\end{note}
\begin{note}
2: If the value of a variable is greater than or equal to 1.7e19,
then that observation is rejected when computing statistics for that variable.
\end{note}
\begin{example}[statex]stat() computes minimums, maximums, means and std devaitons\\
\label{statex}
;if(\textcolor{VioletRed}{type}(data1).ne.DATA)dataex\\
\textcolor{VioletRed}{stat}()\\
\textcolor{VioletRed}{stat}(area,\textcolor{blue}{data->}cd,\textcolor{blue}{sum->}bon20,\textcolor{blue}{filter->}(site.ge.18.5))\\
\textcolor{VioletRed}{stat}(ba,\textcolor{blue}{data->}cd,\textcolor{blue}{weight->}area)\\
\textcolor{VioletRed}{stat}(vol,\textcolor{blue}{weight->}(1/dbh***2))
\end{example}
\subsection{\textcolor{VioletRed}{cov}():  covariance matrix}\index{cov()}
\label{cov}
\textcolor{VioletRed}{cov}() computes the covariance matrix of variables in DATA.
\vspace{0.3cm}
\hline
\vspace{0.3cm}
\noindent output \tabto{3cm} 1 \tabto{5cm}  MATRIX \tabto{7cm}
\begin{changemargin}{3cm}{0cm}
\noindent  symmetric aoutput matrix.
\end{changemargin}
\vspace{0.3cm}
\hline
\vspace{0.3cm}
\noindent \textcolor{blue}{arg} \tabto{3cm}  1-N \tabto{5cm}  LIST or REALV \tabto{7cm}
\begin{changemargin}{3cm}{0cm}
\noindent  variables for which covarianes are computed, listing
individually or given as a LIST.
@@data
\end{changemargin}
\vspace{0.3cm}
\hline
\vspace{0.3cm}
\noindent \textcolor{blue}{weight} \tabto{3cm} -1|1 \tabto{5cm}  CODE \tabto{7cm}
\begin{changemargin}{3cm}{0cm}
\noindent  Codeoption for weight of each observation.
\end {changemargin}
\hline
\vspace{0.2cm}
\begin{note}
the output is not automaticall printed, but it can be printed using ';'
at the end of line.
\end{note}
\begin{note}
The covariance matrix can changed into correaltion matrix with \textcolor{VioletRed}{corrmatrix}()
function.
\end{note}
\begin{note}
If variable \textcolor{teal}{w} in the data is used as the weigth, this can be expressed as
\textcolor{blue}{weight->}w
\end{note}
\begin{example}[covex]Example of covariance\\
\label{covex}
X1=\textcolor{VioletRed}{matrix}(200)\\
X1=\textcolor{VioletRed}{rann}()\\
\textcolor{Red}{;do}(i,2,6)\\
ad=\textcolor{VioletRed}{matrix}(200)\\
ad=\textcolor{VioletRed}{rann}()\\
X"i"=X"i-1"+0.6*ad\\
\textcolor{Red}{;enddo}\\
dat=\textcolor{VioletRed}{newdata}(X1...X6,\textcolor{blue}{read->}(x1...x5))\\
co=\textcolor{VioletRed}{cov}(x1...x5);\\
co=\textcolor{VioletRed}{cov}(dat\%keep);
\end{example}
\subsection{\textcolor{VioletRed}{corr}() computes a correlation matrix.}\index{corr()}
\label{corr}
\textcolor{VioletRed}{corr}(1) works similarly as \textcolor{VioletRed}{cov}()
\subsection{\textcolor{VioletRed}{regr}(): linear regression.}\index{regr()}
\label{regr}
Ordinary or stepwise linear regrwession can be computed using \textcolor{VioletRed}{regr}().

\vspace{0.3cm}
\hline
\vspace{0.3cm}
\noindent output \tabto{3cm} 1 \tabto{5cm}  REGR \tabto{7cm}
\begin{changemargin}{3cm}{0cm}
\noindent  sRegression object..
\end{changemargin}
\vspace{0.3cm}
\hline
\vspace{0.3cm}
\noindent \textcolor{blue}{arg} \tabto{3cm}  1-N \tabto{5cm}  LIST or REALV \tabto{7cm}
\begin{changemargin}{3cm}{0cm}
\noindent  y-variable and x-variables variables listing them
individually or given as a LIST.
@@data
\end{changemargin}
\vspace{0.3cm}
\hline
\vspace{0.3cm}
\noindent \textcolor{blue}{noint} \tabto{3cm} -1|0 \tabto{5cm}    \tabto{7cm}
\begin{changemargin}{3cm}{0cm}
\noindent  \textcolor{blue}{noint->} implies that the model does not include intercept
\end{changemargin}
\vspace{0.3cm}
\hline
\vspace{0.3cm}
\noindent \textcolor{blue}{step} \tabto{3cm} -1|1  \tabto{5cm}  REAL \tabto{7cm}
\begin{changemargin}{3cm}{0cm}
\noindent  t-value limit for stepwise regression. Regression variables are droped one-by-one
until the absolute value of t-value is at least as large as the limit given.
intercept is not considered.
\end{changemargin}
\vspace{0.3cm}
\hline
\vspace{0.3cm}
\noindent \textcolor{blue}{var} \tabto{3cm} -1|0 \tabto{5cm}    \tabto{7cm}
\begin{changemargin}{3cm}{0cm}
\noindent  if \textcolor{blue}{var->} is present \textcolor{VioletRed}{regr}() generated matrix ]output\%var[ for
the variance-covariance matrix of the coeffcient estimates.
\end{changemargin}
\vspace{0.3cm}
\hline
\vspace{0.3cm}
\noindent \textcolor{blue}{corr} \tabto{3cm} -1|0 \tabto{5cm}    \tabto{7cm}
\begin{changemargin}{3cm}{0cm}
\noindent  if vcorr-> is present \textcolor{VioletRed}{regr}() generated matrix ]output\%corr[ for
the correlation matrix of the coeffcient estimates. Standard deviations
are put to the diagonal.
\end {changemargin}
\hline
\vspace{0.2cm}
\begin{note}
If the DATA object contains variables Regr and Resid, then the values of
the regression function and resduals are put into these columns. Space for these e
coluns cab reserved with \textcolor{blue}{extra->} option in \textcolor{VioletRed}{data}() or in \textcolor{VioletRed}{newdata}()
\end{note}
\begin{note}
If \textcolor{teal}{re} is the output of the \textcolor{VioletRed}{regr}() then function re() can be used to compute
the value of the regression function. re() can contain from zero arguments up to the
total number of arguments as arguments. The rest of arguments get
the value they happen to have at the moment when teh function is called.
\end{note}
Information from the REGR object can be obtained with the following functions.
let \textcolor{teal}{re} be the name of the REGR object.
\begin{itemize}
\item[\textbf{J}\.]  \textcolor{VioletRed}{coef}(\textcolor{teal}{re},xvar) = coefficient of variable xvar
\item[\textbf{J}\.]  \textcolor{VioletRed}{coef}(\textcolor{teal}{re},xvar,any) = returns zero if the variable is dropped from
the equation in the setwise procsedure of
due to linear dependencies.
\item[\textbf{J}\.] \textcolor{VioletRed}{coef}(\textcolor{teal}{re},1) or \textcolor{VioletRed}{coef}(]re(,\$1) returns the intercept
\item[\textbf{J}\.] \textcolor{VioletRed}{se}(\textcolor{teal}{re},xvar) standard error of a coeffcient
\item[\textbf{J}\.] \textcolor{VioletRed}{mse}(\textcolor{teal}{re}) MSE of the regression
\item[\textbf{J}\.] \textcolor{VioletRed}{rmse}(\textcolor{teal}{re}) RMSE of the regression
\item[\textbf{J}\.]	\textcolor{VioletRed}{r2}(\textcolor{teal}{re}) adjusted R2. If the intercept is not present this can be negative.
\item[\textbf{J}\.]	\textcolor{VioletRed}{nobs}(\textcolor{teal}{re}) number of observations used
\item[\textbf{J}\.]	\textcolor{VioletRed}{len}(\textcolor{teal}{re}) number of independent variables (including intercept) used
\end{itemize}
\subsection{\textcolor{VioletRed}{nonlin}():: nonlinear regression}\index{nonlin()}
\label{nonlin}
To be raported later, see old manual
\subsection{\textcolor{VioletRed}{varcomp}(): variance and covariance components}\index{varcomp()}
\label{varcomp}
TO BE RAPORTED LATER, see old manual
\subsection{\textcolor{VioletRed}{classify}()}\index{classify()}
\label{classify}
Classifies data with respect to one or two variables, get class
frequencies,
means and standard deviations of
argument variables.
\vspace{0.3cm}
\hline
\vspace{0.3cm}
\noindent Output \tabto{3cm}  1 \tabto{5cm}  Matrix \tabto{7cm}
\begin{changemargin}{3cm}{0cm}
\noindent \noindent Output \tabto{3cm}  1 \tabto{5cm}  Matrix \tabto{7cm}
A matrix containing class information (details given below)
\end{changemargin}
\vspace{0.3cm}
\hline
\vspace{0.3cm}
\noindent Args \tabto{3cm} 1- \tabto{5cm}  REAL \tabto{7cm}
\begin{changemargin}{3cm}{0cm}
\noindent \noindent Args \tabto{3cm} 1- \tabto{5cm}  REAL \tabto{7cm}
Variables for which class means are computed.

@@data
\end{changemargin}
\vspace{0.3cm}
\hline
\vspace{0.3cm}
\noindent \textcolor{blue}{x}  \tabto{3cm} 1 \tabto{5cm}  REAL \tabto{7cm}
\begin{changemargin}{3cm}{0cm}
\noindent The first variable defining classes.
minobs minimum number of observation in a class, obtained by merging classes. Does
not work if \textcolor{blue}{z->} is given

\end{changemargin}
\vspace{0.3cm}
\hline
\vspace{0.3cm}
\noindent \textcolor{blue}{xrange} \tabto{3cm}  -1|0|2 \tabto{5cm}  Real \tabto{7cm}
\begin{changemargin}{3cm}{0cm}
\noindent  Defines the range of x variable. If \textcolor{blue}{xrange->} is given without
arguments and \textbf{J} variables x\%min and x\%max exist, they are used, and
if they do not exist an error occurs. Note that these variables can be
generate with \textcolor{VioletRed}{stat}(\textcolor{blue}{min->},\textcolor{blue}{max->}). Either xtrange-> or \textcolor{blue}{any->} must be presente.
\end{changemargin}
\vspace{0.3cm}
\hline
\vspace{0.3cm}
\noindent \textcolor{blue}{any} \tabto{3cm} -1|0 \tabto{5cm}    \tabto{7cm}
\begin{changemargin}{3cm}{0cm}
\noindent Indicates that each value of the x-variables foms a separate class.
either \textcolor{blue}{xrange->} or nay-> must be present.
\end{changemargin}
\vspace{0.3cm}
\hline
\vspace{0.3cm}
\noindent \textcolor{blue}{classes} \tabto{3cm} -1|1 \tabto{5cm}  Real \tabto{7cm}
\begin{changemargin}{3cm}{0cm}
\noindent  Number of classes, If \textcolor{blue}{dx->} is not given, the default is that range is
divided into 7 classes.
\textcolor{blue}{minobs->} minimum number of observations in one class. Classes are merged so that this can
be obtained. Does not work if \textcolor{blue}{z->} is present.
!
\end{changemargin}
\vspace{0.3cm}
\hline
\vspace{0.3cm}
\noindent \textcolor{blue}{z} \tabto{3cm} -1|1 \tabto{5cm}  REAL \tabto{7cm}
\begin{changemargin}{3cm}{0cm}
\noindent  The second variable (z variable) defining classes in two dimensional classification.
\end{changemargin}
\vspace{0.3cm}
\hline
\vspace{0.3cm}
\noindent \textcolor{blue}{zrange}  \tabto{3cm} -1|0|2 \tabto{5cm}  Real \tabto{7cm}
\begin{changemargin}{3cm}{0cm}
\noindent  Defines the range and class width for a continuous z
variable. If \textbf{J} variables x\%min and x\%max exist,
provided by \textcolor{VioletRed}{stat}(\textcolor{blue}{min->},\textcolor{blue}{max->}), they are used.
\end{changemargin}
\vspace{0.3cm}
\hline
\vspace{0.3cm}
\noindent \textcolor{blue}{dz}  \tabto{3cm} -1|1 \tabto{5cm}  Real \tabto{7cm}
\begin{changemargin}{3cm}{0cm}
\noindent  Defines the class width for a continuous z variable.
mean if z variable is given, class means are stored in a matrix given in the \textcolor{blue}{mean->}
option
classes number of classes, has effect if dx is not defined in xrangedx->. The default is
\textcolor{blue}{classes->}7. If z is given then, there can be a second argument, which gives the
number of classes for z, the default being 7.
@@trans
@@filter
@@reject

\end{changemargin}
\vspace{0.3cm}
\hline
\vspace{0.3cm}
\noindent \textcolor{blue}{print} \tabto{3cm} -1|1 \tabto{5cm}  Real \tabto{7cm}
\begin{changemargin}{3cm}{0cm}
\noindent  By setting \textcolor{blue}{print->}0, the classification matrix is not printed.
The matrix can be utilized directly in \textcolor{VioletRed}{drawclass}() function.
\end {changemargin}
\hline
\vspace{0.2cm}
\begin{note}
If z variable is not given then first column in printed output and the first row in the output
matrix (if given) contains class means of the x variable. In the output matrix the last element is
zero. Second column an TARKASTA VOISIKO VAIHTAArow shows number of observations in
class, and the last element is the total number of observations. Third row shows the class means
of the argument variable. The fourth row in the output matrix shows the class standard
deviations, and the last element is the overall standard deviation
\end{note}
\begin{note}
Variable Accepted gives the number of accepted obsevations.
\end{note}
\subsection{\textcolor{VioletRed}{class}()}\index{class()}
\label{class}
Function \textcolor{VioletRed}{class}() computes the the class of given value when classifying values
similarly as done in \textcolor{VioletRed}{classify}().
\vspace{0.3cm}
\hline
\vspace{0.3cm}
\noindent Output \tabto{3cm} 1 \tabto{5cm}  REAL \tabto{7cm}
\begin{changemargin}{3cm}{0cm}
\noindent The class number.
\end{changemargin}
\vspace{0.3cm}
\hline
\vspace{0.3cm}
\noindent Args \tabto{3cm} 1 \tabto{5cm}  Real \tabto{7cm}
\begin{changemargin}{3cm}{0cm}
\noindent The value whose class is determined.
\end{changemargin}
\vspace{0.3cm}
\hline
\vspace{0.3cm}
\noindent \textcolor{blue}{xrange} \tabto{3cm} 2 \tabto{5cm}  Real \tabto{7cm}
\begin{changemargin}{3cm}{0cm}
\noindent The range of values.
\end{changemargin}
\vspace{0.3cm}
\hline
\vspace{0.3cm}
\noindent \textcolor{blue}{dx} \tabto{3cm} N|1 \tabto{5cm}  Real \tabto{7cm}
\begin{changemargin}{3cm}{0cm}
\noindent The class width.
\end{changemargin}
\vspace{0.3cm}
\hline
\vspace{0.3cm}
\noindent \textcolor{blue}{classes} \tabto{3cm} N|1 \tabto{5cm}  Real \tabto{7cm}
\begin{changemargin}{3cm}{0cm}
\noindent The number of classes.
\end {changemargin}
\hline
\vspace{0.2cm}
\begin{note}
Either \textcolor{blue}{dx->} or \textcolor{blue}{classes->} must be given. If both are given, \textcolor{blue}{dx->} dominates.
\end{note}
\begin{note}
If \textcolor{VioletRed}{stat}() is used earlier for variables including Var1 and
options \textcolor{blue}{min->} and \textcolor{blue}{max->} are present, then
\textcolor{blue}{xrange->}(Var1\%min,Var1\%max) is assumed.
\end{note}
\section{Linear programming functions}
\label{jlpintro}
The linear programming (LP) functions are called jlp-functions. The jlp-functions can be used to define
linear programming problems, solve them and access the results.
The jlp-problems are defined using \textcolor{VioletRed}{problem}() function, and the problems can be solved using
\textcolor{VioletRed}{jlp}() function. There are several ways to access the results.
The main interest in jlp-functions may be in the forest planning problems, where
a simualator has generate treatment schedules.
There are four different applications of \textcolor{VioletRed}{jlp}() function.
\begin{itemize}
\item[\textbf{J}] Small ordinary LP problems with text input.
\item[\textbf{J}] Large ordinary LP problems with MATRIX input.
\item[\textbf{J}] Forest planning problems without factories.
\item[\textbf{J}] Forest palnning problems with factories.
\end{itemize}
The problem definition for each case is generated with \textcolor{VioletRed}{problem}() function.
These applications are presented in different subsections
\subsection{Problem definition object}
\label{problemo}
Problem definition object is a compound object produced by the \textcolor{VioletRed}{problem}() function, and it is
described in Linear programming.
\subsection{\textcolor{VioletRed}{problem}() defines a Lp-problem}\index{problem()}
\label{problem}
An LP-problem is defined in similar way as a TEXT object//
\textcolor{VioletRed}{problem}([repeatdomains->])//
�
/
\color{Green}
\begin{verbatim}
Define a lp problem for jlp() function.
Output:
a problem definition object
OOption:
repeatdomains
if this option is given then the same domain definition can be in several places of
the problem definition, otherwise having the same domain in different places
causes an error (as this is usually not what was intended). If the same domain
definition is in several places is slightly inefficient in computations, e.g. jlp()
function computes and prints the values of x-variables for each domain definition
even if the same values have been computed and printed for earlier occurrence
of the domain definition.
The problem definition paragraph can have two types of lines: problem rows and domain rows.
Examples of problem definitions showing the syntax.
pr=problem() !ordinary lp-problem
7*z2+6*z3-z4==min
2*z1+6.1*z2 >2 <8+ !both lower and upper bound is possible
(a+log(b))*z5-z8=0
-z7+z1>8
/
prx=problem() ! timber management planning problem
All:
npv.0==max
sitetype.eq.2: domain7:
income.2-income.1>0
/
In the above example domain7 is a data variable. Unit belongs to domain if the value of the
variable domain7 is anything else than zero.
The objective row must be the first row. The objective must always be present. If the purpose
is to just get a feasible solution without objective, this can be obtained by minimizing a zvariable which does not otherwise appear in the problem (remember z-> option in the jlp()
function.
In problems having large number of variables in a row it is possible to give the coefficients as
a vector and variables as a list e.g.
ESIMERKKI
In problems with x-variables it is possible to maximize or minimize the objective without any
constraints. In factory problems this would also be quite straightforward to implement, but i
In problems with x-variables it is possible to maximize or minimize the objective without any
constraints. In factory problems this would also be quite straightforward to implement, but it
does not come as a side effect of computations as in the case of maximization of x-variables,
and thus it has not been implemented. The maximization of a factory objective without
constraints can be obtained by adding to the problem constraints which require that the
amounts of transported timber assortments to different factories are greater than or equal to
zero.
Function problem() interprets the problem paragraph, and extracts the coefficients of
variables in the object row and in constraint rows. The coefficients can be defined using
arithmetic statements utilizing the input programming "-sequence or enclosing the coefficient
in parenthesis. The right hand side can utilize arithmetic computations without parenthesis.
The values are computed immediately. So if the variables used in coefficients change their
values later, the problem() function must be computed again in order to get updated
coefficients. Note that a problem definition does not yet define a JLP task. Final interpretation
is possibly only when the problem definition and simulated data are linked in a call to jlp()
function. At the problem definition stage it is not yet known which variables are z-variables,
which are x-variables and which are factory variables (see Lappi 1992).
NNote that �<� means less or equal, and �>� means greater or equal. The equality is always
part of linear programming constraints.
The logic of jlp() function is the same as in the old JLP software. There is one difference
which makes the life a little easier with J. In J the problem definition can use c-variables which
are defined in the stand data. These are used similarly as if they would become from the xdata. It does not make any sense to have on a problem row only c-variables, but there can be
constraints like
vol#1-vol#0>0
where vol#0 is the initial volume, i.e. a c-variable, and vol#1 is the volume during first period.
In old JLP these initial values had to be put into the x-data.
NNote also that problem definition rows are not in one-to-one relation to the constraint rows in
the final lp problem. A problem definition row may belong to several domains, thus several lpconstraint rows may be generated from one problem definition row. The problem obtained by
taking multiple domains in domain definition rows into account is called �expanded problem�.
Domain definitions describe logical or arithmetic statements indicating for what management
units the following rows apply. Problem will generate problem definition object, which is
described below.
Starting from J3.0 it is also possible to specify the period of the row for each row containing xvariables. The period is given between two �#� signs at the beginning of the row, e.g.
#2# income.2-income.1 >0
If the row contains x variables from several periods, the period of the row is the last period of
the x variables. If the period is given for some rows containing x variables, it must be given for
all except for the objective row. The period of the objective is assumed to be the last period as
having any other period for the objective would not make any sense. If wrong period is given
for a row, J computes the correct solution but not as efficiently as with correct periods.
If periods are given for rows, J is utilizing the tree structure of schedules in the optimization.
This leads to smaller amount of additions and multiplications as the computation of the valu
of a branch of the tree can for each node utilize the value of branch before that node.
Unfortunately this was not more efficient e.g. in test problems with five periods.
NNote 1: Only maximization is allowed in problems including factories. To change a
minimization problem to a maximization problem, multiply the objective function by -1.
*** We may later add the possibility to define also minimization problems.
NNote 2: If optimization problem includes factories (see chapter 11.2 Optimization problem
including factories), there have to be variables in the objective function or at least in one
constraint row. Example of problem definition including factories can be found in chapter 11.12
JLP Examples.
NNote 3: An ordinary linear programming problem contains only z-variables.
NNote 4: It is not necessary to define problem() function if the problem includes only zvariables. In jlp() function you can use zmatrix-> option instead of problem-> option.
For more information see chapter 11.8 Solving a large problem with z-variables: jlp( ).
NNote 5: If the problem contains harvest/area constraints for several domains, it saves memory,
if the constraints are written in form
harvest < area_of_domain*constant
instead of
(1/area_of_domain)*harvest < constant.
The latter formulation takes the number of domains times more memory than the former
formulation.
\end{verbatim}
\color{Black}
\subsection{Solving large problems without schedule data.}
\label{jlp}
When solving problems including only a large number of z-variables, it is possible to feed the
coefficients as a matrix with \textcolor{blue}{zmatrix->} option. Unit and schedule data (c- and x-variables)
are not allowed when \textcolor{blue}{zmatrix->} is used.
=\textcolor{VioletRed}{jlp}(\textcolor{blue}{zmatrix->},\textcolor{blue}{max->}|min->[,\textcolor{blue}{rhs->}][,\textcolor{blue}{rhs2->}][,\textcolor{blue}{tole->}]
[,\textcolor{blue}{print->}][,\textcolor{blue}{maxiter->}][,\textcolor{blue}{test->}][,\textcolor{blue}{debug->}])
Output:
Function \textcolor{VioletRed}{jlp}() generates output row vectors output\%zvalues, output\%redcost and
output column vectors output\%rows, output\%shprice output is the name of the output.
Options:
zmatrix Matrix containing coefficients of z-variables for each constraint row.
max Vector containing coefficients of z-variables for the objective row of a
maximization problem. Either \textcolor{blue}{max->} or \textcolor{blue}{min->} option has to be defined but not
both.
min Vector containing coefficients of z-variables for the objective row of a
minimization problem. Either \textcolor{blue}{max->} or \textcolor{blue}{min->} option has to be defined but not
both.
rhs Vector containing lower bound for each constraint row. Value j\_inf is used to
indicate the absence of the lower bound in a row. Either or both of the bound
options (\textcolor{blue}{rhs->}, \textcolor{blue}{rhs2->}) has to be defined.
rhs2 Vector containing upper bound for each constraint row. Value j\_inf is used to
indicate the absence of the upper bound in a row. Either or both of the bound
options (\textcolor{blue}{rhs->}, \textcolor{blue}{rhs2->}) has to be defined.
Other options described above in chapter Solving a problem: \textcolor{VioletRed}{jlp}( ).
NNote. When \textcolor{blue}{zmatrix->} option is used, the solution is not automatically printed. Use jlp
solution objects to access solution. For more information see chapter 11.10 Objects for the JL
enEdnote


There are two versions of \textcolor{VioletRed}{jlp}() function call: one with problems-> option for problems
defined by \textcolor{VioletRed}{problem}() function and the other with \textcolor{blue}{zmatrix->} option for large ordinary
linear programming problems with z-variable coefficients defined by matrix.
A lp problem defined by \textcolor{VioletRed}{problem}() function can be solved using \textcolor{VioletRed}{jlp}() function:
[=]jlp(\textcolor{blue}{problem->}[,\textcolor{blue}{data->}][,\textcolor{blue}{z->}][,\textcolor{blue}{trans->}][,\textcolor{blue}{subtrans->}]
[,\textcolor{blue}{tole->}][,\textcolor{blue}{subfilter->}][,\textcolor{blue}{subreject->}][,\textcolor{blue}{class->}]
[,\textcolor{blue}{area->}][,\textcolor{blue}{notareavars->}][,\textcolor{blue}{print->}][,\textcolor{blue}{report->}]
[,\textcolor{blue}{maxiter->}][,\textcolor{blue}{refac->}][,\textcolor{blue}{slow->}][,\textcolor{blue}{warm->}][,\textcolor{blue}{finterval->}]
[,\textcolor{blue}{fastrounds->}][fastpercent->][,swap->][,\textcolor{blue}{test->}][,\textcolor{blue}{debug->}]
[,\textcolor{blue}{memory->}])
Output:
Necessary for factory problems, otherwise optional. If output is given then function generates
several matrices and lists associated with the solution (e.g. the values of the constraint rows,
the shadow prices of the rows, the values of the z-variables, the reduced costs of z-variables,
the sums of all x-variables of the data in all domains and their shadow prices, lists telling how
problem variables are interpreted. See Objects for the JLP solution for more detailed
description.
Options:
problem problem definition generated by \textcolor{VioletRed}{problem}() function
data data set describing the stand (management unit) data or the schedules data. The
unit data set must be linked to schedule data either using sub-options in the
\textcolor{VioletRed}{data}() function or using \textcolor{VioletRed}{linkdata}() function. Following the JLP terminology,
the unit data is called cdata, and the schedule data is called xdata. The \textcolor{VioletRed}{jlp}()
function tries if it can find a subdata for the data set given. If it finds, it will assume
that the data set is the unitdata. If subdata is not found, it tries to find the upper
level data. If it finds it, then it assumes that the data set given is the schedules
data. If \textcolor{blue}{data->} is not given, then the problem describes an ordinary lp-problem,
and all variables are z-variables. If \textcolor{blue}{data->} option is given but no variable found
in problem is in the schedules data set, then an error occurs.
z If the \textcolor{blue}{data->} option is given then the default is that there are no z-variables in
the problem. The existence of z-variables must be indicated with \textcolor{blue}{z->} option
(later the user can specify exactly what are the z variables, but now it is not
possible). The reason for having this option is that the most jlp-problems do not
have z variables, and variables which \textbf{J} interprets as z-variables are just
accidentally missing from the data sets.
trans transformation set which is executed for each unit.
subtrans transformation set which is executed for each schedule.
NNote: the subtrans transformations can utilize the variables in the unit data and
the output variables of \textcolor{blue}{trans->} transformations.
NNote: transformations already associated with cdata and xdata are taken
automatically into account and they are executed before transformations defined
in \textcolor{blue}{trans->} or \textcolor{blue}{subtrans->} options.
*** Later we may add the possibility to have several data sets (note that several files can be
read into one data object in the \textcolor{VioletRed}{data}()function)
tole the default tolerances are multiplied with the value of the \textcolor{blue}{tole->} option (default
is thus one). Smaller tolerances mean that smaller numerical differences are
interpreted as significant. If it is suspected that \textcolor{VioletRed}{jlp}() has not found the
optimum, use e.g. \textcolor{blue}{tole->}0.1 ,\textcolor{blue}{tole->}0.01 or \textcolor{blue}{tole->}10.
subfilter logical or arithmetic statement (nonzero value indicating True) describing which
schedules will be included in the optimization. If all schedules are rejected, an
error occurs. Examples: \textcolor{blue}{filter->}(.not.clearcut) , filter-
>(ncuts.ge.5), \textcolor{blue}{filter->}harvest (which is equivalent to: filter-
>(harvest.ne.0)). If the subfilter statement cannot be defined nicely using
one statement, the procedure can be put into a transformation set which can be
then executed using value() function.
subreject logical or arithmetic statement (nonzero value indicating True) describing which
schedules will not be included in the optimization. If \textcolor{blue}{subfilter->} is given then
test applied only for such schedules which pass the subfilter test. If the subreject
statement cannot be defined nicely using one statement, the procedure can be
put into a transformation set which can be then executed using value()
function.
Kommentoinut [LR(46]: Repon puolelle
Kommentoinut [LR(47]: Repon puolelle[Natural resources and bioeconomy studies XX/20XX]
95
class \textcolor{blue}{class->}(cvar, cval) Only those treatment units where the variable cvar gets
value cval are accepted into the optimization. The units within the same class
must be consecutive.
area gives the variable in cdata which tells for each stand the area of the stand. It is
then assumed that all variables of cdata or xdata used in the problem rows are
expressed as per are values. In optimization the proper values of variables are
obtained by multiplying area and per area values. Variables of cdata used in
domain definitions are used as they are, i.e. without multiplying with area.
Variables which are not treated as per area values are given with the
\textcolor{blue}{notareavars->} option.
notareavars
If \textcolor{blue}{area->} option is given then this option gives variables which will not be
multiplied with area.
print of output printed, 1 => summary of optimization steps, 2=> also the problem
rows are printed, 3=> also the values of x-variables are printed, 9= the pivot
steps and the point of code where pivoting is done and the value of objective
function are written to fort.16 file (or similar file depending on the operating
system). The value 9 can be used in the case where jlp seems to be stuck. From
fort.16 file one can then infer at what point debugging should be put on. Some
cycling situations are now detected, so it should be rather unlikely that jlp is stuck.
report the standard written output is written into the file given in \textcolor{blue}{report->} option
(.e.g. \textcolor{blue}{report->}'result.txt'). The file remain open and can be written by
several jlp-functions or by additional \textcolor{VioletRed}{write}() functions. Use \textcolor{VioletRed}{close}() function
to close it explicitly if you want to open it with other program.
maxiter maximum number of rounds through all units (default 10000).
refac after refac pivot steps the basis matrix is refactorized. The default value is 1000.
New option since J3.0. Actually refactorization was present in the first version of
\textbf{J} but it had to be dropped because the refactorization corrupted some times the
factors of the basis matrix. The reason was found and corrected for J3.0. The
reason for misbehavior of the refactorization algorithm of Fletcher was such that
it never caused problems in ordinary linear programming problems for which
Fletcher designed his algorithms.
slow if improvement of the solution in one round through all units is smaller than value
given in \textcolor{blue}{slow->} option, then \textbf{J} terminates under condition �slow improvement�.
New option since J3.0. Earlier slow improvement is computed from the tolerances
of the problem, and if the \textcolor{blue}{slow->} option is not present these tolerances are still
used. If \textcolor{blue}{slow->} option gives negative value, then the absolute value of the
option indicates per cent change. Note that in large problems the solution is
often very long time quite close the actual optimum, and hence the optimization
time can be decreased with rather low loss in accuracy using the \textcolor{blue}{slow->} option.
If \textcolor{blue}{slow->} option is give value zero it is equivalent to omitting the option and
hence the slow improvement is determined from the tolerances.
warm If option is present in ordinary problems with x-variables then the key schedules
of previous solution are used as the starting values of the key schedules. In factory[Natural resources and bioeconomy studies XX/20XX]
96
problems also the previous key factories are used as starting values. If there is no
previous solution available or the dimension of the key schedules vector (number
of treatment units) or the dimensions of the key factories matrix (number of
treatment units and number of factories) do not agree with the current problem,
warm option is ignored. Thus it is usually safe to use the option always, the only
exception is that the factories and number of units do agree with the previous
problem even if factories or schedule data are changes. The warm start may
reduce solution time perhaps 40-80\%.
finterval In factory problems the transportations to new factories are checked if the round
number is divisible with finterval. The default value is 3. When there have not
been improvements during the last round, the value is changed into 1.
fastrounds
The shadow prices are used to select an active set of schedules which are
considered as entering schedules. fastrounds gives the number of rounds using
the same active set. The default is 10. When there have not been improvements
during the last round, all schedules are used as the active set.
fastpercent
A schedule belongs to the active set if its shadow price is at least fastpercent \%
of the shadow price of the best schedule. The default is 90.
swap If option is present, the schedules data matrix is written to a direct access binary
file. This option may help if virtual memory overflow occurs. The data needs to
be used from the file until all inquiry function calls are computed. Thereafter the
data can be loaded again into memory using function unswap(). If this does not
happen ordinary functions using this data object work a little slower. If a new
problem is solved with the swap-> option, there is no need to unswap() before
that. In an Intel Fortran application the swap-> option is given without
arguments. In a Gfortran application the option must be given in form swap->4.
If there is shortage of virtual memory, read note 5 for \textcolor{VioletRed}{problem}() function before
starting to use swap->.
test If option is present then \textcolor{VioletRed}{jlp}() is checking the consistency of the intermediate
results after each pivot step of the algorithm. Takes time but helps in debugging.
debug determines after which pivot steps \textcolor{VioletRed}{jlp}() starts and stops to print debugging
information to fort.16 file. If no value given, the debugging starts immediately
(produces much output, so it may be good to use step number which is close to
the step where problems started (print variable Pivots at the error return).
\textcolor{blue}{debug->}(ip1,ip2,ip3) indicates that debugging is put on at pivot step ip1,
off at pivot ip2 and the again on at pivot ip3.
memory gives the amount of memory in millions of real numbers that can be used to store
data needed in solving the problem. In factory problems also the xdata stored in
a direct access file are loaded into memory as much as possible. It is not possible
to figure out how large number memory option can give, so it must be
determined with experimentation. Using disk-> option in \textcolor{VioletRed}{data}() function and
\textcolor{blue}{memory->} option in \textcolor{VioletRed}{jlp}() function makes it in principle possible to solve
arbitrary large problems. In practise the ability of double precision numbers
cannot store accurately the needed quantities in very huge problems.
Kommentoinut [LR(48]: Vain gfortran k��nn�s?
Kommentoinut [LR(49]: Lis�tty. Oliko t�m� mukana
julkaistavassa versiossa?[Natural resources and bioeconomy studies XX/20XX]
97
Function \textcolor{VioletRed}{jlp}() is generating output (amount is dependent on the \textcolor{blue}{print->} option) plus a
\textbf{J}LP-solution stored in special data structures which can be accessed with special \textbf{J} functions
described below and if output is given then several output objects are created (see Objects for
the JLP solution).
NNote 1: a feasible solution (without an objective) can be found by minimizing a z-variable
(remember \textcolor{blue}{z->} option), or by maximizing a unit variable (which is constant for all schedules in
a unit).
NNote 2: If virtual memory overflow occurs, see first Note 5 for \textcolor{VioletRed}{problem}() function and then
\subsection{weigts() weights of schdules}
\label{weights}
TO BE RAPORTED LATER
\subsection{\textcolor{VioletRed}{partweight}() weights of split schedules}\index{partweight()}
\label{partweights}
TO BE RAPORTED LATER
\section{Plotting figures}
\label{Plotting}
The graphiscs of the current version of \textbf{J} is produced Gnuplot. \textbf{J} offers nowan alternative interface
to Gnuplot, and it is quite easy to add more plotinng routines later.
\subsection{Figures}
\label{figu}
Figures are made using Gnuplot. \textbf{J} transmits information into Gnuplot using text files.
\textbf{J} generates the files using .jfig extension. Necessary files are generated  by deleting old
files without asking permission. If \textcolor{teal}{fig} is the output of a figure function, then
\textbf{J} creates always, except in 3D, file with name \textcolor{teal}{fig}.jfig and possibly other files with
.jfi0, jfi1  etc extensions.
All figure functions use the following options.
\vspace{0.3cm}
\hline
\vspace{0.3cm}
\noindent Output  \tabto{3cm}  1 \tabto{5cm}   FIGURE  \tabto{7cm}
\begin{changemargin}{3cm}{0cm}
\noindent  The FIGURE object created or updated
@@figure
\end {changemargin}
\hline
\vspace{0.2cm}
\begin{note}
is possible to show a figure using \textcolor{VioletRed}{show}() function. It can have either the \textcolor{teal}{fig} object as the
argument or the file name of \textcolor{teal}{fig}.jfig -file. Thus it is possible
to edit the file using Gnuplot capabilites.
\end{note}
\begin{note}
It is possible to change the terminal type used by Gnuplot by giving
the name of the terminal to the predefine CHAR variable \textcolor{teal}{Terminal}.
The default is \newline \textcolor{teal}{Terminal}='qt'.
\end{note}
\begin{note}
The default writte by Gnuplot does not look nice and is not implemented.
The user can write own legends using \textcolor{blue}{label->} option in \textcolor{VioletRed}{drawline}().
\end{note}
\begin{note}
Easiest way to delete an nonmatrix object \textcolor{teal}{fi} is \textcolor{teal}{fi}=0, which makes it possible
to use \textcolor{blue}{append->} also for an \textcolor{teal}{fi} wihtout the need to construct if then-
structures.
\end{note}
\subsubsection{Figure object}
\label{figureo}
Graphic functions produce FIGURE objects. Each FIGURE object can consist of
several subfigures. Each FIGURE object stores information of x- and y axes, the
range of all x- and y-values, and for each sub-figure information of the ranges
of x and y in the subfigure plus the subfigure type and the needed data values.
Currently, when Gnuplot is used for graphics, most data values are stored
in text files which Gnuplot reads. Function \textcolor{VioletRed}{plot3d}() is plotting 3-d figures without making a FIGURE object.
See Plotting figures for more information.
\subsection{Scatterplot: \textcolor{VioletRed}{plotyx}()}\index{plotyx()}
\label{plotyx}
\textcolor{VioletRed}{plotyx}() makes scatterplot.
\vspace{0.3cm}
\hline
\vspace{0.3cm}
\noindent Output  \tabto{3cm}  1 \tabto{5cm}   FIGURE  \tabto{7cm}
\begin{changemargin}{3cm}{0cm}
\noindent  The FIGURE object created or updated.
\end{changemargin}
\vspace{0.3cm}
\hline
\vspace{0.3cm}
\noindent Args  \tabto{3cm}  1 | 2  \tabto{5cm}   REAL  \tabto{7cm}
\begin{changemargin}{3cm}{0cm}
\noindent  y and x-variable, if \textcolor{blue}{func->} is not present.
In case y-variable is given with \textcolor{blue}{func->}only,  x-variable is given as argument.
\end{changemargin}
\vspace{0.3cm}
\hline
\vspace{0.3cm}
\noindent \textcolor{blue}{data}  \tabto{3cm}  N | 1  \tabto{5cm}   DATA  \tabto{7cm}
\begin{changemargin}{3cm}{0cm}
\noindent  Data object used, default the last data object created or the dta given
with data=\textcolor{VioletRed}{list}().
@@figu
\end{changemargin}
\vspace{0.3cm}
\hline
\vspace{0.3cm}
\noindent \textcolor{blue}{mark}  \tabto{3cm}  N | 1  \tabto{5cm}   REAL | CHAR  \tabto{7cm}
\begin{changemargin}{3cm}{0cm}
\noindent  The mark used in the plot. Numeric values refer to
mark types of Gnuplot. The mark can be given also as CHAR varible or constant.
\end{changemargin}
\vspace{0.3cm}
\hline
\vspace{0.3cm}
\noindent \textcolor{blue}{func} \tabto{3cm}  N | 1  \tabto{5cm}   Code  \tabto{7cm}
\begin{changemargin}{3cm}{0cm}
\noindent   Code option telling how the y-variable is computed.
\end {changemargin}
\hline
\vspace{0.2cm}
\begin{example}[plotyxex]plotyx()\\
\label{plotyxex}
xmat=\textcolor{VioletRed}{matrix}(\textcolor{blue}{do->}(0,10,0.001))\\
tr=\textcolor{VioletRed}{trans}()\\
y=2+3*x+0.4*x*x+4*rann()\\
/\\
da=\textcolor{VioletRed}{newdata}(xmat,\textcolor{blue}{read->}x,\textcolor{blue}{maketrans->}tr,\textcolor{blue}{extra->}(Regf,Resid))\\
fi=\textcolor{VioletRed}{plotyx}(y,x,\textcolor{blue}{continue->}fcont)\\
fi=\textcolor{VioletRed}{plotyx}(x,\textcolor{blue}{func->}tr(y),\textcolor{blue}{mark->}3,\textcolor{blue}{color->}Orange,\textcolor{blue}{continue->}fcont)\\
reg=\textcolor{VioletRed}{regr}(y,x)\\
fi=\textcolor{VioletRed}{plotyx}(y,x,\textcolor{blue}{show->}0)\\
fi=\textcolor{VioletRed}{plotyx}(Regf,x,\textcolor{blue}{append->},\textcolor{blue}{continue->}fcont)\\
fir=\textcolor{VioletRed}{plotyx}(Resid,x,\textcolor{blue}{continue->}fcont)
\end{example}
\begin{note}
With data with integer values, the default ranges of Gnuplot may be hide point at
borderlines.
\end{note}
\begin{note}
\textcolor{teal}{fi}=\textcolor{VioletRed}{plotyx}() produces or updates file \textcolor{teal}{fi}.jfig] which contains
Gnuplot commands and file \textcolor{teal}{fi}.jfi0 containg data.
\end{note}
\subsection{Draw a function: \textcolor{VioletRed}{draw}()}\index{draw()}
\label{draw}
\textcolor{VioletRed}{draw}() draws a function.
\vspace{0.3cm}
\hline
\vspace{0.3cm}
\noindent Output  \tabto{3cm}  1 \tabto{5cm}   FIGURE  \tabto{7cm}
\begin{changemargin}{3cm}{0cm}
\noindent  The FIGURE object created or updated.
\end{changemargin}
\vspace{0.3cm}
\hline
\vspace{0.3cm}
\noindent \textcolor{blue}{func} \tabto{3cm}  N | 1  \tabto{5cm}   Code  \tabto{7cm}
\begin{changemargin}{3cm}{0cm}
\noindent   Code option telling how the y-variable is computed.
@@draw
\end{changemargin}
\vspace{0.3cm}
\hline
\vspace{0.3cm}
\noindent \textcolor{blue}{mark}  \tabto{3cm}  N | 1  \tabto{5cm}   REAL | CHAR  \tabto{7cm}
\begin{changemargin}{3cm}{0cm}
\noindent  The mark used in the plot.
Numeric values refer to.
mark types of Gnuplot. The mark can be given also as CHAR varible or constant.
\end{changemargin}
\vspace{0.3cm}
\hline
\vspace{0.3cm}
\noindent \textcolor{blue}{width}  \tabto{3cm}  0 | 1  \tabto{5cm}   REAL  \tabto{7cm}
\begin{changemargin}{3cm}{0cm}
\noindent  the width of the line
\end {changemargin}
\hline
\vspace{0.2cm}
\begin{example}[drawex]Example of \textcolor{VioletRed}{draw}()\\
\label{drawex}
fi=\textcolor{VioletRed}{draw}(\textcolor{blue}{func->}\textcolor{VioletRed}{sin}(x),\textcolor{blue}{x->}x,\textcolor{blue}{xrange->}(0,2*Pi),\textcolor{blue}{color->}Blue,\textcolor{blue}{continue->}fcont)\\
fi=\textcolor{VioletRed}{draw}(\textcolor{blue}{func->}\textcolor{VioletRed}{cos}(x),\textcolor{blue}{x->}x,\textcolor{blue}{xrange->}(0,2*Pi),\textcolor{blue}{color->}Red,\textcolor{blue}{append->},\textcolor{blue}{continue->}fcont)\\
\textcolor{VioletRed}{if}(\textcolor{VioletRed}{type}(figyx).ne.FIGURE)plotyxex\\
\textcolor{VioletRed}{show}(figyx,cont->fcont)\\
reg0=\textcolor{VioletRed}{regr}(y,x)\\
\textcolor{VioletRed}{stat}(\textcolor{blue}{data->}datyx,\textcolor{blue}{min->},\textcolor{blue}{max->})\\
figyx=\textcolor{VioletRed}{draw}(\textcolor{blue}{func->}reg0(),\textcolor{blue}{x->}x,\textcolor{blue}{xrange->},\textcolor{blue}{color->}Violet,\textcolor{blue}{append->},\textcolor{blue}{continue->}fcont)\\
tr=\textcolor{VioletRed}{trans}()\\
x2=x*x\\
fu=reg2()\\
/\\
reg2=\textcolor{VioletRed}{regr}(y,x,x2,\textcolor{blue}{data->}datyx,\textcolor{blue}{trans->}tr)\\
figyx=\textcolor{VioletRed}{draw}(\textcolor{blue}{func->}tr(fu),\textcolor{blue}{xrange->},\textcolor{blue}{color->}Orange,\textcolor{blue}{append->},\textcolor{blue}{continue->}fcont)\\
Continue=1 \,\textcolor{green}{!Errors}\\
fi=\textcolor{VioletRed}{draw}(\textcolor{blue}{func->}\textcolor{VioletRed}{sin}(x),\textcolor{blue}{x->}x)\\
fi=\textcolor{VioletRed}{draw}(\textcolor{blue}{xrange->}(1,100),\textcolor{blue}{func->}Sin(x),\textcolor{blue}{x->}x)\\
Continue=0
\end{example}
\begin{note}
\textcolor{teal}{fi}=\textcolor{VioletRed}{draw}() produces or updates file \textcolor{teal}{fi}.jfig] which contains
Gnuplot commands and file \textcolor{teal}{fi}.jfi0 containg data.
\end{note}
\subsection{Draw values in a matrix generated with \textcolor{VioletRed}{classify}(): \textc}\index{classify()drawclass()}
\label{drawclass}
\textcolor{VioletRed}{drawclass}() can plot class means and/or lines connecting class means, with
or without standard errors of class means, within class standard deviations,
within class variances, frequency histograms, which can be scaled so that
density funtions can be drawn in the same figure.
\vspace{0.3cm}
\hline
\vspace{0.3cm}
\noindent Output  \tabto{3cm} 1  \tabto{5cm}   FIGURE  \tabto{7cm}
\begin{changemargin}{3cm}{0cm}
\noindent  FIGURE object updated or generated.
\end{changemargin}
\vspace{0.3cm}
\hline
\vspace{0.3cm}
\noindent Arg  \tabto{3cm}  1 \tabto{5cm}   MATRIX  \tabto{7cm}
\begin{changemargin}{3cm}{0cm}
\noindent  A MATRIX generated with \textcolor{VioletRed}{classify}().
\end{changemargin}
\vspace{0.3cm}
\hline
\vspace{0.3cm}
\noindent \textcolor{blue}{se}  \tabto{3cm}  N | 0  \tabto{5cm}    \tabto{7cm}
\begin{changemargin}{3cm}{0cm}
\noindent Presence of option tells to include that error bars showing standard errors
of class means computed as \textcolor{VioletRed}{sqrt}(sample\_within-class\_variance)/number\_of\_obs)
\end{changemargin}
\vspace{0.3cm}
\hline
\vspace{0.3cm}
\noindent \textcolor{blue}{sd}  \tabto{3cm}  N | 0  \tabto{5cm}    \tabto{7cm}
\begin{changemargin}{3cm}{0cm}
\noindent  Within-calss standard deviations are drawn.
\end{changemargin}
\vspace{0.3cm}
\hline
\vspace{0.3cm}
\noindent \textcolor{blue}{var}  \tabto{3cm}  N | 0  \tabto{5cm}     \tabto{7cm}
\begin{changemargin}{3cm}{0cm}
\noindent Within-class sample variances are drawn.
\end{changemargin}
\vspace{0.3cm}
\hline
\vspace{0.3cm}
\noindent histo  \tabto{3cm}   N | 0  \tabto{5cm}     \tabto{7cm}
\begin{changemargin}{3cm}{0cm}
\noindent Within-class sample variances are drawn.abto{5cm}
\end{changemargin}
\vspace{0.3cm}
\hline
\vspace{0.3cm}
\noindent \textcolor{blue}{freq}  \tabto{3cm}  N | 0  \tabto{5cm}     \tabto{7cm}
\begin{changemargin}{3cm}{0cm}
\noindent Cumulative  frequences are drawn.
\end{changemargin}
\vspace{0.3cm}
\hline
\vspace{0.3cm}
\noindent \textcolor{blue}{area}  \tabto{3cm}  N |0  \tabto{5cm}    \tabto{7cm}
\begin{changemargin}{3cm}{0cm}
\noindent the histogram is scaled so that that it can be overlayed to density function
\end {changemargin}
\hline
\vspace{0.2cm}
\begin{example}[drawclassex]Examples of \textcolor{VioletRed}{drawclass}()\\
\label{drawclassex}
X=\textcolor{VioletRed}{matrix}(\textcolor{blue}{do->}(1,100,0.1))\\
e=\textcolor{VioletRed}{matrix}(\textcolor{VioletRed}{nrows}(X))\\
e=\textcolor{VioletRed}{rann}()\\
X2=0.01*x*.x \textcolor{green}{!elementwise\,product}\\
Y=2*x+0.01*X2+(1+0.3*X)*.e \,\textcolor{green}{!nonequal\,error\,variance,quadratic\,function}\\
dat=\textcolor{VioletRed}{newdata}(x,y,x2,\textcolor{blue}{read->}(x,y,x2),\textcolor{blue}{extra->}(Regf,Resid))\\
\textcolor{VioletRed}{stat}(\textcolor{blue}{min->},\textcolor{blue}{max->})\\
reg=\textcolor{VioletRed}{regr}(y,x) \textcolor{green}{!\,Regf\,and\,resid\,are\,put\,into\,the\,data}\\
fi=\textcolor{VioletRed}{plotyx}(y,x,\textcolor{blue}{continue->}fcont)\\
fi=\textcolor{VioletRed}{drawline}(x\%min,x\%max,reg(x\%min),reg(x\%max),\textcolor{blue}{width->}3,\textcolor{blue}{color->}Cyan,\textcolor{blue}{append->},\textcolor{blue}{continue->}fcont)\\
cl=\textcolor{VioletRed}{classify}(Resid,\textcolor{blue}{x->}x,\textcolor{blue}{xrange->},\textcolor{blue}{classes->}5)\\
fi=\textcolor{VioletRed}{drawclass}(cl,\textcolor{blue}{color->}Blue,\textcolor{blue}{continue->}fcont)\\
fi=\textcolor{VioletRed}{drawclass}(cl,\textcolor{blue}{se->},\textcolor{blue}{continue->}fcont)\\
fi=\textcolor{VioletRed}{drawclass}(cl,\textcolor{blue}{sd->},\textcolor{blue}{continue->}fcont)\\
fi=\textcolor{VioletRed}{drawclass}(cl,\textcolor{blue}{var->},\textcolor{blue}{continue->}fcont)\\
fi=\textcolor{VioletRed}{drawclass}(cl,histo->,\textcolor{blue}{area->},\textcolor{blue}{continue->}fcont)\\
fi=\textcolor{VioletRed}{draw}(\textcolor{blue}{func->}\textcolor{VioletRed}{pdf}(0,\textcolor{VioletRed}{rmse}(reg)),\textcolor{blue}{x->}x,\textcolor{blue}{xrange->},\textcolor{blue}{append->},\textcolor{blue}{continue->}fcont) \textcolor{green}{!\,xrange\,comes\,from\,\textcolor{VioletRed}{stat}()}
\end{example}
\begin{note}
In previous versions of \textbf{J} if \textcolor{blue}{se->} and \textcolor{blue}{sd->} were both present, the error
bot bars were plotted. This possibility will be included later.
\end{note}
\subsection{Draw a polygon through points: \textcolor{VioletRed}{drawline}()}\index{drawline()}
\label{drawline}
\textcolor{VioletRed}{drawline}() draws a function through points.
\vspace{0.3cm}
\hline
\vspace{0.3cm}
\noindent Output  \tabto{3cm}  1 \tabto{5cm}   FIGURE  \tabto{7cm}
\begin{changemargin}{3cm}{0cm}
\noindent  The FIGURE object created or updated.
\end{changemargin}
\vspace{0.3cm}
\hline
\vspace{0.3cm}
\noindent Args \tabto{3cm} 1-  \tabto{5cm}  REAL | MATRIX  \tabto{7cm}
\begin{changemargin}{3cm}{0cm}
\noindent  The points which are connected:
\begin{itemize}
\item[\textbf{J}\.] x1,...,xn,y1,...,yn The x-coordinates and y-coordinates,
\$n \geq 1\$
\item[\textbf{J}\.]  If there is only one argument which is a
matrix object having two rows, then the first row is assumed to give the x values
and the second row the y values.
\item[\textbf{J}\.]  If there are two matrix (vector) arguments, then
the first matrix gives the x-values and the second matrix gives the y-values.
It does not matter if arguments are row or column vectors.
\end{itemize}
@@figure
\end{changemargin}
\vspace{0.3cm}
\hline
\vspace{0.3cm}
\noindent \textcolor{blue}{label}  \tabto{3cm}  N | 1  \tabto{5cm}   CHAR  \tabto{7cm}
\begin{changemargin}{3cm}{0cm}
\noindent  Label written to the end of line. If arguments define only one point,
then with \textcolor{blue}{label->} option one can write text to any point.
\end{changemargin}
\vspace{0.3cm}
\hline
\vspace{0.3cm}
\noindent \textcolor{blue}{mark}  \tabto{3cm}  N | 1  \tabto{5cm}   REAL | CHAR  \tabto{7cm}
\begin{changemargin}{3cm}{0cm}
\noindent  The mark used in the plot.
\end{changemargin}
\vspace{0.3cm}
\hline
\vspace{0.3cm}
\noindent \textcolor{blue}{break}  \tabto{3cm}  N | 0  \tabto{5cm}    \tabto{7cm}
\begin{changemargin}{3cm}{0cm}
\noindent  The line is broken when a x-value is smaller than the previous one.
\end{changemargin}
\vspace{0.3cm}
\hline
\vspace{0.3cm}
\noindent \textcolor{blue}{set}  \tabto{3cm}  N|1  \tabto{5cm}   REAL<6  \tabto{7cm}
\begin{changemargin}{3cm}{0cm}
\noindent  Set to which lines are put. If the option is not present,
then a separate Gnuplot plot command with possible color and width information
is generated for each \textcolor{VioletRed}{drawline}() and data points are stored
in file \textcolor{teal}{fi}.jfi0, i.e. the same file used by \textcolor{VioletRed}{plotyx}().
If set is given e.g as \textcolor{blue}{set->}3, then it is possible to plot a large number of lines
with the same width and color. The data points are stored into file \textcolor{teal}{fi}.jfi3. This is
useful e.g. when drawing figures showing transportation of timber to factories
for huge number of sample plots.
Numeric values refer to Gnuplot mar types.
The mark can be given also as CHAR varible or constant.
\end{changemargin}
\vspace{0.3cm}
\hline
\vspace{0.3cm}
\noindent \textcolor{blue}{width}  \tabto{3cm}  0 | 1  \tabto{5cm}   REAL  \tabto{7cm}
\begin{changemargin}{3cm}{0cm}
\noindent  the width of the line. Default: \textcolor{blue}{width->}1
\end{changemargin}
\vspace{0.3cm}
\hline
\vspace{0.3cm}
\noindent \textcolor{blue}{label} \tabto{3cm}  N |1  \tabto{5cm}  CHAR  \tabto{7cm}
\begin{changemargin}{3cm}{0cm}
\noindent  Text plotted to the end of line.
\end {changemargin}
\hline
\vspace{0.2cm}
\begin{example}[drawlineex]Example of \textcolor{VioletRed}{drawline}()\\
\label{drawlineex}
fi=\textcolor{VioletRed}{draw}(\textcolor{blue}{func->}\textcolor{VioletRed}{sin}(x),\textcolor{blue}{xrange->}(0,2*Pi),\textcolor{blue}{color->}Blue,\textcolor{blue}{continue->}fcont)\\
fi=\textcolor{VioletRed}{drawline}(Pi,\textcolor{VioletRed}{sin}(Pi)+0.1,\textcolor{blue}{label->}'\textcolor{VioletRed}{sin}()',\textcolor{blue}{append->},\textcolor{blue}{continue->}fcont)\\
xval=\textcolor{VioletRed}{matrix}(\textcolor{blue}{do->}(1,100))\\
mat=\textcolor{VioletRed}{matrix}(\textcolor{blue}{values->}(xval,xval+1,xval,xval+2,xval,xval+3))\\
fi=\textcolor{VioletRed}{drawline}(mat,\textcolor{blue}{color->}Red)\\
fi=\textcolor{VioletRed}{drawline}(mat,break,\textcolor{blue}{color->}Orange,\textcolor{blue}{break->},\textcolor{blue}{continue->}fcont)\\
x=\textcolor{VioletRed}{matrix}(\textcolor{blue}{do->}(0,100,1)\\
e=\textcolor{VioletRed}{matrix}(101)\\
e=\textcolor{VioletRed}{rann}()\\
y=2*x+0.4+e\\
da=\textcolor{VioletRed}{newdata}(x,y,\textcolor{blue}{read->}(x,y))\\
reg=\textcolor{VioletRed}{regr}(y,x)
\\
\textcolor{VioletRed}{if}(\textcolor{VioletRed}{type}(figyx).ne.FIGURE)plotyxex\\
\textcolor{VioletRed}{show}(figyx)\\
reg0=\textcolor{VioletRed}{regr}(y,x)\\
\textcolor{VioletRed}{stat}(\textcolor{blue}{data->}datyx,\textcolor{blue}{min->},\textcolor{blue}{max->})\\
figyx=\textcolor{VioletRed}{draw}(\textcolor{blue}{func->}reg0(),\textcolor{blue}{x->}x,\textcolor{blue}{xrange->},\textcolor{blue}{color->}Violet,\textcolor{blue}{append->},\textcolor{blue}{continue->}fcont)\\
tr=\textcolor{VioletRed}{trans}()\\
x2=x*x\\
fu=reg2()\\
/\\
reg2=\textcolor{VioletRed}{regr}(y,x,x2,\textcolor{blue}{data->}datyx,\textcolor{blue}{trans->}tr)\\
figyx=\textcolor{VioletRed}{draw}(\textcolor{blue}{func->}tr(fu),\textcolor{blue}{xrange->},\textcolor{blue}{color->}Orange,\textcolor{blue}{append->},\textcolor{blue}{continue->}fcont)\\
Continue=1 \,\textcolor{green}{!Errors}\\
fi=\textcolor{VioletRed}{draw}(\textcolor{blue}{func->}\textcolor{VioletRed}{sin}(x),\textcolor{blue}{x->}x)\\
fi=\textcolor{VioletRed}{draw}(\textcolor{blue}{xrange->}(1,100),\textcolor{blue}{func->}Sin(x),\textcolor{blue}{x->}x)\\
Continue=0
\end{example}
\begin{note}
if a line is not visible, this may be caused by the fact that
the starting or ending point is outside the range specified by \textcolor{blue}{xrange->} or \textcolor{blue}{yrange->}.
\end{note}
\subsection{Show figure: \textcolor{VioletRed}{show}()}\index{show()}
\label{show}
An figure stored in a figure object or in Gnuplot file can be plotted. If the
argument is FIGURE, the
parameters of the figure can be changed. If the argument is the name of
Gnuplot file, the file must be edited.
\vspace{0.3cm}
\hline
\vspace{0.3cm}
\noindent Args  \tabto{3cm}  1 \tabto{5cm}   FIGURE | CHAR  \tabto{7cm}
\begin{changemargin}{3cm}{0cm}
\noindent  The figure object or the name of the file containg Gnuplot commans,
@@figure
\end {changemargin}
\hline
\vspace{0.2cm}
\begin{note}
If the argument is the file name with .jfig extension, and you edit the file, its is safe to change the name,
becase if an figure with teh same name is generated, the edited fike is autimatically
deleted. If the file refers other files, it is wise to rename also these files and change
the names in the beginning of the .jfig file.
\end{note}
\begin{note}
You may wish to use show also if you cnange the window size
\end{note}

\begin{example}[showex]Example of \textcolor{VioletRed}{show}()\\
\label{showex}
fi=\textcolor{VioletRed}{draw}(\textcolor{blue}{func->}\textcolor{VioletRed}{sqrt2}(x),\textcolor{blue}{x->}x,\textcolor{blue}{xrange->}(-50,50),\textcolor{blue}{continue->}fcont)\\
\textcolor{VioletRed}{show}(fi,\textcolor{blue}{xrange->}(-60,60),\textcolor{blue}{xlabel->}'NEWX',\textcolor{blue}{ylabel->}'NEWY')\\
\textcolor{VioletRed}{show}(fi,\textcolor{blue}{axes->}10)\\
\textcolor{VioletRed}{show}(fi,\textcolor{blue}{axes->}01)\\
\textcolor{VioletRed}{show}(fi(\textcolor{blue}{axes->}00)\\
Window='400,800'\\
\textcolor{VioletRed}{show}(fi)\\
Window='700,700'\\
fi=\textcolor{VioletRed}{drawline}(1,10,3,1,\textcolor{blue}{color->}Red,\textcolor{blue}{continue->}fcont)\\
\textcolor{VioletRed}{show}(fi,\textcolor{blue}{xrange->}(1.1,11)) \textcolor{green}{!the\,line\,is\,not\,visible}\\
dat=\textcolor{VioletRed}{data}(\textcolor{blue}{read->}(x,y),\textcolor{blue}{in->})\\
1,4\\
2,6\\
3,2\\
5,1\\
/\\
\textcolor{VioletRed}{stat}()\\
fi=\textcolor{VioletRed}{plotyx}(y,x,\textcolor{blue}{continue->}fcont) \,\textcolor{green}{!\,Gnuplot\,hides\,points\,at\,border}\\
\textcolor{VioletRed}{show}(fi,\textcolor{blue}{xrange->}(0,6),\textcolor{blue}{yrange->}(0,7))

\end{example}

\subsection{Plot 3d-figure: \textcolor{VioletRed}{plot3d}()}\index{plot3d()}
\label{plot3d}
Plot 3d-figure with indicater contours  with colours.
\vspace{0.3cm}
\hline
\vspace{0.3cm}
\noindent Output  \tabto{3cm}  1   \tabto{5cm}    \tabto{7cm}
\begin{changemargin}{3cm}{0cm}
\noindent  \textcolor{teal}{fi}=\textcolor{VioletRed}{plot3d}() generates Gnuplot file \textcolor{teal}{fi}.jfig.
No figure object is produced.
\end{changemargin}
\vspace{0.3cm}
\hline
\vspace{0.3cm}
\noindent Args \tabto{3cm}  1  \tabto{5cm}   MATRIX  \tabto{7cm}
\begin{changemargin}{3cm}{0cm}
\noindent  The argument is a matrix having 3 columns for x,y and z.
\end{changemargin}
\vspace{0.3cm}
\hline
\vspace{0.3cm}
\noindent \textcolor{blue}{sorted}  \tabto{3cm}  N | 1  \tabto{5cm}    \tabto{7cm}
\begin{changemargin}{3cm}{0cm}
\noindent \textcolor{VioletRed}{plot3d}() uses the Gnuplot function splot, which requires that the data
is sorted withe respect to the x-variable. \textcolor{blue}{sorted->} indicates that the argument matrix is sorted
either natrurally or with \textcolor{VioletRed}{sort}() function. If \textcolor{blue}{sort->} is not presented, plot3
sorts the data.
\end {changemargin}
\hline
\vspace{0.2cm}
\begin{example}[plot3dex]plot3d() example see p.  328 in Mehtatalo & Lappi 2020\\
\label{plot3dex}
mat=\textcolor{VioletRed}{matrix}(1000000,3)\\
mat2=\textcolor{VioletRed}{matrix}(1000000,3)\\
tr=\textcolor{VioletRed}{trans}() \textcolor{green}{!second\,order\,response\,surface}\\
x=0\\
x2=0\\
xy=0\\
irow=1\\
\textcolor{VioletRed}{do}(ix,1,1000)\\
y=0\\
y2=0\\
xy=0\\
\textcolor{VioletRed}{do}(iy,1,1000)\\
mat(irow,1)=x\\
mat(irow,2)=y\\
mat(irow,3)=12+8*x-7*x2+124*y+8*xy-13*y2\\
mat2(irow,1)=x\\
mat2(irow,2)=y\\
mat2(irow,3)=50+160*x-5*x2-40*y-20*xy+10*y2\\
irow=irow+1\\
y=y+0.01\\
y2=y*y\\
xy=x*y\\
enddo\\
x=x+0.01\\
x2=x*x\\
enddo\\
/\\
\textcolor{VioletRed}{call}(tr)\\
fi=\textcolor{VioletRed}{plot3d}(mat,\textcolor{blue}{sorted->})\\
fi=\textcolor{VioletRed}{plot3d}(mat2,\textcolor{blue}{sorted->})
\end{example}

\section{Splines, stem splines,  and volume functions}
\label{spline}
There are several spline functions.
\subsection{\textcolor{VioletRed}{tautspline}()}\index{tautspline()}
\label{tautspline}
\textcolor{VioletRed}{tautspline}(x1,…,xn,y1,…,yn[,\textcolor{blue}{par->}][,\textcolor{blue}{sort->}][,\textcolor{blue}{print->}])//
Output://
An interpolating cubic spline, which is more robust than an ordinary cubic spline. To prevent
oscillation (which can happen with splines) the function adds automatically additional knots
where needed.//
Arguments://
x1,…,xn the x values//
d1,…,dn the y values.//
There must be at least 3 knot point, i.e. 6 arguments.//
Options://
par gives the parameter determining the smoothness of the curve. The default is zero,
which produces ordinary cubic spline. A typical value may 2.5. Larger values mean
that the spline is more closely linear between knot points.//
sort the default is that the x’s are increasing, if not then \textcolor{blue}{sort->} option must be given//
print if \textcolor{blue}{print->} option is given, the knot points are printed (after possible sorting).
The resulting spline can be utilized using value() function.
The taut spline algorithm is published by de Boor (1978) on pages 310-314. The source code
was loaded from Netlib.
\subsection{\textcolor{VioletRed}{stemspline}(): splines for stems}\index{stemspline()}
\label{stemspline}
TO bE REPORTED LATER,  see old manual
\subsection{\textcolor{VioletRed}{stempolar}(): polar coordinates}\index{stempolar()}
\label{stempolar}
TO bE REPORTED LATER, see old manual
\subsection{\textcolor{VioletRed}{laasvol}(): svolume eqaitions of Laasasenaho}\index{laasvol()}
\label{laasvol}
TO bE REPORTED LATER,  see old manual
\subsection{\textcolor{VioletRed}{laaspoly}(): polynomila stem curves of Laasasenaho}\index{laaspoly()}
\label{laaspoly}
TO bE REPORTED LATER,  see old manual
\subsection{\textcolor{VioletRed}{integrate}(): integrate}\index{integrate()}
\label{integrate}
TO bE REPORTED LATER,  see old manual
\section{Bit functions}
\label{bit}
bit functions help to store large amound of binary variables in small space.
These functions are used in domain calcualtions
\subsection{Bitmatrix}
\label{bitmatrixo}
A BITMATRIX is an object which can store in small memory space large matrices
used to indicate logical values. A BITMATRIX object is produced by \textcolor{VioletRed}{bitmatrix}()
function or by closures() function from an existing bitmatrix. Bitmatrix values
can be read from the input stream or file or set by setvalue() function. The
values of bitmatrix elements can be accessed with value() function.
\begin{note}
Also ordinary real variable can be used to store bits. See bit functions.
\end{note}
\subsection{\textcolor{VioletRed}{setbits}(): setting bits on}\index{setbits()}
\label{setbits}
To be reported alter
\subsection{\textcolor{VioletRed}{clearbits}(): clearing bits}\index{clearbits()}
\label{clearbits}
To be reported alter
\subsection{\textcolor{VioletRed}{getbit}() : get bit value}\index{getbit()}
\label{getbit}
To be reported later, see old manual
\subsection{\textcolor{VioletRed}{getbitch}() : get bit value}\index{getbitch()}
\label{getbitch}
To be reported later, see old manual
\subsection{\textcolor{VioletRed}{bitmatrix}() : define a matrix for bits}\index{bitmatrix()}
\label{bitmatrix}
To be reported later,  see old manual
\subsection{setvalue() : set value for a bitmatrix}
\label{setvalue}
To be reported later,  see old manual
\subsection{closures() :convex closures}
\label{closures}
To be desrribed later,  see old manual
\section{Misc. functions}
\label{misc}
There are some functions which do not belong to previous classes.
\subsection{properties(): defining properties of some subjects.}
\label{properties}
This function has been used to define properties of factories.
It will replaced with othe means in later versions.
\subsection{cpu() gives the elapsed cpu time}
\label{cpu}
\begin{example}[cpuex]Example of cpu-timing\\
\label{cpuex}
cpu0=cpu()\\
a=\textcolor{VioletRed}{matrix}(100000)\\
a=\textcolor{VioletRed}{ran}() \textcolor{green}{!uniform}\\
\textcolor{VioletRed}{mean}(a),\textcolor{VioletRed}{sd}(a),\textcolor{VioletRed}{min}(a),\textcolor{VioletRed}{max}(a);\\
cpu1=cpu()\\
elapsed=cpu1-cpu0;
\end{example}
\subsection{seconds() gives the elapsed clock time}
\label{seconds}
\begin{example}[secondsex]Example of elapsed time\\
\label{secondsex}
cpu0=cpu()\\
sec0=seconds()\\
a=\textcolor{VioletRed}{matrix}(100000)\\
a=\textcolor{VioletRed}{ran}() \textcolor{green}{!uniform}\\
\textcolor{VioletRed}{mean}(a),\textcolor{VioletRed}{sd}(a),\textcolor{VioletRed}{min}(a),\textcolor{VioletRed}{max}(a);\\
cpu1=cpu()\\
sec1=seconds()\\
elapsed=cpu1-cpu0;\\
selapsed=sec1-sec0;
\end{example}
\subsection{Object names}
\label{onames}
Object names start with letter or with '\$'. Object names can contain any of
symbols '\#.\%§$\backslash$\_'. \textbf{J} is using '\%' to name objects related to some other objects.
E.g. function \textcolor{VioletRed}{stat}(x1,x2,\textcolor{blue}{mean->}) will store means of variables x1 and x2 into
variables x1\%mean and x2\%mean. Objects with name starting with '\$' are not
stored in the automatically created lists of input and output variables when
defining transformation objects. The variable Result which
is the output variable, if no output is given, is not put into these lists.
Object names can contain special characters (e.g. +-*=()) if these are closed
within ‘[‘ and ‘]’, e.g. a[2+3]. This possibility to include additional information
is borrowed from Markku Siitonen, the developer of Mela software.
If an transformation object is created with \textcolor{VioletRed}{trans}() function, and the
intended global arguments are given in the list of arguments,
then a  local object {ob} created e.g. with transformation object {tr} have prefix
{tr$\backslash$} yelding {tr\textcolor{green}{!\ob}.\,Actually\,also\,these\,objects\,are\,global,\,but\,their\,prefix}
protects them so that they do not intervene with objects having the same name in the
calling transformation objec.
There are many objects intitilized automatically.
Some of these are locked so that the users cannot change them.
Names of objects having a predefined interpretation start with capital letter.
The user can freely use lower or upper case letters. \textbf{J} is case sensitive.
All objects known at a given point of a \textbf{J} session can be listed by command:
\textcolor{VioletRed}{print}(Names)
\subsection{Copying object: a=b}
\label{copy}
A copy of object can be made by the assignment statement a=b.
\subsection{Deleting objects: delete\_o()}
\label{delete}
When an object with a given name is created, the name cannot be removed. With
delete\_0() function one can free all memory allocated for data structures needed
by general objects:
delete\_o(obj1,…,objn)
After deleting an object, the name refers to a real variable (which is
initialized by the delete\_o() function into zero).
\begin{note}
Other objects except matrices can equivalently be deleted by giving
command
obj1,…,objn = 0
This is because the output objects of any functions are first deleted before
defining them anew. Usually an object is automatically deleted if the object
name is as an output object for other functions.
\end{note}
\begin{note}
One can see how much memory each object is using \textcolor{VioletRed}{print}(Names).mes
\end{note}
\begin{note}
Deleting a compound object deletes also such
subobjects which have no meaning when the main object is deleted. But e.g. if a
data object is deleted then the as-sociated transformation object is not
deleted as the transformation can be used independently.
\end{note}
\begin{note}
Files can be deleted with delete\_f(file). See IO-functions for
details.
\end{note}
\begin{note}
If the user has defined own new compound objects in the open source
\textbf{J} software she/he needs to define the associated delete function.
\end{note}
\subsection{Object types}
\label{otypes}
The following description describes shortly different object types available in
\textbf{J}. More detailed descriptions are given in connection of \textbf{J} functions which
create the objects and in Developers' guide.
\subsubsection{Real variables and constants}
\label{realo}
A REAL variable is a named object associated with a single
double precision value. Before version J3.0 the values were in single
precision, and thus this objecttype is still called REAL.
The
value can be directly defined at the command level, or the variable can get
the value from data structures.  E.g.
\\
\textcolor{VioletRed}{stat}(D,H,\textcolor{blue}{min->},\textcolor{blue}{max->}) &\textcolor{green}{!\,Here\,arguments\,must\,be\,variable\,names}
\\
\\
a = \textcolor{VioletRed}{sin}(2.4) &\textcolor{green}{!\,argument\,is\,in\,radians\,\textcolor{VioletRed}{sind}()\,is\,for\,degrees}
\\
h = \textcolor{VioletRed}{data}(\textcolor{blue}{read->}(x1…x4))  & \textcolor{green}{!\,x1,\,x2\,,x3,\,x4\,are\,variables\,in\,the\,data\,set,\,and}
\\
get their values when doing operations for the data.
\\


\begin{note}
All objects have also an associated REAL value. In order to make
arithmetic operations fast, the argument types in simple arithmetic functions
are not checked. If a general object is used as an argument in an arithmetic
operation, then the REAL value associated with the object is used. This will
usually prevent the program to stop due to Fortran errors, but will produce
unintended results.
\end{note}

\begin{note}
In this manual 'variable' refers to a \textbf{J} object whose type is REAL.
\end{note}
\subsubsection{Character constants and variables}
\label{charo}
Character constants are generated by closing text within apostrophe signs ( '
). Apostrophe character ( ' ) within a character constant is indicated with ($\sim$)
(if the character $\sim$\,is\,not\,present\,in\,the\,keyboard,\,it\,can\,be\,produced\,by\,<Alt
>126, where numbers are entered from the numeric keyboard) Character constants are used e.g.  in I/O functions for file names, formats
and to define text to be written.  , e.g
\\
a = \textcolor{VioletRed}{data}(\textcolor{blue}{in->}'file1.dat', \textcolor{blue}{read->}(x1,x2)) &
\\

\\
\textcolor{VioletRed}{write}('output.txt', '($\sim$kukkuu=$\sim$,4f7.0)', \textcolor{VioletRed}{sqrt}(a)) &
\\
Character variables are pointers to character constants. An example of a
character variable definition:


\\
cvar='file1.dat' &
\\
After defining a character variable, it can be used exactly as the character
constants.
\begin{note}
The quotation mark (") has special meaning in the input programming. See
Input programming how to use character constants within character constants.
\end{note}
\subsubsection{Logical values}
\label{logical0}
There is no special object type for logical variables. Results of logical
operations are stored into temporary or named real variables so that 0 means
False and 1 means True. In logical tests all non-zero values will mean True.
Thus e.g. \textcolor{VioletRed}{if}(6)b=7 is legal statement, and variable b will get value 7. E.g.
\\
\textcolor{Red}{sit>}h=a.lt.b.and.b.le.8 &
\\
\textcolor{Red}{sit>}\textcolor{VioletRed}{print}(h) &
\\
h=   1.00000 &
\\
\subsection{Predefined objects}
\label{pref}
The following objects are generated during the initilaization.


\noindent \textcolor{teal}{Names} \tabto{25mm }   Text \tabto{45mm }   Text object containg the names of named objects


\noindent \textcolor{teal}{Pi} \tabto{25mm }  REAL \tabto{45mm }  The value of Pi (=3.1415926535897931)















\noindent \$\textcolor{teal}{Cursor}\$ \tabto{25mm }   TRANS  \tabto{45mm }   The transformation object used to run \textcolor{Red}{sit>} prompt


\noindent \$\textcolor{teal}{Cursor2}\$ \tabto{25mm }   TRANS  \tabto{45mm }   Another transformation object used to run \\
\tabto{45mm } \textcolor{Red}{sit>} prompt


\noindent \textcolor{teal}{Val} \tabto{25mm }  TRANS \tabto{45mm }   Transformation object used to extract values of mathematical statements, used,


\noindent \textcolor{teal}{Round}  \tabto{25mm }  REAL \tabto{45mm }   \textcolor{VioletRed}{jlp}(): The current round through treatment units in \textcolor{VioletRed}{jlp}() function.


\noindent \textcolor{teal}{Change} \tabto{25mm }  REAL \tabto{45mm }   \textcolor{VioletRed}{jlp}(): The change of objective in \textcolor{VioletRed}{jlp}()  in one round before finding feasible and thereafter


\noindent \textcolor{teal}{Imp} \tabto{25mm }  REAL \tabto{45mm }   \textcolor{VioletRed}{jlp}(): The number of improvements obtained from schedules outside the current active


\noindent \$\textcolor{teal}{Data}\$  \tabto{25mm }   List \tabto{45mm }   Default data set name for a new data set created by \textcolor{VioletRed}{data}()-function.





\noindent \textcolor{teal}{Obs} \tabto{25mm }   REAL  \tabto{45mm }   The default name of variable obtaining the the number of

\noindent \textcolor{teal}{Maxnamed}  \tabto{25mm }  REAL \tabto{45mm }   The maximum number of named objects. Determined via j.par in


\noindent \textcolor{teal}{Record} \tabto{25mm }   REAL  \tabto{45mm }   The name of variable obtaining the the number of


\noindent \textcolor{teal}{Subecord} \tabto{25mm }   REAL  \tabto{45mm }   The name of variable obtaining the the number of


\noindent \textcolor{teal}{Duplicate} \tabto{25mm }   REAL \tabto{45mm }   A special variable used in \textcolor{VioletRed}{data}() function when duplicating observations


\noindent \textcolor{teal}{LastaData}  \tabto{25mm }  List \tabto{45mm }  	A list object referring to the last data set made, used as default data set.


\noindent \$\textcolor{teal}{Buffer} \tabto{25mm }  Char \tabto{45mm }   A special character object used by the \textcolor{VioletRed}{write}() function.


\noindent \$\textcolor{teal}{Input}\$ \tabto{25mm }   Text  \tabto{45mm }   Text object used for original input line.


\noindent \textcolor{teal}{1}\$Input1\$  \tabto{25mm }  Text \tabto{45mm }   Text object for input line after removing blanks and comments.


\noindent \textcolor{teal}{Data}  \tabto{25mm }  List \tabto{45mm }    List object used to indicate current data setsDat

































\noindent \$ \\textcolor{teal}{tabto{25mm} }  REAL \tabto{45mm }   Object name used to indicate console and '*' format in reading and writing



\noindent \textcolor{teal}{x}\# \tabto{25mm }  REAL \tabto{45mm }   Variable used when drawing functions.

\noindent \textcolor{teal}{Selected} \tabto{25mm }   REAL  \tabto{45mm }  Variable used to indicate the simulator selected in simulations

\noindent \textcolor{teal}{Printinput} \tabto{25mm }   REAL \tabto{45mm }   Variable used to specify how input lines are printed. Not properly used.

\noindent \textcolor{teal}{Prinoutpu} \tabto{25mm }   REAL \tabto{45mm }   Variable used to indicate how much output is printed. Not properly used.






\noindent \$\textcolor{teal}{Debug} \tabto{25mm }  REAL \tabto{45mm }   Variable used to put debugging mode on.









\noindent \textcolor{teal}{Accepted}  \tabto{25mm }  REAL \tabto{45mm }   The number of accepted observations in functions using data sets.


\noindent \textcolor{teal}{Arg} \tabto{25mm }   REAL  \tabto{45mm }   The default argument name when using transformation object as a function.



\noindent \textcolor{teal}{Continue}  \tabto{25mm }  REAL \tabto{45mm }   If Continue has nonzero value then the control does not return to the



\noindent \textcolor{teal}{Err} \tabto{25mm }  REAL \tabto{45mm }   If Continue prevents the control from returning to \textcolor{Red}{sit>} prompt







\noindent \textcolor{teal}{Result} \tabto{25mm }   ?  \tabto{45mm }   The default name of output object.

\subsubsection{Transformation object}
\label{transo}
A transformation object groups several operation commands together so that they
can be used for different purposes by \textbf{J} functions and \textbf{J} objects. A
transformation object contains the interpreted transformations. For more details
see \textbf{J} function for defining transformation objects: \textcolor{VioletRed}{trans}().
Transformation objects can be called using \textcolor{VioletRed}{call}() function, so that all
transformations defined in the object are done once. Function result() also calls
transformations but is also returning a value. When transformation objects are
linked to data objects, then the transformations defined in trans-formation object
are done separately for each observation.
There is an implicit transformation object \$Cursor\$ which is used to run the
command level. The name \$Cursor\$ may appear in error messages when doing
commands at command level.  An-other transformation object \$Val\$ which is used to
take care of the substitutions of "-sequences in the input programming. Some \textbf{J}
functions use also implicitly transformations object \$Cursor2\$.
\subsection{Code options}
\label{codeopt}
There are some special options which do not refer to object names or values. Some options
define a small one-statement transformations to be used to compute something repeatedly.
As these one-statement can use transformation objects as functions, the code option can actually
execute long computations.
\begin{example}[codeoptex]Codeoptions\\
\label{codeoptex}
dat=\textcolor{VioletRed}{data}(\textcolor{blue}{read->}(D,H),\textcolor{blue}{in->})\\
3,2\\
2,4\\
4,1\\
/\\
\textcolor{VioletRed}{stat}(D,H,\textcolor{blue}{filter->}(H.gt.D)) \textcolor{green}{!\,only\,those\,observations\,are\,accepted\,which\,pass\,the\,filter}\\
fi=\textcolor{VioletRed}{draw}(\textcolor{blue}{func->}(\textcolor{VioletRed}{sin}(\$x)+1),\textcolor{blue}{x->}\$x,\textcolor{blue}{xrange->}(0,10),\textcolor{blue}{color->}Red,\textcolor{blue}{ylabel->}'\textcolor{VioletRed}{sin}(x)+1',\textcolor{blue}{xlabel->}'x',\textcolor{blue}{width->}2) \textcolor{green}{!\,the\,\textcolor{blue}{func->}\,option\,transmits\,the\,function\,to\,be\,drawn\,not
\end{example}
\subsection{Common options}
\label{comoptions}
There are some options which are used in many \textbf{J} functions. Such options are e.g.
@@in2
@@data
All data sets will be
treated logically as a single data set.
If the function is using data sets, the daenta sets are given in \textcolor{blue}{data->} option. All data sets will be
treated logically as a single data set. If a \textbf{J} function needs to access data, and the \textcolor{blue}{data->}
option is not given then \textbf{J} uses default data which is determined as follows.
If the user has defined an object list Data consisting of one or more data sets, then these will
be used as the default data set. E.g.
Data=\textcolor{VioletRed}{list}(dataa,datab)
When a data set is created, it will automatically become the only element in LastData list. If
the Data list has not been defined and there is no \textcolor{blue}{data->} option, then the LastData dataset
will be used.
@@trans
In all functions which are using data sets, \textcolor{blue}{trans->} option defines a
transformation set which is used in this function.
@@filter
\hline
\vspace{0.2cm}
\begin{example}[comoptex]data1\\
\label{comoptex}
dat=\textcolor{VioletRed}{data}(\textcolor{blue}{read->}(x,y),\textcolor{blue}{in->})\\
1,2\\
3,4\\
5,6\\
/\\
tr=\textcolor{VioletRed}{trans}()\\
xy=x*y\\
x,y,xy;\\
/\\
\textcolor{VioletRed}{stat}(\textcolor{blue}{trans->}tr)
\end{example}
\subsection{Excuting transformation object explicitly \textcolor{VioletRed}{call}()}\index{call()}
\label{call}

Interpreted transformations in a transformation object can be automatically executed by other \textbf{J}
functions or they can be executed explicitly using \textcolor{VioletRed}{call}() function.

\vspace{0.3cm}
\hline
\vspace{0.3cm}
\noindent Arg \tabto{3cm} 1 \tabto{5cm}  TRANS \tabto{7cm}
\begin{changemargin}{3cm}{0cm}
\noindent  The transformation object executed.
\end {changemargin}
\hline
\vspace{0.2cm}

\begin{note}
A transformation objects can be used recursively, i.e. a transformation can be called from
itself. The depth of recursion is not controlled by J, so going too deep in recursion will
eventually lead to a system error.
\end{note}
\begin{example}[recursion]Recursion produces system crash.\\
\label{recursion}
tr=\textcolor{VioletRed}{trans}() \textcolor{green}{!level\,will\,be\,initialized\,as\,zero}\\
level;\\
level=level+1\\
\textcolor{VioletRed}{call}(tr)\\
/\\
Continue=1 \,\textcolor{green}{!error\,is\,produced}\\
\textcolor{VioletRed}{call}(tr)\\
Continue=0
\end{example}
\subsection{\textbf{J} transformations}
\label{jtrans}
Most operation commands affecting \textbf{J} objects can be entered directly at the command level or
packed into transformation object. In both cases the syntax and working is the same. A
command line can define arithmetic operations for real variables or matrices, or they can
include functions which operate on other \textbf{J} objects. General \textbf{J} functions can have arithmetic
statements in their arguments or in the option values. In some cases the arguments must be
object names. In principle it is possible to combine several general \textbf{J} functions in the same
operation command line, but there may not be any useful applications yet, and possibly some
error conditions would be generated.
Definition: A numeric function is a \textbf{J} function which returns a single real value. These functions
can be used within other transformations similarly as ordinary arithmetic functions. E.g.
\textcolor{VioletRed}{weights}() is a numeric function returning the number of schedules having nonzero weight
in a JLP-solution. Then \textcolor{VioletRed}{print}(\textcolor{VioletRed}{sqrt}(\textcolor{VioletRed}{weights}())+Pi) is a legal transformation.
\subsection{Generating a transformation object \textcolor{VioletRed}{trans}()}\index{trans()}
\label{trans}
\textcolor{VioletRed}{trans}() function interprets lines from input paragraph following the \textcolor{VioletRed}{trans}() command and puts the
interpreted code into an integer vector, which can be excuted in several places.
If there are no arguments in the function, the all objected used within the
transforamations are global. This may cause conflicts if there are several recursive
functions operating at the same time with same objects. \textbf{J} checks some of
these conflict situations, but not all.  These conflicts can be avoided by giving
intended global arguments  in the list of arguments.
Then an object 'ob' created e.g. with transformation object \textcolor{teal}{tr} have prefix
]tr$\backslash$[ yelding ]tr$\backslash$ob[. Actually also these objects are global, but their prefix
protects them so that they do not intervene with objects having the same name in the
calling transformation objec.

Each line in the input paragraph is read and interpreted and packed into a transformation
object, and associated tr\%input and tr\%output lists are created for input and output
variables. Objects can be in both lists. Objects having names starting
with '\$' are not put into the input or output lists. The source code is saved in a text object
tr\%source. List tr\%arg contains all arguments.
!
If a semicolon ';'  is at the end of an input line, then
the output is printed if REAL variable Prindebug has value 1 or value>2 at
the execution time. If the double semicolon ';;' is at the end then the output is
printed if Printresult>1. If there is no output, but just list of objects, then these
objects will be printed with semicolns.

!
\vspace{0.3cm}
\hline
\vspace{0.3cm}
\noindent Output \tabto{3cm} 1 \tabto{5cm}  Data \tabto{7cm}
\begin{changemargin}{3cm}{0cm}
\noindent The TRANS object generated.
\end{changemargin}
\vspace{0.3cm}
\hline
\vspace{0.3cm}
\noindent Args \tabto{3cm} N|1- \tabto{5cm}    \tabto{7cm}
\begin{changemargin}{3cm}{0cm}
\noindent  Global objects.
\end {changemargin}
\hline
\vspace{0.2cm}
\begin{note}
Options input->, \textcolor{blue}{local->}, \textcolor{blue}{matrix->}, \textcolor{blue}{arg->}, result->, \textcolor{blue}{source->} of previous
versions are obsolte.
\end{note}
\begin{note}
The user can intervene the execution from console if the code calls \textcolor{VioletRed}{read}(\$,),
\textcolor{VioletRed}{ask}(), \textcolor{VioletRed}{askc}() or \textcolor{VioletRed}{pause}() functions. During the pause one can give any command excepts
such input programming command as \textcolor{Red}{;incl}.
\end{note}
\begin{note}
The value of Printresult can be changed in other parts of the transformation, or
in other transforamations called or during execution of \textcolor{VioletRed}{pause}().
\end{note}

\begin{note}
Output variables in \textcolor{blue}{maketrans->} transformations whose name start with \$ are not put into the new data object.
\end{note}
\begin{example}[transex]Demonstrates also error handling\\
\label{transex}
tr=\textcolor{VioletRed}{trans}()\\
\$x3=x1+3\\
x2=2/\$x3;\\
/\\
tr\%input,tr\%output,tr\%source;\\
x1=8\\
\textcolor{VioletRed}{call}(tr)\\
tr2=\textcolor{VioletRed}{trans}(x1,x2)\\
\$x3=x1+3\\
x2=2/\$x3;\\
x3=x1+x2+\$x3;\\
/\\
tr2\%input,tr2\%output,tr2\%source;\\
\textcolor{VioletRed}{call}(tr2)\\
tr2\x3; \,\,\textcolor{green}{!x3\,is\,now\,local}\\
tr3=\textcolor{VioletRed}{trans}()\\
x1=-3\\
\textcolor{VioletRed}{call}(tr) \textcolor{green}{!this\,is\,causing\,division\,by\,zero}\\
/\\
Continue=1 \,\,\textcolor{green}{!\,continue\,after\,error}\\
\textcolor{VioletRed}{call}(tr3)
\end{example}
\color{Green}
\begin{verbatim}
sit>transex
<;incl(exfile,from->transex)


<tr=trans()
<$x3=x1+3
<x2=2/$x3;
</
<tr%input,tr%output,tr%source;
tr%input is list with            2  elements:
x1 $x3
tr%output is list with            1  elements:
x2
tr%source is text object:
1 $x3=x1+3
2 x2=2/$x3;
3 /
///end of text object

<x1=8


<call(tr)
x2=0.18181818

<tr2=trans(x1,x2)
<$x3=x1+3
<x2=2/$x3;
<x3=x1+x2+$x3;
</
<tr2%input,tr2%output,tr2%source;
tr2%input is list with            1  elements:
x1
tr2%output is list with            1  elements:
x2
tr2%source is text object:
1 $x3=x1+3
2 x2=2/$x3;
3 x3=x1+x2+$x3;
4 /
///end of text object

<call(tr2)
x2=0.18181818

tr2\x3=19.1818181

<tr2\x3;
tr2\x3=19.1818181

<tr3=trans()
<x1=-3
<call(tr)
</
<Continue=1
<call(tr3)
*division by zero
*****error on row            2  in tr%source
x2=2/$x3;
recursion level set to    3.0000000000000000

*****error on row            2  in tr3%source
call(tr)
recursion level set to    2.0000000000000000

*err* transformation set=$Cursor$
recursion level set to    1.0000000000000000
****cleaned input
call(tr3)
*Continue even if error has occured
<;return
\end{verbatim}
\color{Black}
\subsection{Using a transformation object as a function}
\label{transfunc}
It is now possible to use a transformation object as a function which computes new
objects when generating arguments for functions or options,
or values of code options, or in any place within a transformation object.
If {tr} is a transformation and the transformation computes an object {A} then
{tr(A)} is first calling transformation {tr} and provides then object {A} into this place.
As the transformation computes also other objects which are computed within it, also thes objects
are available. At this point it is important to note that arguments of
a transformation line are computed from right to left, because options must be computed before
entering into a function.
\begin{example}[transfunc]Transformation as a function\\
\label{transfunc}
delete\_o(a,c)\\
tra=\textcolor{VioletRed}{trans}()\\
a=8;\\
c=2;\\
/\\
trb=\textcolor{VioletRed}{trans}()\\
a=5;\\
c=1;\\
/\\
c=2\\
a=c+trb(a)+c+tra(a);
\end{example}
\subsection{Transformation control structures}
\label{transcont}
Within \textbf{J} transformations, there can be similar controls structures as in the input programming.
The  difference  is  that  these  will  remain  as  part  of  the  transformation  set.  Only  the
'\textcolor{VioletRed}{if}()output=…'structure is allowed at the command level, other are possible only within a
transformations set.
\subsubsection{\textcolor{VioletRed}{if}()}\index{if()}
\label{if}

\textcolor{VioletRed}{if}()j\_statement… \newline
The one line if-statement.
\subsubsection{\textcolor{VioletRed}{if}() elseif() else endif}\index{if()}
\label{ifthen}
There can be 4 nested \textcolor{VioletRed}{if}()then structures. If-then-structures are not
allowed at command level.

\\
\textcolor{VioletRed}{if}()then
\\
….
\\
elseif()then
\\
…
\\
else
\\
….
\\
endif
\\
\subsection{Loops and control strucures}
\label{loops}
This section describes nonstadard functions.
\subsection{return}
\label{return}
Return from a transformation set to the transformation object where \textcolor{VioletRed}{call}()
function was, or  to the include file with \textcolor{VioletRed}{call}(), or to the \textcolor{Red}{sit>} promt,
is \textcolor{VioletRed}{call}() was at \textcolor{Red}{sit>}.
\begin{example}[retex]example of return and goto ()\\
\label{retex}
tr=\textcolor{VioletRed}{trans}()\\
ad1:r=\textcolor{VioletRed}{ran}();\\
\textcolor{VioletRed}{if}(r.lt.0.2)return\\
\textcolor{VioletRed}{goto}(ad1)\\
/\\
\textcolor{VioletRed}{call}(tr)
\end{example}
\begin{note}
return is automatically put to the end of transformation object.
\end{note}
\begin{note}
Labels in a transformation object are without ';' as the addresses in an include file start with ';'.
\end{note}

\subsection{\textcolor{VioletRed}{errexit}()}\index{errexit()}
\label{errexit}
Function \textcolor{VioletRed}{errexit}() returns the control to \textcolor{Red}{sit>} prompt with a message similarly
as when an error occurs.

\begin{example}[errexitex]itex\\
\label{errexitex}
tr=\textcolor{VioletRed}{trans}()\\
\textcolor{VioletRed}{if}(a.eq.0)\textcolor{VioletRed}{errexit}('illegal value ',a)\\
s=3/a; \textcolor{green}{!\,division\,with\,zero\,is\,teste\,automatically}\\
/\\
a=3.7\\
\textcolor{VioletRed}{call}(tr)\\
tr(s); \textcolor{green}{!tr\,can\,also\,be\,used\,as\,a\,function}\\
a=0\\
Continue=1 \,\textcolor{green}{!Do\,not\,stop\,in\,thsi\,seflmade\,error}\\
\textcolor{VioletRed}{call}(tr)\\
Continue=0
\end{example}
\subsection{\textcolor{VioletRed}{goto}()}\index{goto()}
\label{goto}
Control can be transfered to a line in a transformation set with \textcolor{VioletRed}{goto}().
\begin{example}[gotoex]\\
\label{gotoex}
tr=\textcolor{VioletRed}{trans}()\\
i=0\\
\textcolor{VioletRed}{if}(i.eq.0)\textcolor{VioletRed}{goto}(koe)\\
'here';\\
koe:ch='here2';\\
/\\
\textcolor{VioletRed}{call}(tr)\\
ch;

\end{example}

\begin{note}
It is not allowed to jump in to a loop or into if -then structure. This is
checked already in in the interpreter.
\end{note}
\begin{note}
It is not yet possible to continue within an include file using Continue=1.
\end{note}
\begin{note}
It is not recommended to use \textcolor{VioletRed}{goto}() according to modern computation practices.
However, it was easier to implement cycle and exitdo with \textcolor{VioletRed}{goto}(), especially if
cycle or exitdo does not apply hte innermost do-loop.
\end{note}
\subsection{Numeric operations}
\label{numer}
An arithmetic expression consisting of ordinary arithmetic operations is formed in
the standard way. The operations are in the order of their precedence.
\begin{itemize}
\item[\textbf{J}] - unary minus
\item[\textbf{J}] *** integer power
\item[\textbf{J}] ** or \^ real power
\item[\textbf{J}] * multiplication
\item[\textbf{J}] / division
\item[\textbf{J}] + addition
\item[\textbf{J}] - subtraction
\end{itemize}

The reason for having a different integer power is that it is faster to compute and a negative
value can have an integer power but not a real power.

In matrix computations there are two addtional opertations.

\begin{itemize}
\item[\textbf{J}] *. elementwise product (Hadamard product)
\item[\textbf{J}] /. elementwise division
\end{itemize}

The matrix operations are explained in section ?. Their operation rules extent
the standard rules.
\subsection{Logical and relational expressions}
\label{logic}
There are following relational and logical operations. The first alternatives
follow Fortan style:
\begin{itemize}
\item[\textbf{J}] .eq. == equal to
\item[\textbf{J}] .ne. <> not equal
\item[\textbf{J}] .gt. > greater than
\item[\textbf{J}] .ge. >= greater or equal
\item[\textbf{J}].lt. < less than
\item[\textbf{J}] .le. <= less or equal
\item[\textbf{J}].not. $\sim$\,negation
\item[\textbf{J}] .and. & conjunction
\item[\textbf{J}] .or. disjunction
\item[\textbf{J}] .eqv. equivalent.
\item[\textbf{J}] .neqv. not equivalent
\end{itemize}

The relational and logical expressions produce value 1 for True and value 0 for False.
Note: Testing equivalence can be done also using 'equal to' and 'not equal', as the same truth
value is expressed with the same numeric value.
\begin{note}
when the truth value of an expression is tested with \textcolor{VioletRed}{if}(), then all nonzero real values
means that the expression is true.
\end{note}


\subsection{Arithmetic functions}
\label{arfu}
The arithmetic functions return single REAL value or a MATRIX.
\textcolor{VioletRed}{sqrt}(), \textcolor{VioletRed}{sqrt2}(), \textcolor{VioletRed}{exp}(), \textcolor{VioletRed}{log}(), \textcolor{VioletRed}{log10}(), \textcolor{VioletRed}{abs}()
\begin{itemize}
\item[\textbf{J}] \textcolor{VioletRed}{sqrt}(x) square root, \textcolor{VioletRed}{sqrt}(0) is defined to be 0, negative argument produce error.
If {x} is matrix, then an error occurs if any elemet is negative.
\item[\textbf{J}] \textcolor{VioletRed}{sqrt2}(x) If {x} or an element of {x} is negative then \$\textcolor{VioletRed}{sqrt2}()=-\textcolor{VioletRed}{sqrt}()\$. Actually \textcolor{VioletRed}{sqrt2}() might be a
useful sigmoidal function in modeling context.
\item[\textbf{J}] \textcolor{VioletRed}{exp}(x) \$e\$ to power {x}. If {x}>88, then \textbf{J} produces error in order to avoid system
crash.
\item[\textbf{J}] \textcolor{VioletRed}{log}(x) natural logarithm
\item[\textbf{J}] \textcolor{VioletRed}{log10}(x) base 10 logarithm
\item[\textbf{J}] \textcolor{VioletRed}{abs}(x) absolute value
\end{itemize}

Real to integer conversion
\begin{itemize}
\item[\textbf{J}] \textcolor{VioletRed}{nint}(x) nearest integer value
\item[\textbf{J}] \textcolor{VioletRed}{nint}(x,modulo) returns modulo*nint(x/modulo) ,e.g.\textcolor{VioletRed}{nint}(48,5)=50; \textcolor{VioletRed}{nint}(47,5)=45;
\item[\textbf{J}] \textcolor{VioletRed}{int}(x) integer value obtained by truncation
\item[\textbf{J}] \textcolor{VioletRed}{int}(x,modulo) returns modulo*int(x/modulo), e.g. \textcolor{VioletRed}{int}(48,5)=45
\item[\textbf{J}] \textcolor{VioletRed}{ceiling}(x) smallest integer greater than or equal to {x}.
\item[\textbf{J}] \textcolor{VioletRed}{ceiling}(x,modulo) returns modulo*ceiling(x/modulo), e.g. \textcolor{VioletRed}{ceiling}(47,5)=50.
\item[\textbf{J}] \textcolor{VioletRed}{floor}(x) greatest integer smaller than or equal to {x}.
\item[\textbf{J}] \textcolor{VioletRed}{floor}(x,modulo) returns modulo*floor(x/modulo), e.g. \textcolor{VioletRed}{floor}(47,5)=45.
\end{itemize}

The following rules apply both for \textcolor{VioletRed}{min}() and \textcolor{VioletRed}{max}(). The rules are presented here only for \textcolor{VioletRed}{min}().
\begin{itemize}
\item[\textbf{J}] \textcolor{VioletRed}{min}(x1,...,xn): \$n>1\$ and all arguments are REAL, \textcolor{teal}{Result} is REAL.
\item[\textbf{J}] \textcolor{VioletRed}{min}(A): If \textcolor{teal}{A} is matrix then the result result is row vector containg minumum
of each column. If \textcolor{teal}{A} is a column vector, \textcolor{teal}{Result} is REAL.
\item[\textbf{J}] \textcolor{VioletRed}{min}(A,\textcolor{blue}{any->}) If \textcolor{teal}{A} is matrix, then \textcolor{teal}{Result} minimum over all elements.
\item[\textbf{J}] \textcolor{VioletRed}{min}(x,A) If \textcolor{teal}{x} is REAL and \textcolor{teal}{A} is matrix, then \textcolor{teal}{Result} is
matrix with the same dimensions as \textcolor{teal}{A} and
\textcolor{teal}{Result}(i,j)=\textcolor{VioletRed}{max}(\textcolor{teal}{x},\textcolor{teal}{A}(i,j)). The order of arguments does not matter.
\item[\textbf{J}] \textcolor{VioletRed}{min}(A,B), \textcolor{teal}{A} and \textcolor{teal}{B} compatible matrices. \textcolor{teal}{Result} is amatrix with the same dimesnions containg elementwise
minimums.
\end{itemize}
\subsubsection{Text objects}
\label{textob}
Currently there are two text object types, the old text object TEXT and the new
TXT. The TEXT object stores text in a long vector of single characters. The TXT
object stores text in lines of 132 characters. The TEXT objects save memory but
are not so easy to modify and use. Several \textbf{J}
functions create associated text objects. \textbf{J} functions \textcolor{VioletRed}{text}() and \textcolor{VioletRed}{txt}()
can be used to
create text objects directly. All the names of \textbf{J} objects are also stored in a
TEXT object called Names. The number of lines in a text objects can be obtained
with \textcolor{VioletRed}{nrows}() function and the total number of characters can be obtained with
\textcolor{VioletRed}{len}() function.
\subsection{The main components of \textbf{J} program}
\label{compo}
The structure of \textbf{J} program
\begin{itemize}
\item[\textbf{J}] Input programming which generates text input for the interpreter (subroutine
j\_getinput). \textbf{J} commands are obtained:
\begin{itemize}
\item[\textbf{J}] from \textcolor{Red}{sit>} prompt
\item[\textbf{J}] form possibly nested include files
\end{itemize}
\item[\textbf{J}] Interpreter which generates from text lines integer vectors containing function indices,
option indices and object indices (subroutine j\_interpret).
\item[\textbf{J}] Function driver which executes the code in the interpreted integer vector
(subroutine dotrans). The the function driver is using:
\begin{itemize}
\item[\textbf{J}] \textbf{J} functions which operate on arguments which are determined either as formal arguments or
via options.
\item[\textbf{J}] \textbf{J} objects
\item[\textbf{J}] Global variables and matrices
\item[\textbf{J}] Utility subroutines
\end{itemize}


\end{itemize}
A user of \textbf{J} needs only input programming and \textbf{J} functions, but understanding of
the other properties may help to understand waht is going on in a \textbf{J} session.
\section{Future development}
\label{future}
I think that the current version of \textbf{J} provides many possibilities for future developments.
For instance:
\begin{itemize}
\item\item[\textbf{J}] The current version does not have any special functions
for making simulators. The new \textcolor{VioletRed}{goto}() commands, possibility to work with submatrices,
and the new \textcolor{VioletRed}{transdata}() function provide muc more efficient ways to develop simulators.
Examples will be provided shortly.
\item[\textbf{J}] Using the possibility to compute derivatives using the analytic derivates makes it quite straightforward
to make it possible to have a nonlinear objective function
\item[\textbf{J}] It would be quite easy to develop \textbf{J} so that integer solution is produced with respect to the
schedule weight.
\item[\textbf{J}] It would be interesting to see how \textbf{J} can put to work with Heureka.
\item[\textbf{J}] The possiblity to run R scripts from \textbf{J} and \textbf{J} scripts from R provide new possiblities.
\item[\textbf{J}] \textbf{J} can now be used as an interface to Gnuplot. Google search show how many possiblities Gnuplot provides.
It is quite straightforward to implement these grpahs if it is not currently possible.
\item[\textbf{J}] The possibility to generate random numbers from any discrete or continuous distribution provide new
possibilities to stdy the effects of random errors in the optimization.
\item[\textbf{J}] The new tools for analyzing grouped data are usefulf when studying the grouped data. it would
be straightforward to implement mixed model methods based on expected means squares.
\end{itemize}
\section{\textbf{J} story}
\label{preface1}
!
she drawed for my 70 yr birthday. \\
\textbf{J} software version 3.0 is now publisched as a open source software. The preparation of the new version was
very complicated and frustrating process. The whole story of \textbf{J} is
presented in section \ref{jstory} where I make also some critical comments on how Luke did
treat this process. I got 7.10.2021 in a co-operation agreement with Luke

I will here describe the history of \textbf{J} in this preface in unusual
length and detail. Readers interested ony in the use of the current
software can perhasp look briefly the subsection Freedom. The reason
why I'm going through in detail
the difficulties with Luke is that I think that the policy insisted by the leaders of Luke
that the researchers of Luke do not have any moral or legal right to publish the results of their
research should be discussed. I think also that the leaders of Luke
were willing to waste large amount of taxpayers' money when the prevented the
development and timely publication of \textbf{J} software. I hope that this part of the
preface gets some attention also outside the potential users of the software e.g.
in the ministry. As an amateur actor, I also hope that some play writer would
notice this and would make tragicomedy on the stage.
\subsection{JLP and first version of \textbf{J}}
\label{preface2}

I started to develop \textbf{J} software as a successor of the linear programming software
\textbf{J}LP \cite{JLP} around
2004. I had also done several programs for statistical computing (AKTA and Jakta)
since late seventies
when there was not
easy to use software even for ordinary linear regression.
Later I had to write own software
for mixed linear models. AKTA, Jakta, and JLP contained  means to use script files and
make new variables described in arithmetic expressions. The starting idea in \textbf{J} was to put all my
previous software developments into one program.
Version 0.9.3 of \textbf{J} was published in August 2004.

In JLP the crucial property was the utilization of the generalized upper bound (GUB) technique which
I invented and, which I could later find also in the literature (\cite{dant}), as I expected. GUB is
just the right technique which is
needed in foreast management planning. JLP made it possible to solve
significantly larger LP problems significantly faster
than using standard commercial LP software. In JLP I programmed all the necessary matrix
operations needed to change columns of the basis matrix in the revised simplex method.
I updated the inverse of the basis matrix explicitly,
instead of using factorization procedures as is the standard way in LP.
I treated nonbinding constraints in a nonstandard way
by reducing the dimension of the basis matrix instead of using residual
basic variables, as is normally done.

When I started to develop J, I thought that it might be good idea to use matrix routines
based on factorizations and developed by a professional in this field.
I found the LP software bqbd by Fletcher
based on \cite{flet}. The bqbd software was meant to use 'as is' software for linear and quadratic programming.
When I told Fletcher that I plan to separate the matrix routines needed in the GUB algorithm,
he told he is certain that I will not succeed, as his Fortran code did not have any comments.
He wrote that he will not have any time to help me.
However, it was not that difficult becasue his code was divided
to proper subroutines. His code did computations in
single precision, and it was necessary to change computations
into double precision. Because I do not have
a proper training in the rounding error business,
I had hoped his advice in changing the several tolerance
parameters to correspont double precision computations. In the development of the original JLP algorith,
I spent two-three times more time to fight rounding errors than to
get the code mathematically correct. Now I understood the rounding errors better, but
anyhow they are the permanent nuisance. The fundamental question
with respect to rounding errors is: is a small-looking number zero or non-zero.

When I later told him how large problems and how fast my algorithm could solve,
his attitude towards
my efforts clearly changed.

Even Bergseng from NMBU started to use \textbf{J} in forest management problems
in a quite premature
stage, and we had quite tough time to keep his project in time.
It took some time to clear a bug in Fletcher
code which every now and then took the optimization into a wrong path.
I had a quite slow computer, and the
data transwer between Norway and Finland was slow. The bug was such that
it had never caused problems in
standard applications of bqbd software. In forest planning problems
it can happen that after many
pivot operations only residual variables corresponding nonbinding
constrains are in the basis.
And when \textbf{J} refactorized of the basis after fixed number of pivot operations,
as is the common procedure, the basis became slightly corrupted.
In such a case, the refactorization is not needed. NMBU started to use \textbf{J}
together with their GAYA simulator, and called the whole system GAYA-J,
as the previous system was called GAYA-JLP.
I appreciate that they have acknowledged our co-operation this way.
\textbf{J} version 0.9.3 was published in
August 2004.

\textbf{J} contained also a simulator language by which one describe a simulator
which can be used to geneate
treatment schedules which are later optimized with the
LP funtions of J. When Simosol started
to develop sofware which would compete with the Mela, which had had a
monopoly status in planning software
in Finland, and had all the problems which are inevitable in all monopoly projects.
Simosol made first simulator developments using J, but moved to other solutions before first
published version of Simo software. Simosol however used the
linear programming routines
of J.  They are still using \textbf{J} when being part of Afry.
\subsection{\textbf{J} with factories: J2.0}
\label{preface3}
For me the most disturbing feature of \textbf{J} and JLP was that they were not able to treat factories.
I thought that the clever heuristic DTRAN algorithm of Howard Hoganson
(\cite{dtran} and \cite{howard} showed
that it should be possible to extend basic ideas of the
\textbf{J}LP algorithm to treat with factory
problems. I was able to derive the needed equations (see \cite{lappilem}).
Reetta Lempinen started to work with me to implement the formulas. Reetta has professional
training in programming, while I'm a self-learning amateur. The necessary
data structures were quite complicated.
I had made some wrong decisions which made the data structes even
more complicated than was necessary.
Reetta had a clear head to keep data structures in order.
It was a nice new experience to
work together through video connection on the code which both could
see simultaneously.
I just gave commands, but it was not necessary to give detailed
instructions because Reetta
understood from a half word what to do. An important
contribution of her was that when there were several dead ends,
and I started to doubt whether we can really
do what we were aiming at, she told, 'listen Juuso, we can solve this problem as we have
solved all previous problems'. Interestingly, the problems were usually not in any programming errors,
but the path could be found by raising to a little upper level and going to the main ideas.
Reetta's role was essential in the factory optimization. Also later she has been helpful in
minor technical matters, in writing the manual, testing the new developments etc. Recently she
adviced me how to use Git and Github. Reetta has also been
solidary during the painful Luke years, when she was not
allowed to work with \textbf{J} as she had liked
to do. I appreciate her contibution very much. She is the second author of this
manual and the paper \cite{lappilem} for good reasons. However, I use word 'I'
for such situtions where Reetta has not been involved in the decisions how to proceed. The factory optimization was implemeted in
\textbf{J} version 2.0. published in 2013.
\subsection{Battle over open-source J3.0  starts in Luke}
\label{preface4}

Initially I made the software so that only I could understand the structure and the code.
When Reetta started to work with me in factory problems,
I learned what are the weakest points in the software if it should be maintained
by a group of people. But as I was always able to tell Reetta how the software worked,
only very slight improvements in the documentation and comments of the code etc. were made.
When I realized that my retirement age is approaching,
I started to worry how the software could be maintained and developed after my retirement.

I had worked for Finnish Forest Research Institute (Metla) with small Academy
and University of Joensuu brakes since 1978. In 2015 Metla and
two other institutes were merged into Institute of Natural Resources Finland (Luke).
Based on my long service in a government organization, I expected that
the new organization would mean
new leaders, new lawyers, new bureaucrats, and  much more bureaucracy,
and less time
for researchers to do research, as always happens when
government organizations simplify
their administration and make it more efficient.
My expectations were in the right direction, but even in my worst
nightmares I could not have imagined
what Luke would mean to me personally.

Lauri Meht\"atalo and I had agreed with an international publisher to
make a book about forest biometry. In Metla this kind of agreement had been
considered as a merit to Metla. But Luke forbad me to put any working time
to this book project. After retirement I was free to prepare this book with Lauri.
On the back-cover of the book I presented myself as a researcher of Metla
in order to give the affilation credit to Metla.
The co-operation
among statisticians of Luke, led by Juha Heikkinen, the continued co-operation
with Jaana Luoranen, and the support of the whole personell
of the Research Station of Suonenjoki when the leaders of Luke
started to humillate me
were my only positive experiences at Luke.

I started to clean and reorganize the \textbf{J} code so that it would be easier to maintain it.
My superior Olli Salminen forbad me to continue that effort because
he was certain that I would
never get that work ready. I didn't obey, of course not. The eternity of
this first cleaning round lasted two and
half weeks. Reorganization of files took 2-3 days, but programming
of a precompiler took
two weeks. In JLP I already had learned that a precompiler could help to manage
data areas of JLP. In J, the precompiler has greatly helped to manage
global variables.
The precompiler allows the users access global module variables
without knowing where they are. To separate the global variables
from local variables, the global variables have
j\_ or jlp\_ prefix. By changing the data structures,
\textbf{J} also became significantly faster
(main reason was that allocatable pointers were changed into allocatable arrays).


In 2014 when we were looking at how \textbf{J} could be utilized in Mela software, I encountered
old JLP concept ‘own function’, which was the method for allowing users of JLP to
add own (arithmetic) functions if they had access to the source code. I thought that this
idea would be useful also with respect to J. So, I organized \textbf{J} so that
different users could add their own \textbf{J} functions,
object types and options.

I was then convinced that the best way to utilize these improvements would be to
make the soft-ware open source. This would also allow me to continue my work freely
with \textbf{J} after my retirement.


The leaders of Luke agreed with me that the open-source \textbf{J} is a good idea.
Unfortunately they
also thought that open \textbf{J} is so important for Luke that the leaders could not
allow me to take care of the publication and prepare and present
the necessary decisions for their approval, or even participate as
member in the group which
started to hustle with the publication.
The leaders wanted themselves take care of the publication. Unfortunately, none
of the leaders involved
considered that the publication of \textbf{J} would be even so
important that he/she
had taken care that \textbf{J} would be eventually published.
They just wanted keep important
in the process. This absurd deadlock lasted almost seven years.

The previous version utilized some subroutines of IMSL package and from
the ‘Numerical Recipes in Fortran’ book. I replaced these subroutines mainly by
open access routines from Lapack and Ranlib packages from the Netlib library.
Roger Fletcher allowed to distribute his linear programming sub-routines in
the open \textbf{J} package. Fletcher gave a very liberal permission to use
his subroutines in open source distribution (the
price of his bqbd sofware was perhaps 100-200 Euros).
But the lawyers of Luke studied his permission
so long that he died before giving permission with better formulation.
A permission with
better formualations was obtained from his department in Dundee University.
But the leading
laywer of Luke, Emilia Katajajuuri was not satisfied, she wanted make
sure that the
legal position of the person
who gave the new permission was such that he could give the permission.



The following persons were involved in
process: P/"aivi Eskelinen, Olli Salminen, Kari T. Korhonen, Kimmo Kukkavuori,
Tuula Packalen, Sirpa Thessler, Sari Forsman-Hugg, Emilia Katajajuuri and Johanna
Buchert. Iikka Sainio was an outside legal consult.

The leading lawyer of Emilia Katajuuri made my position clear
in a video meeting to which I also was accidentally invited to
listen their new instructions.
When I accidentally said 'could we decide that..', she interrupted me and said:
'Listen Juha,
you do not decide anything in this matter'. Their message was that
I have done my part, and now
I should keep out and not disturb their play with their new conquest.

When it was evident that the leaders of Luke wanted to get rid of me,
I threatened that I will not retire before the opening decision is done.
Sari Forsman-Hugg told she will make the decision. But as my threat
seemed to make no influence,
I retried at the beginning of 2017. When I despite of all
the humiliation and frustration still hoped that
some kind a co-operation with Luke would be possible,
I applied the position outside researcher,
which had provided me access to Luke's e-mail and library services. Sari Forsman-Hugg was not
able to make a decision also with recpect to my application.
I did withdraw my application
when my e-mail connections were interrupted two times.

Later I heard from Reetta that Luke had made the opening decision but
she did not know what kind decsion and
I was not informed of the decision either, in disargeeent with the distribution list
of the decision, as I noticed in when I finally could see the decision after
three years.
\subsection{Retired, frozen conflict}
\label{preface5}
After my retirement I was still hoping that some kind of co-operation
without outside researcher status with Luke would be possible, and I
participated in the application of factory optimization which was published in \ref{pekka}.
Prof. Packalen told late 2017 that the publication of \textbf{J} is approaching.
When I asked about the opening decision, she 'did not
remember' what it contained but promised to check it out, which she
did forget to do or atl least forget to tell me what the decision contained.
Prof. Packalen was planning a very high level launch of
\textbf{J} with prominent international guests. I wrote that such
seminar would not probably be worth of
for trans-Atlantic flights. I was so stupid that
I believed her promise that the publication of \textbf{J} was approaching
and I submitted a correction to a bug which produced some amount of negative timber
quantities.

During 2018-2021 I made some small parameter adjustments for Simosol and Norway and made some additions
for my own research. Prof. Packalen had lost all interest in the publication,
and she did not allow Reetta to
develop the publication documents. When I heard 2020 that prof Packalen
had left Luke and could
not prevent publication any more, I asked Reetta to
ask her superiors whether she could continue
the preparation of manuals. She got the permission to use some time for that.
This work with all style details was not very nice for her as she
had already done big part of the work for the previous format of
Lukes's report series.
The report series had changed the formats, and Reetta had to do
such tedious work she already had done once before.
Reetta's superiors did never  discuss with me anything related to the
manuals or to the code. This
way they had been free to pretent that I have not contributed to
the work if \textbf{J} had been some time published.


In spring 2021 Hannu Salminen from Luke contacted me. He had memory
problems with a very large data set. I made a rapid fix for his acute
problem and did develop
more permanent solutions for large data sets using direct access files.
Actually that work was
unncessary as Viktor Strimbu from  NMBU  adviced me later how to make
\textbf{J} a 64-bit
application using MSYS2 environment. The Gfortran environment I had
used before did not
support 64-bit applications. The 64-bit program allows to put the whole country
into one run also in factory optimization. Hannu organized a video
meeting where we discussed the
development and utilization of the \textbf{J} software. He was the first person
in Lukes six years history who
was willing to discuss with me such things.

I started to work with \textbf{J} again. Reetta had told that \textbf{J} had could
collapse after some hours of computations in  large data sets in factory optimization.
I asked Luke to get the problematic
data sets so that I could try to fix the problem. I got the answer that Luke is not allowed to send the
data sets, and the answer did  not indicate
that Luke would do anything to get such permission.
I was quite upset from such answer,
and I started to think how to open the deadlock situation. Later I got the data,
and I was able to fix the problem, which was not exactly
a clear bug but rather a thin-ice place in the algorithm sot that \textbf{J} did drop
throug the ice when going over the place repeatedly.
\subsection{Operation Charles XII}
\label{preface6}
When I strarted to think how I could find a way out from the deadlock,
I remembered Charles XII,
the king of Sweden (wich included  also Finland). He was in war with Denmark
and he decided to fight Denmark by attacking Norway, which belonged to Denmark.
He failed as almost his solders died in snowstorms
in Northern Norway and he was shot by his own soldier.
I was in conflict with Luke, even if Luke did
acknowledge that such confilict existed
because Luke insisted that in a legal sense,
I did not even exist in this \textbf{J} affair.
I thought to conquer Norway with nice \textbf{J} codes.
Renovation of my co-operation with GAYA group
suited their plans as they were rebuilding their simulator.

I wrote to Johanna Buchert, director general of Luke (who was already in the group of
Luke leaders hustling in the publication of \textbf{J} when I was in Luke)
Antti Asikainen, research director of Luke,  and Kari T. Korhonen, who has
been involved in this \textbf{J} affair and was the superior of Reetta, and
told that I will not send to Luke new \textbf{J} codes including working
factory optimization codes without a written co-operation agreement.
I wrote also to registry-office and asked for the opening decision, and
to a lawyer of Luke and asked what kind of legal actions
Luke will take agaisnt me if I started to publish factory optimization
things with GAYA group using codes which I would not give to Luke.
The leaders did not comment in any way my co-operation proposal. The research director
just wrote that
all codes done in METLA are property of Luke. Buchert did not respond, but
the the Senior Vice President Ilkka P. Laurila answered that Luke regrets
for the delays in the publication of \textbf{J} caused by personell changes
(a funny expression for the fact that the publication process
could just start in 2020
when prof Packalen left Luke)
and
that Luke has started to prepare the publication and will publish the \textbf{J} on its
web pages. Kari T Korhonen did not respond.

Thus the leaders of Luke insisted that Luke will publish tens of thousands code
and manual lines I had written, and insisted that it has no reason
or obligation to discuss with
me what and in what form it will these lines publish.
It is evident that Luke had not been able to publish
\textbf{J} in my life time (I'm over 70 yrs and I have a heart disease).
But if I had been alive when Luke had
published \textbf{J} under my name, I had made a lawsuit againt Luke
for shaming my researcher honor
by publishing something I had not accepted for publication. Or perhaps
Luke had followed
the previous  custom of  Romania where all chemistry
research was published under the the name of Elena Ceaușescu,
and had published the the software and
manuals under the name of some equivalent to Elena.
Also in that case I had made a lawsuit
against Luke (if I had been alive). When the director general of Luke,
the research director of Luke, the Senior Vice President of Luke
and the lawyers of Luke
decided that Luke has absolutely no interest in
working factory optimization \textbf{J} codes or
other up to date \textbf{J} codes, did they discuss with any researchers?

After getting such a reponse to my co-opertation suggestion, it was evident that
I would not give Luke any new codes, and would require Luke to remove all codes I had
submitted after my retirement as Luke would not akckowledge that
it had got any codes from me after my retirement, and it would publish my codes
as if they were developed in Luke. So I would continue my Charles XII operation.
Unfortunately NMBU did not have factory and transportation cost data ready.  In Norway
the transportation costs can not be estimated as easily from coordinates as in Finland.

Laurila did send the decision to publish \textbf{J} as a open source software
after the lawyers of Luke
had considered two weeks whether Luke should do that. The decision was
signed by the research director Johanna Buchert 30.8. 2017.
The process of trying to
get permission to use Fletchers codes after his death was not described properly.
It was written
that after Fletchers death the university of Dundee did not
respond to queries of Luke. As decribed above, the things did not go that way.
When Luke was not satisfied with the corrected  permission sent
by the university, the university had probably decided that
it is vain to continue correspondece with Luke. However,
the university wanted to communicate with me,
when they asked my statement about the importance of Flethers
code in planning of natural resources.
The government wanted to figure out whether the
university is doing something useful. I think that my statement
was more valuable for
the university than the euros the university would loose when
not getting payments for the bqbd code.

The decision told that \textbf{J} is published under licence AGPLv3.
The licence is specially
planned for programs which provide computing services through the net.
Luke has not done anything to develop such net work services, as it has not
developed \textbf{J} in any way after my retirement in 2017.
The leaders of Luke did not discuss with
me the license matters, of cource not.

The decision nominated prof Packalen to take care of the launch of the software.
Unfortunately she could not organize the lauch, as \textbf{J} was not ready to publication
because prof Packalen did not allow Reetta to do the necessary
preparations with respect to the code and manuals.

The lawyers of Luke did not tell anything of the plans of Luke to make lawsuits
against me if I would co-operate with Norway and would not send new codes to Luke.

Lauri Meht\"atalo became Professor in Mathematical Modeling for Forest Planning
in Luke in Mai 2021.
Coming from an university environment, he is interested
in the development of science
and research. He wanted to get \textbf{J} finally published after 6 year deadlock.
Lauri did have anything againts that I would take care
of the publication. Lauri even thought that it
would be good for science and research that I would
work without payment. It was a refreshing new attitude in Luke
to put science into priority.
Lauri told that he would try to convince the leaders of Luke
that it would not harm Luke's interests that I would
take care of the publication of
the software I had made (for most part). Lauri told initially that
he wants to persuade
me to give all new codes to Luke. When I protested his choice of words,
he apologized, and used thereafter the word 'negotiate'. When Lauri started to 'negotiate' with
me, I consider that Luke first time in its history acknowledged that I do exist
as an legal entity is this \textbf{J} affair.

Lauri was able to rotate the head of the research director 180 degrees. The research director even told
that he would be interested to become a coauthor in the publication dealing the factory
optimization. It was, however, not mentally feasible for me, as
he did not even
answer my previous co-operation suggestion, but just told that Luke has everything
it needs. Being a researcher, Lauri understood what
were my aims and was able to get approval of the research director.
So it seemed that
the agreement was about ready. But then the agreement was
stuck on the table of the research director.
\subsection{Last battle, victory and peace in Olavinlinna}
\label{preface7}
I was invited to the 100 year anniversary celebration of the Finnish National Forest Inventory
as I had worked also in the NFI group. I was interested to see my former colleagues,
but it started to disturb me
that I would not know whether I would be a guest of my enemy
or my co-operations partner.
I had a very exciting video meeting with the GAYA people. They promised me access
to the Norwegian supercomputers needed in large factory optimizations, funding for traveling
and for any consulting work. I started to think that have been stupid
when I spent years of my short residual life waiting for such co-operation
with Luke. As Lauri had tried
to establish such co-operation between me and Luke,
I decided to give Luke a last chance.
I made an ultimatum that if we do not get a agreement with Luke within a week,
I would break all negotiations with Luke and would continue only with Norwegian people,
even if Luke might try to make a lawsuit against me.
Twenty-four hours before the time
limit Lauri called that Luke is willing to make treaty with me.
The lawyers of Luke did finalize the treaty
two hours before the NFI celebration.

Antti Asikainen, the research director of Luke and I signed the co-opertation treaty in the king's hall in
Olavinlinna (St. Olaf's Castle) in Savonlinna 7.10. 2021.
In the treaty I got almost everything I had
fighted for: The \textbf{J} software will be published, and I have
full responsibility of the code and manuals.
Luke will provide me data of factory locations, simulated treatment schedules for
all inventory plots for the whole country, and the matrix of distances between the
factories and inventory plots. Luke will also provide me access to CSC
supercomputers which are necessary for the whole country optimizations. CSC does not allow
independent retirees to use its computers. I'm allowed to make a methodological
paper from developments, whereafter I will provide Luke all the
new and improved \textbf{J} functions and the developed \textbf{J} scripts. I had hoped that
Luke had promised me access to Mela so I could study how Mela is doing interest
rate computations, as I have suspects that these computations are not in order.
I had asked Olli Salminen some Mela results to study my suspects, but he told, after some
day considerations, that Luke is not willing to make such computations.
I suspect that the reason is
Luke wants to keep all possible Mela problems secret. I did not insist to get
such permissions as Lauri thought that this would delay and complicat
the \textbf{J} process. In the treaty it
is said that the code files state that Juha Lappi
and Natural Resources Institute Finland have the copyright for the code.
In the last minutes
before the signatures
the lawyers tried to reverse the order of copyright owners. I did not, of course,
accept the reveresed order, bacause I have made the \textbf{J} really operable after my retirement,
Now, the recent developments have made Lukes or Metlas contribution
even smaller, and the overall influence of Luke has been negative.
It must, however to be taken into account that Pekka M\"akinen has done valuable
work in collecting data of factories and
transportation costs, and this data greatly helps my work when I'm now allowed
to start to develop factory optimization for the whole Finland.
\subsection{Total balance with Luke}
\label{preface8}
At the signing of the peace treaty, Johanna Buchert told that there has not been any
disagreement in this \textbf{J} busisness, and let bygones be bygones.
From my point of view, I had six and half years fight to get permission to
publish the results of my research.  For Luke, there was no
disagreement because Luke did not acknowledge that I even existed as a
legal entity. How could Luke
have disagreement with something which does not even exist? For me,
the behavior of the leaders of Luke was a 'me too' experience. In
the heart of my researcher
identity is the conviction that every researcher, at least in a government
research institute, has the right and even the obligation to publish the
results of his research let it be a stastistical method or computer code or
ordinary research publication.
The leaders of Luke denied
me the right to publish the results of my statistical work
in our biometry book and the results of my software development.  In this \textbf{J} case,
the reason was that they
wanted to publish my results themselves.  This has been a very traumatic
harrasment experience. All the victims of sexual harassment are also
told that 'let bygones be bygones'.
As in the peace treaty I was given all responsibility of manuals, I'm
now utilizing this responsibility in this preface hoping the leaders of
Luke and perhaps also others
would be more careful when making decisions which may not tolerate daylight,
similarly
as bosses are slowly learning to keep better control of
their hands which would like
to grope the bottoms of their subordinates.

From the legal perspective, the leaders of Luke have insisted that
Luke has full copyright to all work I've done in Mela or Luke. So they can freely
follow their whims and waste hundreds of thousands of euros
taxpayers money used in the \textbf{J} project.
With Lukes overhead multipliers we may speak about millions.
Perhaps they should also pay
attention to the by-law telling what is the duty of Luke. How has
Luke promoted forest research and its utilization by prevented
the development and
utilization of \textbf{J} software for over six years?
Perhaps the ministry should  give
the leaders of Luke training about the duties of Luke. The leaders
of Luke never gave any rational justification why Luke should prevent me
to take care of the publication.

If I had been allowed to publish and develop \textbf{J} as I wanted,
Luke and forest industry would have
now available methods to analyze e.g. what can be done
when timber import from Russia suddenly stops.
Luke is also still doing static regional sustainability analyzes,
which are not reasonable
as they do not take factories into account. Mentally Luke
is stuck to the German foresters from
1800's who rotated a fixed cutting area for each year.
If Luke had let me publish the open-source software in 2015
Luke had been a forerunner in the open source publication, and I had praised it
for its progressive attitude. Now Luke has
has establised its status in the history of open-source as an institute which
decided to open a software
but keeps the decision and the code locked for several years because
it did not allow the principal
author to take care of or even participate in the publication.
\subsection{Freedom}
\label{preface9}
After the Olavinlinna peace treaty, I thought initially that
I would like to publish the software before Christmas 2021.
But when the iron cage around me disappeared, the pressure which had increased
in my mind for seven years did burst in intensive code development.
This development had already started when negotiations with Lauri started to look promising,
and I developed the program to make Latex files for manuals and script files
for examples automatically from comments embedded within the code or from a
separate text files.

After the peace treaty I have rewritten all the central parts of the
software using simpler and more easily maintained solutions,
and added many new properties. The most important was the rewriting
of the interpreter
which interprets the \textbf{J} code generated using the input programming capabilities.
I had been shamed if the previous interpreter had be offered to the eyes of
professional programmers. I did rewrite the precompiler so it takes only some
seconds to do the precompilation, and if the precompiled file
does not change the new precompiled file is deleted so that the compiler
does not do any more unnecessary compilation. I did put the matrix computations
into a new level implementing
all properties I had found useful in Matlab when providing a method for a client
using Matlab codes. I added new tools for debugging
and code development. Now \textbf{J} can treat nicely submatrices and do
all arithmetic operations and logical operations
on matrices in addition to scalars. The previous version of  \textbf{J} could make rough
figures using a separate program written in Intel
fortran, which I do no more have available. Publication level
figures could be done using R sripts generated by J.
Now \textbf{J} is producing figures using Gnuplot which is automatically run from within J.
Currently only basic 2D and 3D figures
can be done, but this Gnuplot connection makes it possible to develop \textbf{J}
into full scale interface to all Gnuplot properties. \textbf{J} can now generate random numbers from any
continuous or discrete distibution providing new possiblities to study the
effects of random errors in the simulation of treatment schedules or in
optimization. It is now  possible to run R scripts from within J.
Now \textbf{J} can be used easily also as a subroutine. Lauri has utilized this property
by calling \textbf{J} from R. This offers many new possiblities,
which we will utilize when
continuing the co-operation with the GAYA-group.

I made first cleaning of the important \textcolor{VioletRed}{jlp}() function in the spring of 2021
when correcting
the fragility which prevented the solution of large problems which take several hours in
a ordinary PC. There are many things to be done in \textcolor{VioletRed}{jlp}() function, but I start
to work with them when developing factory optimization in CSC supercomputers.
I know already how parallel computations can be utilized in \textcolor{VioletRed}{jlp}()-function.
Previous versions of \textbf{J} had special functions for developing forest simulators and
computing the simulated schedules. Now all simulators can be developed and utilized
using standard properties of \textbf{J} transformations either for using stand level
or tree level variables
I have not yet made good examples to demonstrate these properties. Because the new \textbf{J} provides
a platform for making all kind of new devlopments, including
integer optimization (i.e preventing treatment unit to be divided among several
schedules), nonlinear objective function
(which would utlize the capability of \textbf{J} to compute derivatives
according to analytic derivation rules), it was difficult to breathe a little and make the
new codes available for users. The codes now published contain many errors.
I invite the users to help me to correct them. First users should have some
patience with respect to errors. I claime that the bug percentage is smaller than
estimated 23-27\% bug percentage in the first version
of Windows 1, which accintally happened be the same as the bug percentage
Soviet masons used in the mortar they used when masoning the US embassy in Moscow.
Currently I can correct most bugs
causing system crash in 5-10 minutes. It should be emphasized that
Luke does not have any kind of responsibility of the errors.
I take the full moral responsibilty of the errors. Similarly, it should
be emphasized that Luke has no responsibility of the code which works correctly.
I take the main responsibility of the correctly working code,
giving Reetta her fair
share. It is easier for me to correct the errors while alive.
This fundamental fact of life
was ignored by Luke when it prevented the publication for seven years.

During the intensive work period after the Olavinlinna peace treaty, my deep
'me too' traumas have started to heal slowly. Probably this healing process takes
a long time, and I cannot very soon say let bygones be bygones.

\textbf{J}uha Lappi, Suonenjoki 22.4.2022

\begin{thebibliography}{9}

\bibitem{dan}
Dantzig, G.B. and VanSlyke, R.M.  (1967)
\emph{Generalized upper bounding techniques}
\textbf{J} Compt Sys Sci 1(10),213-226

\bibitem{flet}Fletcher,R.  1996. Dense factors of Sparse matrices. Dundee
Numerical Analysis Report NA/170.


\bibitem{howard},
Hoganson, H.M. and Rose, D.W.  (1984),
\emph{A simulation approach for optimal timber management scheduling}
Forest Science, 30:220-238

\bibitem{dtran}Hoganson, H.M. and Kapple,  D.C. (1991),
\emph{DTRAN version 1.0. A multi-market timber supply model. Users’ guide}
Minneapolis: University of Minnesota Department of Forest
Resources Staff Series Paper 82,

\bibitem{pekka}
Hyv\"onen, Pekka, Lempinen, Reetta,
Lappi, Juha, Laitila, Juha  and Packalen, Tuula (2019)
\emph{Joining up optimisation of wood supply chains with forest},
Forestry an international journal of forestry,
93(1):163--177,
DOI = https://doi.org/10.1093/forestry/cpz058

\bibitem{JLP} Lappi, Juha (1992) \emph{JLP – a linear programming package for
management planning} Finnish Forest Research Institute
Research papers; 414, 134 p.



\bibitem{LL},
Lappi, Juha and Lempinen, Reetta (2014)
\emph{A linear programming algorithm and software
for forest-level planning problems
including factories}
Scandinavian Journal of Forest Research,29 Supplement 178–-184",

DOI =  http://dx.doi.org/10.1080/02827581.2014.886714






\end{thebibliography}
